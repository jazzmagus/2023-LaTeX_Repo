\documentclass[10pt, a4paper twoside, notitlepage, notoc, justified]{tufte-handout}
\usepackage{lipsum}
\usepackage[T1]{fontenc}
%\usepackage[utf8, latin1]{inputenc}
\usepackage[italian]{babel}
\usepackage{multicol}
\usepackage{xcolor}
\usepackage{caption}
\captionsetup{justification   = raggedright,
              singlelinecheck = false}
\setlength{\columnsep}{1cm}
%\usepackage[margin=1in]{geometry}
\usepackage{amsfonts, amsmath, amssymb}
\usepackage[none]{hyphenat}
\usepackage{fancyhdr}
\usepackage[Glenn]{fncychap}
\usepackage{tcolorbox}
\usepackage{graphics}
\usepackage{float}
\usepackage[nottoc, notlot, notlof]{tocbibind}
\usepackage{pgf,tikz,pgfplots} 
\pgfplotsset{compat=1.15}
\usepackage{mathrsfs}
\usetikzlibrary{arrows}


\setlength{\columnseprule}{0.5pt}
\def\columnseprulecolor{\color{black}}

\pagestyle{fancy}
\fancyhead{}
\fancyfoot{}
\fancyhead[L]{\small {\MakeUppercase {ISS "G. A. Remondini"} - Bassano del Grappa}}
\fancyhead[R]{\small \emph{Anno Scolastico 2020-21}}
\fancyfoot[C]{\thepage}
\renewcommand{\headrulewidth}{0.15pt}
\renewcommand{\footrulewidth}{0.1pt}

\parindent 0ex
%\setlength{\parindent}{2em}
%\setlength{\parskip}{1em}
%\renewcommand{\baselinestretch}{1.5}

\newcommand{\numberset}{\mathbb}
\newcommand{\N}{\numberset{N}}
\newcommand{\Z}{\numberset{Z}}
\newcommand{\Q}{\numberset{Q}}
\newcommand{\R}{\numberset{R}}
%%%%%%%%%%%%%%%%%%%%%%%%%%%%%%%%%%%%%%%%%%%% title page

\begin{document}
\noindent
{\LARGE{\textbf{PROGRAMMAZIONE CLASSI - MATEMATICA}}}\\[3mm]
%\begin{titlepage}
%\begin{center}
%\vspace{1cm}
{\textit {\small {\em prof.:} Diego Fantinelli}}\\[4mm]
{\textbf{Anno Scolastico 2020-21}}\\
%\vfill
%\line(1,0){400}\\[.5mm]
%{\Huge{\textbf{Diario di Bordo 2021}}}
%\Large{\textbf{- Sottotitolo:  -}}\\[1mm]
%\abstract{spiegazione: il presente documento serve da traccia per la Programmazione Didattica}
\line(1,0){400}\\
\begin{abstract}
	\lipsum[2]
\end{abstract}

\tableofcontents
%\vfill
%{\scriptsize By Student Name}\\
%{\scriptsize Candidate \#} \\
%{\scriptsize \today} \\
%\end{center}
%
%\end{titlepage}

%------------------------------------- ESEMPIO Evento
%\begin{loggentry}{30.09.2020}{\em Presentazioni e ripasso}\\
%Ad un primo approccio la classe pare collaborativa e sufficientemente silenziosa.
%Il rispetto delle prescrizioni Covid-19
%\end{loggentry}
%------------------------------------- introduzione

%\thispagestyle{empty}
%\clearpage

\setcounter{page}{1}

\newenvironment{loggentry}[2]% date, heading
{\noindent\textbf{#2}\marginnote{#1}\\}{\vspace{0.5cm}}

%------------------------------------- LA CLASSE 2^C
\newpage
\section{La Classe 2C}
\marginpar {{\em Coordinatore di Classe:}\\ Prof.ssa Mongelli}

\newthought{osservazioni}\\ 
La Classe si presenta piuttosto numerosa {\em (27 alunni)}, ma piuttosto silenziosa - eccezion fatta per un piccolo gruppo da tenere sotto controllo -.\\ 

$$
\begin{array}{|l|l|l|l|l|}
\hline \text {\bf n.} & \text {\bf cognome } & \text {\bf nome } & \text {\bf email } & \text {\bf note }\\
\hline \text {\bf 1 } & \text { Alberti } & \text { Leonardo } & \text { s8443999i@remondini.net } & \text {\emph {\footnotesize Brevissime note essenziali, comportamento}}\\
\hline \text {\bf 2* } & \text { Aslam } & \text { Daniyal } & \text { s8443979e@remondini.net } & \\
\hline \text {\bf 3 } & \text { Baldin } & \text { Eleonora } & \text { s8444239a@remondini.net } & \\
\hline \text {\bf 4 } & \text { Boafo } & \text { Rexford } & \text { s8444247a@remondini.net } & \text {\emph {\footnotesize BES Linguistico}}\\
\hline \text {\bf 5 } & \text { Canova } & \text { Jacopo } & \text { s8233910n@remondini.net } & \text {\emph {\footnotesize R - BES ADHD, deficit attenzione}}\\
\hline \text {\bf 6 } & \text { Caon } & \text { Leonardo } & \text { s8444020b@remondini.net } & \text {\emph {\emph {\footnotesize R}}}\\
\hline \text {\bf 7 } & \text { Caspon } & \text { Edward } & \text { s8444016h@remondini.net } & \text {\footnotesize DSA}\\
\hline \text {\bf 8 } & \text { Cavalli } & \text { Susy } & \text { s8444188k@remondini.net } & \\
\hline \text {\bf 9 } & \text { Cristofari } & \text { Giorgia } & \text { s8502603y@remondini.net } & \\
\hline \text {\bf 10 } & \text { Dannoun } & \text { Amal } & \text { s7798946p@remondini.net } & \text {\footnotesize DSA}\\
\hline \text {\bf 11* } & \text { Dedoro } & \text { Nicole } & \text { s8444186l@remondini.net } & \text {\emph {\footnotesize Brevissime note essenziali, comportamento}}\\
\hline \text {\bf 12 } & \text { Descube } & \text { Beatrice } & \text { s7799179u@remondini.net } & \\
\hline \text {\bf 13 } & \text { Fraccaro } & \text { Sarah } & \text { s8444168s@remondini.net } & \\
\hline \text {\bf 14 } & \text { Hanouni } & \text { Jihad } & \text { s8444000n@remondini.net } & \\
\hline \text {\bf 15 } & \text { Lollato } & \text { Giulia } & \text { s8444143g@remondini.net } & \text {\emph {\footnotesize Tutoring }}\\
\hline \text {\bf 16 } & \text { Lunardon } & \text { Irene } & \text { s8444140u@remondini.net } & \text {\footnotesize DSA}\\
\hline \text {\bf 17 } & \text { Maggiolo } & \text { Anita } & \text { s8444135s@remondini.net } & \text {\footnotesize DSA}\\
\hline \text {\bf 18 } & \text { Militello } & \text { Giuseppe } & \text { s8443996x@remondini.net } & \text {\emph {\footnotesize Tutoring }}\\
\hline \text {\bf 19 } & \text { Operti } & \text { Chiara Maria } & \text { s8444101a@remondini.net } & \text {\emph {\footnotesize Tutoring - DSA}}\\
\hline \text {\bf 20 } & \text { Prandina } & \text { Lucia } & \text { s8444103g@remondini.net } & \\
\hline \text {\bf 21 } & \text { Pupaza } & \text { Alex Vladut } & \text { s8443947g@remondini.net } & \text {\emph {\footnotesize Ha frequentato il "Da Ponte" per un breve periodo}}\\
\hline \text {\bf 22 } & \text { Sefedin } & \text { Jetka } & \text { s8444074f@remondini.net } & \text {\emph {\footnotesize R}}\\
\hline \text {\bf 23 } & \text { Spagnolo } & \text { Giulia } & \text { s8444066a@remondini.net } & \\
\hline \text {\bf 24 } & \text { Stegarescu } & \text { Elisa Rebecca } & \text { s8443953w@remondini.net } & \text {\emph {\footnotesize R - BES ADHD, deficit attenzione}}\\
\hline \text {\bf 25 } & \text { Tafa } & \text { Zyra } & \text { s8444259o@remondini.net } & {\emph {\footnotesize BES Linguistico}}\\
\hline \text {\bf 26 } & \text { Tekyi } & \text { Emmanuela } & \text { s8444061v@remondini.net } & \text {\footnotesize DSA}\\
\hline \text {\bf 27 } & \text { Zanchetta } & \text { Eyasu } & \text { s8443965m@remondini.net } & {\emph {\footnotesize Brevissime note essenziali, comportamento}}\\
\hline
\end{array}
$$

\newthought{*} Certificazione DSA, L.104
%\newthought{Studenti in tutorship}\\ LOLLATO Giulia, MILITELLO Giuseppe; Chiara Maria OPERTI

%------------------------------------- LA CLASSE

\newpage
\section{Tutorship}
\newthought{Studenti in tutorship}\\ LOLLATO Giulia, MILITELLO Giuseppe; Chiara Maria OPERTI

\subsection{\small \bf Scheda Funzione Tutor}

Per ogni alunno è prevista la nomina di un docente (del CdC) che lo segue con attenzione non solo da un punto di vista didattico ma emotivo - relazionale, cioè segue la persona.
Questa è la figura del docente tutor, un coach...un mentore, istituita dalla riforma per gli alunni del professionale che appaiono, nel contesto scolastico, come i più fragili e maggiormente esposti al rischio abbandono scolastico.

\subsection{\small \bf Requisiti Minimi}

		\begin{itemize}
			\item Essere docente dell'Istituto
			\item Essere membro del Consiglio di Classe
			\item Avere capacità di relazione e comunicativa
			\item Avere capacità di attivare con lo studente un rapporto stabile incontrandolo periodicamente
		\end{itemize}
		
\subsection{\small \bf Principali mansioni}
\begin{itemize}
\item	Raccogliere informazioni relative al profitto e al comportamento o ai problemi incontrati dallo studente nei rapporti con la classe, con i docenti e rispetto agli impegni scolastici
\item Fornire allo studente suggerimenti in ordine al metodo di studio o alle relazioni personali per superare gli ostacoli che possono caratterizzare il suo andamento scolastico
\item Collaborare con il Consiglio di Classe fornendo informazioni e richiedendo, se necessario, incontri con la famiglia 
\item Informare il coordinatore sulla eventuale necessità di riorientare lo studente

\end{itemize}



%------------------------------------- LA CLASSE

\newpage
\section{Programmazione Didattica 2 C}
\subsection{\small \bf MODULO 1 - Ripasso Fattorizzazione}

\begin{loggentry}{Settembre - Ottobre}{\em Insiemi Numerici}
Integer sapien est, iaculis in, pretium quis, viverra ac, nunc. Praesent eget sem vel leo ultrices bibendum. Mauris ut leo. Cras viverra metus rhoncus sem.

\marginpar {\footnotesize {{\em competenze:}\\ $A_1$ - Utilizzare le tecniche e le procedure del calcolo aritmetico rappresentandole anche sotto forma grafica.}}
\begin{multicols}{2}
{\small
	\begin{enumerate}
		\item {\bf Conoscenze:}
		\begin{itemize}
			\item Frazioni algebriche:\\ semplificazione
			\item Equazioni numeriche di primo grado fratte.
			\item Tecniche risolutive di un problema, anche utilizzando equazioni di primo grado.
			\end{itemize}
		\item {\bf Abilità:}
		\begin{itemize}
			\item Saper fattorizzare un polinomio
			\item Utilizzo dell'algebra per risolvere problemi numerici e algebrici
			\end{itemize}
	\end{enumerate} 

\columnbreak

\begin{enumerate}
		\item {\bf Conoscenze:}
		\begin{itemize}
			\item Frazioni algebriche:\\ semplificazione
			\item Equazioni numeriche di primo grado fratte.
			\item Tecniche risolutive di un problema, anche utilizzando equazioni di primo grado.
			\end{itemize}
		\item {\bf Abilità:}
		\begin{itemize}
			\item Saper fattorizzare un polinomio
			\item Utilizzo dell'algebra per risolvere problemi numerici e algebrici
			\end{itemize}
	\end{enumerate} 
}
\end{multicols}
\newthought{nota 1}\\ La programmazione subirà uno slittamento a causa di un indispensabile ripasso: hanno rimosso tutto ciò che avevano eventualmente acquisito lo scorso anno scolastico. 
\newthought{osservazioni}\\ Programmare la Verifica di ripasso sulla Fattorizzazione, per la IV settimana di Ottobre.
\end{loggentry}

\newpage
\subsection{\small \bf MODULO 2: Calcolo Letterale}

\begin{loggentry}{Settembre - Ottobre}{\em Insiemi Numerici}
Integer sapien est, iaculis in, pretium quis, viverra ac, nunc. Praesent eget sem vel leo ultrices bibendum. Mauris ut leo. Cras viverra metus rhoncus sem.

\marginpar {\footnotesize {{\em competenze:}\\ $A_1$ - Utilizzare le tecniche e le procedure del calcolo aritmetico rappresentandole anche sotto forma grafica.}}
\begin{multicols}{2}
{\small
	\begin{enumerate}
		\item {\bf Conoscenze:}
		\begin{itemize}
			\item Frazioni algebriche:\\ semplificazione
			\item Equazioni numeriche di primo grado fratte.
			\item Tecniche risolutive di un problema, anche utilizzando equazioni di primo grado.
			\end{itemize}
		\item {\bf Abilità:}
		\begin{itemize}
			\item Saper fattorizzare un polinomio
			\item Utilizzo dell'algebra per risolvere problemi numerici e algebrici
			\end{itemize}
	\end{enumerate} 

\columnbreak

\begin{enumerate}
		\item {\bf Conoscenze:}
		\begin{itemize}
			\item Frazioni algebriche:\\ semplificazione
			\item Equazioni numeriche di primo grado fratte.
			\item Tecniche risolutive di un problema, anche utilizzando equazioni di primo grado.
		\end{itemize}
		\item {\bf Abilità:}
		\begin{itemize}
			\item Saper fattorizzare un Polinomio
			\item Utilizzo dell'algebra per risolvere problemi numerici e algebrici
		\end{itemize}
	\end{enumerate} 
}
\end{multicols}
\newthought{nota 1}\\ Nulla et lectus vestibulum ur na fringilla ultrices. 
\newthought{osservazioni}\\ Nulla et lectus vestibulum ur na fringilla ultrices. ellus eu tellus.
\end{loggentry}

%------------------------------------- Lessons
\newpage
\section{\small \bf LEZIONI}
\subsection{\small \bf {\em MODULO 1: Ripasso Fattorizzazione}}
\begin{loggentry}{mer 30.09.2020}{\em Presentazioni e ripasso}\\
Ad un primo approccio la classe pare collaborativa e sufficientemente silenziosa.
Il rispetto delle prescrizioni Covid-19
\end{loggentry}

\begin{loggentry}{ven 02.10.2020}{\em Presentazione Programma}\\
salvo poi ricredermi, il giorno dopo. Testa di serie Dedoro Nicole (RdC), che con altri 3/4, movimentano/disturbano parecchio le lezioni.
\end{loggentry}

\begin{loggentry}{mer 07.10.2020}{\em Inizio Ripasso: Monomi e Polinomi}\\
{\em Intro: la precisione in matematica; il calcolo letterale e le imprecisioni.}\\
"Tutto apposto"; "un'altro"; "perchè";\\ sono tutti esempi di un linguaggio di comunicazione - la lingua italiana appunto - che richiede precisione, come nella matematica e in altri linguaggi: non si può quindi sostenere che gli altri linguaggi non richiedano precisione. 
\begin{enumerate}
			\item Il calcolo letterale: perchè usiamo le lettere? 
			\item I Polinomi.
			\item Le operazioni con {\em monomi e polinomi}.
		\end{enumerate}
\end{loggentry}

\begin{loggentry}{ven 09.10.2020}{\em Intro: Calcolo Letterale: perchè fattorizzare un Polinomio?}\\
Scomporre/Fattorizzare un Polinomio: qual è l'utilità?
\end{loggentry}
\newthought{osservazioni}\\ La gestione della lezione con i gruppi {\em "a distanza"} è piuttosto complicato\\
\newthought{nota 1}\\ Il livello di attenzione sfiora i 15'.\\

\begin{loggentry}{mer 14.10.2020}{\em Fattorizzazione}\\
{\em Intro: la precisione in matematica; il calcolo letterale e le imprecisioni.}\\
"Tutto apposto"; "un'altro"; "perchè" sono tutti esempi di un linguaggio di comunicazione - la lingua italiana appunto - che richiede precisione, come nella matematica e in altri linguaggi: non si può quindi sostenere che gli altri linguaggi non richiedano precisione. 
\begin{enumerate}
			\item Come e perchè si {\em fattorizza} un Polinomio? 
			\item I metodi:
			\begin{enumerate}
				\item {\em Raccoglimento TOTALE}
				\item {\em Raccoglimento PARZIALE}
				\item {\em Prodotti Notevoli}
				\item {\em Trinomio "particolare"}
				\item {\em Trinomio di secondo grado}
				\item {\em Regola di Ruffini:}
				\begin{itemize}
					\item Teorema del Resto
					\item Divisione tra Polinomi
					\item Teorema di Ruffini
				\end{itemize}
			\end{enumerate}
		\end{enumerate}
\end{loggentry}

\newthought{nota 1}\\ Si sono visti solo alcuni esempi di Prodotti Notevoli, un paio di esercizi:\\ La classe è molto arrugginita.
\begin{itemize}
	\item $(2x-1)^2$;
	\item $(x+1) \cdot (x-1)$;
	\item $4x^2-4x+1 = (2x-1)^2$.
\end{itemize}

\newthought{Esercizi assegnati:}\\ Es. dal 12 al 15, e 30 a pag. 15; Es. n. 56, 57 e 72 a pag. 17.\\

\begin{loggentry}{ven 16.10.2020}{\em Fattorizzazione}\\
Riprendere la lezione di mer 14.10.2020;
\end{loggentry}

\begin{loggentry}{ven 13.11.2020}{\em Verifica}\\
Verifica di ripasso sul prgramma dell'anno scorso: 
\begin{enumerate}
	\item Lacune enormi sulle proprietà del calcolo letterale
	\item Prodotti Notevoli e Fattorizzazione praticamente sconosciuti;
	\item Verifica di Recupero da fissare per mercoledì prossimo: 25/11/2020
	\item Prevedere lezioni più toste e, soprattutto, un lavoro a casa in autonomia che va controllato, periodicamente.
	\item Assegnare molti più compiti per casa tramite Classroom e controllare sempre i lavori - dovranno essere caricati su Classroom.
	\item 
\end{enumerate}
\end{loggentry}



\begin{loggentry}{ven 25.11.2020}{\em Verifica di Recupero}\\
Verifica di ripasso sul prgramma dell'anno scorso: 
\begin{enumerate}
	\item Lacune enormi sulle proprietà del calcolo letterale
	\item Prodotti Notevoli e Fattorizzazione praticamente sconosciuti;
	\item Verifica di Recupero da fissare per mercoledè prossimo: 25/11/2020
	\item Prevedere lezioni più toste e, soprattutto, un lavoro a casa in autonomia che va controllato, periodicamente.
	\item Assegnare molti più compiti per casa tramite Classroom e controllare sempre i lavori - dovranno essere caricati su Classroom.
	\item 
\end{enumerate}
\end{loggentry}

%------------------------------------- CLASSE 3^Qa - COMM.LE SER -------------------------

\newpage
\section{La Classe 3QA - Servizi Commerciali Serali}
\marginpar {{\em Coordinatore di Classe:}\\ Prof.ssa M. Rebecchi}
La classe si presenta molto eterogenea in termini di preparazione e competenze sulla specifica materia, ma già sufficientemente coesa da permettere un'azione didattica efficace.\\
Si tratta di persone adulte che, per la maggior parte, lavora.\\
Il problema principale sembra essere il carico di lavoro che la classe è in grado di reggere: è prossimo allo zero.

$$
\begin{array}{|l|l|l|l|l|}
\hline \text {\bf n.} & \text {\bf cognome } & \text {\bf nome } & \text {\bf email } & \text {\bf note }\\
\hline \text {\bf 1 } & \text { Arias Moreno } & \text { Freudy Josue } & \text { @remondini.net } & \text {\emph {\footnotesize Brevissime note essenziali, comportamento}}\\
\hline \text {\bf 2 } & \text { Bellini } & \text { Daniela } & \text { @remondini.net } & \\
\hline \text {\bf 3 } & \text { Bonato } & \text { Simone } & \text { @remondini.net } & \\
\hline \text {\bf 4 } & \text { Bortoli } & \text { Veronica } & \text { @remondini.net } & \\
\hline \text {\bf 5 } & \text { Dalle Nogare } & \text { Filippo } & \text { @remondini.net } & \\
\hline \text {\bf 6 } & \text { Faggion } & \text { Ilaria } & \text { @remondini.net } & \\
\hline \text {\bf 7 } & \text { Fantinelli } & \text { Matteo } & \text { @remondini.net } & \\
\hline \text {\bf 8 } & \text { Furlan } & \text { Giulia } & \text { @remondini.net } & \\
\hline \text {\bf 9 *} & \text { Laqnissi } & \text { Linda } & \text { @remondini.net } & \\
\hline \text {\bf 10 } & \text { Mascarello } & \text { Elia } & \text { @remondini.net } & \\
\hline \text {\bf 10 } & \text { Pegoraro } & \text { Marco } & \text { @remondini.net } & \\
\hline \text {\bf 11 } & \text { Pizzato } & \text { Chiara } & \text { @remondini.net } & \text {\emph {\footnotesize Brevissime note essenziali, comportamento}}\\
\hline \text {\bf 12 } & \text { Proietti } & \text { Valentina } & \text { @remondini.net } & \\
\hline \text {\bf 13 } & \text { Tommasini } & \text { Valentina } & \text { @remondini.net } & \\
\hline \text {\bf 14 } & \text { Vladoiu } & \text { Elena Diana } & \text { @remondini.net } & \\
\hline \text {\bf 15 } & \text { Zanon } & \text { Angelica } & \text { @remondini.net } & \\
\hline
\end{array}
$$

\newthought{nota 1}\\ Nulla et lectus vestibulum ur na fringilla ultrices. 
\newthought{osservazioni}\\ Nulla et lectus vestibulum ur na fringilla ultrices. ellus eu tellus

%------------------------------------- LA CLASSE
\newpage
\section{Programmazione Didattica 3 QA}
\subsection{\small \bf MODULO 1 - Gli Insiemi Numerici}

\begin{loggentry}{Settembre - Ottobre}{\em Insiemi Numerici}
Integer sapien est, iaculis in, pretium quis, viverra ac, nunc. Praesent eget sem vel leo ultrices bibendum. Mauris ut leo. Cras viverra metus rhoncus sem.

\marginpar {\footnotesize {\em competenze:}\\This is a margin note using the geometry package, set at 
3cm vertical offset to the line it is typeseted}
\begin{multicols}{2}
{\small
	\begin{enumerate}
		\item {\bf Conoscenze:}
		\begin{itemize}
			\item Frazioni algebriche: semplificazione
			\item Equazioni numeriche di primo grado fratte.
			\item Tecniche risolutive di un problema, anche utilizzando equazioni di primo grado.
			\end{itemize}
		\item {\bf Abilità:}\\ Integer sapien est, iaculis in, pretium quis, viverra ac, nunc. Praesent eget sem vel leo ultrices bibendum. Mauris ut leo. Cras viverra metus rhoncus sem.
		\item {\bf Attenzione:}\\ Integer sapien est, iaculis in, pretium quis, viverra ac, nunc. Praesent eget sem vel leo ultrices bibendum. Mauris ut leo. Cras viverra metus rhoncus sem.
	\end{enumerate} 
\columnbreak

\begin{enumerate}
		\item {\bf Conoscenze:}\\ Frazioni algebriche: semplificazione
Equazioni numeriche di primo grado fratte.
Tecniche risolutive di un problema, anche utilizzando equazioni di primo grado
		\item {\bf Abilità:}\\ Integer sapien est, iaculis in, pretium quis, viverra ac, nunc. Praesent eget sem vel leo ultrices bibendum. Mauris ut leo. Cras viverra metus rhoncus sem.
		\item {\bf Attenzione:}\\ Integer sapien est, iaculis in, pretium quis, viverra ac, nunc. Praesent eget sem vel leo ultrices bibendum. Mauris ut leo. Cras viverra metus rhoncus sem.
\end{enumerate}
}
\end{multicols}
\newthought{nota 1}\\ Nulla et lectus vestibulum ur na fringilla ultrices. \newthought{nota 2}\\ Nulla et lectus vestibulum ur na fringilla ultrices. ellus eu tellus sit amet tortor gravida placerat. 
\newthought{osservazioni}\\ Nulla et lectus vestibulum ur na fringilla ultrices. ellus eu tellus.
\end{loggentry}

\newpage
\subsection{\small \bf MODULO 2: Calcolo Letterale}

\begin{loggentry}{Settembre - Ottobre}{\em Insiemi Numerici}
Integer sapien est, iaculis in, pretium quis, viverra ac, nunc. Praesent eget sem vel leo ultrices bibendum. Mauris ut leo. Cras viverra metus rhoncus sem.

\marginpar{{\em competenze:}\\This is a margin note using the geometry package, set at 
3cm vertical offset to the line it is typeseted}
\begin{multicols}{2}
{\small
	\begin{enumerate}
		\item {\bf Conoscenze:}\\ Integer sapien est, iaculis in, pretium quis, viverra ac, nunc. Praesent eget sem vel leo ultrices bibendum. Mauris ut leo. Cras viverra metus rhoncus sem.
		\item {\bf Abilità:}\\ Integer sapien est, iaculis in, pretium quis, viverra ac, nunc. Praesent eget sem vel leo ultrices bibendum. Mauris ut leo. Cras viverra metus rhoncus sem.
		\item {\bf Attenzione:}\\ Integer sapien est, iaculis in, pretium quis, viverra ac, nunc. Praesent eget sem vel leo ultrices bibendum. Mauris ut leo. Cras viverra metus rhoncus sem.
	\end{enumerate} 
\columnbreak

\begin{enumerate}
		\item {\em Lezione:}\\ Integer sapien est, iaculis in, pretium quis, viverra ac, nunc. Praesent eget sem vel leo ultrices bibendum. Mauris ut leo. Cras viverra metus rhoncus sem.
		\item {\em Comportamento:}\\ Integer sapien est, iaculis in, pretium quis, viverra ac, nunc. Praesent eget sem vel leo ultrices bibendum. Mauris ut leo. Cras viverra metus rhoncus sem.
	\item {\em Attenzione:}\\ Integer sapien est, iaculis in, pretium quis, viverra ac, nunc. Praesent eget sem vel leo ultrices bibendum. Mauris ut leo. Cras viverra metus rhoncus sem.
	\end{enumerate}
	}
\end{multicols}
\newthought{nota 1}\\ Programmare verifica di fine ottobre. 
\newthought{nota 2}\\ Nulla et lectus vestibulum ur na fringilla ultrices. ellus eu tellus sit amet tortor gravida placerat. 
\newthought{osservazioni}\\ Nulla et lectus vestibulum ur na fringilla ultrices. ellus eu tellus.
\end{loggentry}

%------------------------------------- Lessons
\newpage
\section{\small \bf LEZIONI}
\subsection{\small \bf MODULO 1: Gli Insiemi Numerici}
\begin{loggentry}{gio | 01.10.2020}{\em Presentazioni e ripasso}\\
Trattandosi di una classe di adulti, gli studenti sono sufficientemente responsabili da non presentare ostacoli/rallentamenti allo svolgimento della programmazione didattica.\\Vengono rispettate le prescrizioni Covid-19
\end{loggentry}

\begin{loggentry}{ven | 03.10.2020 - DAD}{\em Presentazioni e presa visione Programma}\\
La lezione a distanza ha visto tutta la classe presente.\\
E' indispensabile adottare soluzioni che permettano la verifica della partecipazione.\\
\end{loggentry}

\begin{loggentry}{gio | 08.10.2020 }{\em Inizio Programmazione: I Numeri Naturali: $\N$}
\begin{enumerate}
	\item Intro agli Insiemi Numerici: $\N$, $\Z$, $\Q$, $\R$.
	\item L'Insieme $\N$ dei  Numeri Naturali
	\item Le Operazioni
\end{enumerate}
\end{loggentry}

\begin{loggentry}{ven | 09.10.2020 - DAD}{\em Esercizi sui num. naturali}\\
\marginpar {\footnotesize {\em proposta alle classe:}\\ modalità "3x1":\\ 3 controlli quaderno = 1 voto Orale}
La lezione a distanza ha visto quasi tutta la classe presente: solo due assenti.
\begin{itemize}
	\item es. a pag. 36 n. 50, 51, 52, 53.
\end{itemize}
\end{loggentry}

\begin{loggentry}{gio | 15.10.2020 }{\color{red}{\em > I Numeri Naturali: $\N$}}
\begin{enumerate}
	\item Le operazioni in $\N$.
	\item Massimo Comune Divisore e minimo comune multiplo in $\N$
	\item Scomporre un Numero Naturale
	\item I Numeri Primi
	\item Elevamento a potenza: proprietà delle {\em potenze}
	\item Radicali: proprietà dei {\em radicali}
\end{enumerate}
\end{loggentry}

\begin{loggentry}{ven | 16.10.2020 - DAD}{\em Esercizi sui num. naturali}\\
La lezione a distanza ha visto quasi tutta la classe presente: solo due assenti.
\begin{itemize}
	\item es. 
	\item 2
	\item 3 
\end{itemize}
\end{loggentry}
%=======================================================================
%
%
%%------------------------------------- DICEMBRE 2020
%
%
%%------------------------------------- GENNAIO 2021
%
%
%%------------------------------------- FEBBRAIO 2021
%
%
%%------------------------------------- MARZO 2021
%
%
%%------------------------------------- APRILE 2021
%
%
%%------------------------------------- MAGGIO 2021
%
%
%%------------------------------------- GIUGNO 2021
\end{document}