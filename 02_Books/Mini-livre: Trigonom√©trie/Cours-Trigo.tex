\documentclass{article}
\usepackage[utf8]{inputenc}
\usepackage[]{pan2}
\usepackage[marginratio=2:1,textwidth=16.5cm,textheight=25.4cm,marginparwidth=0.8cm,headheight=2cm]{geometry}
\usepackage{pstricks-add} 
\usepackage{color}
\usepackage{yhmath}%pour wideparen
\usepackage[colorlinks]{hyperref}
\newenvironment{exemple}{\par\medskip\noindent\textbf{Exemple: }}{\par\smallskip}
\begin{document}


\section{Enroulement de la droite sur le cercle.}
\noindent\begin{minipage}{\textwidth - 3.5cm}
\begin{Def}
   Le \emph{radian} est, comme le degré ou le grade, une unité de mesure d'angles définie de la façon suivante :\\
   Le radian est tel que si l'arc $\wideparen{MN}$ d'un cerle de rayon $R$ a pour longueur $R$, alors l'angle $\widehat{MON}$ vaut $1$ radian. \end{Def}

\begin{rem}
   La mesure en degrés celle en radians sont proportionnelles:\\ $360$ degrés $\leftrightarrow 2\pi$ radians ou encore $180$ degrés $\leftrightarrow \pi$ radians
\end{rem}
\paragraph{Conversion} degré / radians.
\end{minipage}
\hfill
\begin{minipage}{3cm}
\begin{center}
\scalebox{0.75}{
   \begin{pspicture}(2,2)(-2,-2)
      \pscircle(0,0){2}
      \psline{->}(0,0)(2,0)
      \rput(2.5,0){$M$}
      \psline{->}(0,0)(1,1.7)
      \rput(1.2,2){$N$}
      \psarc[linecolor=blue]{-}(0,0){0.8}{0}{60}
      \rput(1,-0.4){$R$}
      \rput(1.2,0.7){\textcolor{blue}{$1$ radian}}
      \rput(-0.2,-0.2){$0$}
      \psarc[linewidth=0.1,linecolor=red]{-}(0,0){2}{0}{60}
      \rput(2.2,1){\textcolor{red}{$R$}}
   \end{pspicture}}
\end{center}
\end{minipage}



  % \begin{center}
 \noindent \renewcommand{\arraystretch}{1.8}%   
\begin{tabular}{|c|*9{c|}}
         \hline 
         Mesure en degrés & $360$ &$180$& $120$ & $90$ & $60$ &  & $30$ & &\\ 
         \hline 
         Mesure en radians & $2\pi$ &\phantom{$\sum_.^{1}$}  &  &  & &$\dfrac{\pi}{4}$ &  & $\dfrac{7\pi}{4}$& $\dfrac{11\pi}{12}$\\ 
         \hline 
      \end{tabular} 
  % \end{center}


\subsection*{Cercle trigonométrique}%____________________________
\noindent\begin{minipage}{\textwidth - 7cm}
\begin{Def}     Le \emph{cercle trigonométrique} est le cercle $(\mathscr{C})$ de
        centre $O$, de rayon $1$ muni d'un sens de rotation,
        \emph{i.e.} orienté de telle sorte que le sens positif (ou direct, ou trigonométrique) est
        celui du sens inverse de rotation des aiguilles d'une montre.
   \end{Def}
       
\begin{exemple}
Placer des mesures positives puis négatives en radian sur le cercle trigonométrique.
\end{exemple}
\textbf{Ainsi tout nombre réel $x$ correspond à un point M du cercle.} 
\end{minipage}
\hfill
\begin{minipage}{5cm}
\psset{unit=0.75}
      \begin{pspicture}(3,3.5)(-3,-3)
         \pscircle(0,0){3}
         \psarc[linewidth=0.05]{->}(0,0){4.7}{8}{35}
         \rput(5,1.8){$\textbf{+}$}
         \psline{-}(-3,0)(0,0)
         \psline[linecolor=blue]{-}(0,0)(3,0)
         \psline{-}(0,-3)(0,0)
         \psline[]{-}(0,0)(0,3)
         \psline{-}(-2.6,-1.5)(0,0)
         \psline[]{-}(0,0)(2.6,1.5)
         \psline[linestyle=dashed]{-}(-2.1,-2.1)(2.1,2.1)
         \psline{-}(-1.5,-2.6)(0,0)
         \psline[linecolor=blue]{-}(0,0)(1.5,2.6)
         \psline{-}(-2.6,1.5)(2.6,-1.5)
         \psline[linestyle=dashed]{-}(-2.1,2.1)(2.1,-2.1)
         \psline{-}(-1.5,2.6)(1.5,-2.6)
         \psline[linestyle=dotted,linecolor=blue]{-}(3,0)(1.5,2.6)
         %\psline[linestyle=dashed,linecolor=red]{-}(0,3)(2.6,1.5)
         \put(3.3,0){$0,\ 2\pi$}
         \rput(3,1.5){$\frac{\pi}{6}$}
         \rput(2.4,2.5){$\frac{\pi}{4}$}
         \rput(1.6,3.1){$\textcolor{blue}{\frac{\pi}{3}}$}
         \rput(0,3.5){$\frac{\pi}{2}$}
%         \rput(-3,1.5){$\frac{5\pi}{6}$}
%         \rput(-2.4,2.5){$\frac{3\pi}{4}$}
%         \rput(-1.6,3.1){$\frac{2\pi}{3}$}
%         \rput(-3.3,0){$\pi$}
%         \rput(-3,-1.5){$\frac{7\pi}{6}$}
%         \rput(-2.4,-2.5){$\frac{5\pi}{4}$}
%         \rput(-1.6,-3.2){$\frac{4\pi}{3}$}
%         \rput(0,-3.5){$\frac{3\pi}{2}$}
%         \rput(3.1,-1.5){$\frac{11\pi}{6}$}
%         \rput(2.4,-2.5){$\frac{7\pi}{4}$}
%         \rput(1.6,-3.2){$\frac{5\pi}{3}$}
      \end{pspicture}
\end{minipage}
   
   
  



\section{Fonctions sinus et cosinus}%____________________________

%\subsection{Défintions}%____________________________
\noindent\begin{minipage}{\textwidth - 7cm}
\begin{Def}
   Soit $x$ un réel quelconque. Il lui correspond un unique point $M$ de $\mathscr{C}$.
On appelle \emph{cosinus} de $x$, noté $\cos x$ et \emph{sinus} de $x$, noté $\sin x$, les coordonnées du point $M$ dans le repère $(O;\vec{\imath},\vec{\jmath}\,) $.
% tel que $x$ soit une mesure en radians de $(\widehat{\V{OA}, \V{OM}})$
   
\end{Def}
D'après le cercle trigonométrique, on a les propriétés:

   \begin{dingautolist}{168}
      \item $\cos^{2}x + \sin^{2}x=\dots{}\qquad$ d'après le théorème de\dots{}
      \item $\dots{} \leqslant \cos x \leqslant \dots{}$ \quad et \quad $\dots{} \leqslant \sin x \leqslant \dots{}$
      \item $\cos(x+2\pi)=\cos x$ \quad et \quad $\sin(x+2\pi)=\sin x$
   \end{dingautolist}


\end{minipage}
\hfill
\begin{minipage}{5cm}
   \begin{center}
      \psset{unit=2.2}
      \begin{pspicture}(-1,-0.8)(1,1.1)
         \pscircle(0,0){1}
         \psline{-}(-1,0)(0,0)
         \psline{->}(0,0)(0,1)
         \psline{-}(0,-1)(0,0)
         \psline{->}(0,0)(1,0)
         \psline[linestyle=dashed,linecolor=blue]{-}(0.5,0)(0.5,0.87)
         \psline[linestyle=dashed,linecolor=blue]{-}(0,0.87)(0.5,0.87)
         \psline[linewidth=0.02,linecolor=red]{-}(0,0)(0.5,0.87)
         \psarc[linecolor=red]{->}(0,0){0.3}{0}{60}
         \rput(0.6,1.05){\textcolor{red}{$M$}}
         \rput(0.35,0.2){\textcolor{blue}{$x$}}
         \rput(0.5,-0.21){\textcolor{blue}{$\cos x$}}
         \rput(-0.23,0.75){\textcolor{blue}{$\sin x$}}
         \rput(-0.1,-0.1){$0$}
         \rput(-0.1,0.5){$\vj$}
         \rput(0.62,0.15){$\vi$}
      \end{pspicture}
   \end{center}
\end{minipage}
 


\subsection*{Variations et valeurs remarquables} %____________________________
\noindent\begin{minipage}{\textwidth - 7cm}
Par lecture sur le cercle trigonométrique, compléter les deux tableaux de variation:

\medskip
\begin{variations}
    x       &0&   &  \frac{\pi}{2}  &   &\pi\\
    \filet
    \m{\sin(x)}&&\phantom{\c} &&\phantom{\d} &\\
  \end{variations}
\hfill \begin{variations}
    x       &0&   &  \frac{\pi}{2}  &   &\pi\\
    \filet
    \m{\cos(x)}&&\phantom{\c} &&\phantom{\d} &\\
  \end{variations}
%%%%%%%%%% COURBES

Identifier les deux courbes.
\medskip

\noindent
   \psset{unit=0.8cm} 
   \begin{pspicture}(-7,-1.5)(7,1.5)
      \psgrid[subgriddiv=1,griddots=1,gridlabels=0](-7,-1)(7,1) 
     

      \begin{footnotesize}
      \psaxes[labels=y]{->}(0,0)(-7,-1.5)(7,1.5)
      \end{footnotesize}


      %COS
      \psplot[plotpoints=100,linewidth=0.02,linecolor=blue]{-7}{7}{x 90 mul 1.57 div sin} 
      \psplot[plotpoints=100,linewidth=0.07,linecolor=blue]{0}{3.14}{x 90 mul 1.57 div sin}
%SIN
      \psplot[linestyle=dashed,plotpoints=100,linewidth=0.04,linecolor=red]{-7}{7}{x 90 mul 1.57 div cos} 
      \psplot[linestyle=dashed,plotpoints=100,linewidth=0.08,linecolor=red]{0}{3.14}{x 90 mul 1.57 div cos} 
      
      %
      \rput(1.57,-0.6){ $\frac{\pi}{2}$}

      \rput(3.15,-0.5){ $\pi$}

      \rput(-1.6,-0.6){ $-\frac{\pi}{2}$}

      \rput(-3.3,0.5){ $-\pi$}

      \rput(-6.3,-0.5){ $-2\pi$}
\psdots(-6.3,0)(6.3,0)(-3.14,0)(3.14,0)(-1.57,0)(1.57,0)(4.7,0)(-4.7,0)
      
      \rput(6.4,-0.5){ $2\pi$}
     
   \end{pspicture} 


\end{minipage}
\hfill
\begin{minipage}{5cm}
%\begin{center}
   \psset{unit=1.3cm} 
   \begin{pspicture}(-1,-0.2)(3.5,3.5)
      \psarc(0,0){3}{0}{90}
      \psline{-}(0,0)(3,0)
      \psline{-}(0,0)(0,3)
      \psline[linestyle=dashed,linecolor=blue](2.6,0)(2.6,1.5)(0,1.5)
      \psline[linestyle=dashed,linecolor=red](2.1,0)(2.1,2.1)(0,2.1)
      \psline[linestyle=dashed,linecolor=blue](1.5,0)(1.5,2.6)(0,2.6)
      \pspolygon(0,0)(1.5,2.6)(3,0)
      \rput(3.3,0){$0$}
      \rput(3,1.5){\textcolor{blue}{$\dfrac{\pi}{6}$}}
      \rput(2.4,2.5){\textcolor{red}{$\dfrac{\pi}{4}$}}
      \rput(1.6,3.1){\textcolor{blue}{$\dfrac{\pi}{3}$}}
      \rput(0,3.4){$\dfrac{\pi}{2}$}
      \rput(1.5,-0.4){$\dfrac{1}{2}$}
      \rput(2.1,-0.4){$\dfrac{\sqrt{2}}{2}$}
      \rput(2.6,-0.4){$\dfrac{\sqrt{3}}{2}$}
      \rput(-0.3,-0.3){$0$}
      \rput(-0.3,1.5){$\frac{1}{2}$}
      \rput(-0.3,2.1){$\frac{\sqrt{2}}{2}$}
      \rput(-0.3,2.6){$\frac{\sqrt{3}}{2}$}
   \end{pspicture}
%\end{center}
\end{minipage}
\end{document}
