% ----------------------------------------------------------------------
% Template VERIFICA
% ----------------------------------------------------------------------
% 2020 di d!egofantinelli at jazzmagus@gmail.com
% ----------------------------------------------------------------------

% ---------------------------------- Preambolo
\documentclass[11pt, a4paper]{exam}
\usepackage[T1]{fontenc}
\usepackage{mdframed}
%\usepackage{nicefrac}
%\usepackage[applemac]{inputenc}
%\usepackage[utf8]{inputenc}
\usepackage[italian]{babel}
\usepackage[margin=1.3in]{geometry}
\usepackage{amsmath,amssymb, systeme}
\usepackage{multicol}
\usepackage{anyfontsize}
\usepackage{cancel}
\usepackage{graphicx}
\usepackage{tikz}
\usepackage{upquote}
\usepackage{caption}
\usepackage[normalem]{ulem}
%\usepackage{fancyhdr}
\usepackage{float}
\usepackage{array}


\renewcommand{\questionshook}{%
    \setlength{\leftmargin}{0pt}%
}
\renewcommand{\choiceshook}{%
    \setlength{\leftmargin}{20pt}%
}

\newenvironment{sistema}% 
{\left\lbrace\begin{array}{@{}l@{}}}% 
{\end{array}\right.} 

% ---------------------------------- Intestazione
\newcommand{\class}{\huge {Verifica di Matematica}}
\newcommand{\term}{2� Quadrimestre}
\newcommand{\examnum}{Recupero insufficienze a.s. 2020/'21}
\newcommand{\examdate}{data: \uline{\hspace{6em}}}
\newcommand{\timelimit}{60 minuti}

\CorrectChoiceEmphasis{\color{red}}
\SolutionEmphasis{\color{red} \footnotesize}
\renewcommand{\solutiontitle}{\noindent\textbf{Soluzione:}\par\noindent}
% ---------------------------------- Intestazione

\pagestyle{headandfoot}
\firstpageheader{IIS "G.A. Remondini" - Bassano del Grappa (VI)}{}{\examdate}
\runningheader{\footnotesize VERIFICA di MATEMATICA}{}{\emph{CASPON Edward}}
\runningheadrule

\firstpagefooter{}{}{pag. \thepage\ di \numpages}
\runningfooter{}{}{pag. \thepage\ di \numpages}
\runningfootrule

% ---------------------------------- Punteggi
\pointpoints{punto}{\em punti}
\pointformat{[{\footnotesize \thepoints}]}
\bonuspointpoints{punto bonus}{\em punti bonus}
\bonuspointformat{[{\footnotesize \thepoints}]}
\pointsinrightmargin
\setlength{\rightpointsmargin}{.2cm}
\chqword{Esercizio}
\chpword{Punti}
\chbpword{Punti Bonus}
\chsword{Punteggio}
\chtword{Totale}


%\printanswers

\begin{document}

% ---------------------------------- Title Page
\begin{center}
	\rule[2ex]{\textwidth}{0.5pt}\\
	{\huge{\bf \class}}\\[12pt]
	%{\huge -\, \term \, - }\\[8pt]
	{\huge -\, \examnum \, - }\\[8pt]
	\rule[2ex]{\textwidth}{0.5pt}\\
	\vspace{1cm}
%\color{red} {\fontsize{50}{10}\selectfont {\bf SOLUZIONI}}\\
\end{center}
\vspace{3cm}
\begin{tabular*}{\textwidth}{l @{\extracolsep{\fill}} r @{\extracolsep{6pt}} l}
\textbf{} & \textbf{Cognome e Nome:} & \makebox[2.5in]{\Large{\bf CASPON Edward}}\\

\textbf{} &&\\
\textbf{} & \textbf{Classe:} & \makebox[2.5in]{\Large{\bf 3\string^C}}\\
\textbf{} &&\\
\textbf{} & Tempo a disposizione: & \makebox[2.5in]{\timelimit}
\end{tabular*}\\[3cm]

\vspace{3cm}
% ---------------------------------- Avvertenze

\noindent
\rule[2ex]{\textwidth}{0.2pt}
\textbf{Avvertenze}:
\begin{itemize}
	\item La presente Verifica di Recupero - che viene somministrata in modalit� IN PRESENZA - contiene \numquestions \; quesiti, per un totale di \numpoints \;punti;
	\item Per gli eventuali studenti che dovessero svolgere la prova in DDI, la webcam dovr� rimanere accesa per tutto il tempo della verifica (\timelimit), salvo impossibilit� concrete di connessione; il microfono rester� spento e verr� acceso soltanto per chiarimenti e domande, che saranno consentite negli ultimi 20 min di prova.
	\item E' vietato l'utilizzo di calcolatrici - scientifiche e non -, smartphone, tablet e altri dispositivi digitali, nonch� la consultazione di testi, appunti e siti web.

\end{itemize}
%\rule[2ex]{\textwidth}{0.2pt}
\vfill
\newpage

% ================================== ESERCIZI ===================================
% todo completare punto a
% ---------------------------------- Esercizio 1 --------------------------------

\begin{questions}

\addpoints

\question[6]
Fattorizza il seguente polinomio con il metodo che ritieni pi� opportuno:\\
\small{\emph{suggerimento: ti sar� sufficiente ricordare i Prodotti Notevoli }}\\[10pt]
\(4x^2 - 12x - 40\)

\fillwithdottedlines{.75in}

{\footnotesize
\begin{solution}
	\(\left[ \; 4(x - 5)(x + 2) \; \right]\)
\end{solution}
}
%\vspace{12pt}


\addpoints

\question
Semplifica la seguenti espressioni e determina le condizioni di esistenza:\\

\begin{parts}

\part[6]
\(\left( x - \dfrac{x}{x + 1} \right) : \left( 1 - \dfrac{2x}{x - 1}\right)\cdot \left( \dfrac{1}{x^2} + \dfrac{2}{x} + 1 \right)\)

\fillwithdottedlines{1.25in}
\vspace{12pt}
{\footnotesize
\begin{solution}
	\(\left[ \; 1 - x; \quad x \neq \pm 1 \wedge x \neq 0 \; \right]\)
\end{solution}
}
\vspace{12pt}

\part[8]
\(\left( \dfrac{x}{x + 1} + \dfrac{x}{1 + x} \right) : \left( 1 - \dfrac{1}{1 - x^2}\right)\)

\fillwithdottedlines{1.5in}
\vspace{12pt}
{\footnotesize
\begin{solution}
	\(\left[ \; 2 \; \right]\)
\end{solution}
}

\end{parts}

\vspace{12pt}
%\pagebreak

\addpoints

\question[10]
Determina la soluzione della seguente Equazione Frazionaria:\\[4pt]
\small{\emph{suggerimento: ricorda di verificare le Condizioni di Esistenza: C.E. }}\\[10pt]
\(\dfrac{1}{x^{2}-4}-\dfrac{2}{x+2}=\dfrac{3}{x-2}\)

\fillwithdottedlines{1.5in}
\vspace{12pt}

{\footnotesize
\begin{solution}
	Risolviamo l'equazione:

\(\dfrac{1}{x^{2}-4}-\dfrac{2}{x+2}=\dfrac{3}{x-2}\)

{\bfseries \(1^{\mathrm{o}}\) passo:} Scomponendo i denominatori dell'equazione, riscriviamo l'equazione nella forma:
\[
\dfrac{1}{(x-2)(x+2)}-\dfrac{2}{x+2}=\dfrac{3}{x-2}
\]
Poniamo le condizioni di esistenza.
C.E.: \(\quad x \neq 2 \wedge x \neq-2\)\\[10pt]
{\bfseries \(2^{\mathrm{o}}\) passo:} Il m.c.m. dei denominatori � \((x-2)(x+2)\).

\[
\begin{array}{l}
\dfrac{1}{(x-2)(x+2)}-\dfrac{2}{x+2} = \dfrac{3}{x-2} \\[8pt]
(x-2)\cancel{(x+2)} \cdot \dfrac{1}{(x-2)\cancel{(x+2)}} - (x-2)\cancel{(x+2)} \cdot \dfrac{2}{\cancel{x+2}} = \\[8pt]
=\cancel {(x-2)}(x+2) \cdot \dfrac{3}{\cancel{x-2}} \\[8pt]
1-2(x-2)=3(x+2) \\[8pt]
1-2 x+4=3 x+6 \\[8pt]
-2 x-3 x=6-1-4
\end{array}
\]

\(-5 x=1 \Rightarrow x=-\dfrac{1}{5} \)\\[10pt]
{\bfseries \(3^{\mathrm{o}}\) passo:} La soluzione trovata,  \(x=-\dfrac{1}{5}\), soddisfa le condizioni di esistenza dell'equazione, quindi � {\bfseries accettabile.}
\end{solution}
}
\end{questions}
\pagebreak



% ===============================================================================

\begin{center}
\rule[2ex]{\textwidth}{1pt}
\vspace{.5cm}
{\bf Tabella dei punteggi}
\vspace{10pt}

\combinedgradetable[h][questions]
\end{center}
\vspace{4pt}
\small La sufficienza � fissata a 14 punti, ma potr� subire delle modifiche in fase di correzione, al fine di garantire la validit� della prova anche nel caso in cui si riscontrassero prestazioni della classe sensibilmente lontane dalla media stimata.

\end{document}
