% ----------------------------------------------------------------------
% Template VERIFICA
% ----------------------------------------------------------------------
% 2020 di d!egofantinelli at jazzmagus@gmail.com
% ----------------------------------------------------------------------

% ---------------------------------- Preambolo
\documentclass[11pt, a4paper]{exam}
\usepackage[T1]{fontenc}
\usepackage{mdframed}
%\usepackage{nicefrac}
%\usepackage[applemac]{inputenc}
%\usepackage[utf8]{inputenc}
\usepackage[italian]{babel}
\usepackage[margin=1.3in]{geometry}
\usepackage{amsmath,amssymb}
\usepackage{multicol}
\usepackage{graphicx}
\usepackage{tikz}
\usepackage{upquote}
\usepackage{caption}
%\usepackage{fancyhdr}
\usepackage{float}


% ---------------------------------- Command

\renewcommand{\questionshook}{%
    \setlength{\leftmargin}{0pt}%
}
\renewcommand{\choiceshook}{%
    \setlength{\leftmargin}{20pt}%
}

\newcommand{\class}{\huge {Verifica di Matematica}}
\newcommand{\term}{n. 01 | quad. 02}
%\newcommand{\examnum}{Verifica numero: 1}
\newcommand{\examdate}{12 marzo 2021}
\newcommand{\timelimit}{40 minuti}

\CorrectChoiceEmphasis{\color{red}}
\SolutionEmphasis{\color{red} \footnotesize}
\renewcommand{\solutiontitle}{\noindent\textbf{Soluzione:}\par\noindent}

% ---------------------------------- Headers and Footers

\pagestyle{headandfoot}
\firstpageheader{IIS "G. A. Remondini" - Bassano del Grappa (VI)}{}{\examdate}
\runningheader{\footnotesize VERIFICA di MATEMATICA}{}{Classe 3\string^QA}
\runningheadrule

\firstpagefooter{}{}{pag. \thepage\ di \numpages}
\runningfooter{}{}{pag. \thepage\ di \numpages}
\runningfootrule

% ---------------------------------- Punteggi
\pointpoints{punto}{\em punti}
\pointformat{[{\footnotesize \thepoints}]}
\bonuspointpoints{punto bonus}{\em punti bonus}
\bonuspointformat{[{\footnotesize \thepoints}]}
\pointsinrightmargin
\setlength{\rightpointsmargin}{.2cm}
\chqword{Esercizio}
\chpword{Punti}
\chbpword{Punti Bonus}
\chsword{Punteggio}
\chtword{Totale}

%\printanswers
\begin{document}

% ---------------------------------- Title Page
\begin{center}
\rule[2ex]{\textwidth}{0.5pt}\\
{\huge{\bf \class}}\\[12pt]
{\huge \, \term}\\[8pt]
\rule[2ex]{\textwidth}{0.5pt}\\
\end{center}
\vspace{3cm}
\begin{tabular*}{\textwidth}{l @{\extracolsep{\fill}} r @{\extracolsep{6pt}} l}
\textbf{} & \textbf{Cognome e Nome:} & \makebox[2.5in]{\hrulefill}\\
\textbf{} &&\\
\textbf{} & \textbf{Classe:} & \makebox[2.5in]{\Large{\bf 3 \string^ QA}}\\
\textbf{} &&\\
\textbf{} & Tempo a disposizione: & \makebox[2.5in]{\timelimit}
\end{tabular*}\\[3cm]
\vspace{4.5cm}

% ---------------------------------- Avvertenze

\noindent
\rule[2ex]{\textwidth}{0.2pt}
\textbf{Avvertenze}:
\begin{itemize}
	\item La presente Verifica - che viene somministrata in modalit� DDI - contiene \numquestions \; quesiti, per un totale di \numpoints \;punti, uno dei quali facoltativo, che verr� valutato soltanto se saranno stati risolti anche tutti gli altri.
	\item La webcam dovr� rimanere accesa per tutto il tempo della verifica (\timelimit), salvo impossibilit� concrete di connessione; il microfono rester� spento e verr� acceso soltanto per chiarimenti e domande, che saranno consentite negli ultimi 20 min di prova.
	\item E' vietato l'utilizzo di calcolatrici scientifiche, smartphone, tablet e altri dispositivi digitali, nonch� la consultazione di testi, appunti e siti web.

\end{itemize}
\vspace{5pt}
\noindent
\rule[2ex]{\textwidth}{0.2pt}
\vfill
\newpage

% ---------------------------------- Esercizio 1
\begin{questions}

\addpoints
\question Semplifica le seguenti espressioni algebriche:\\
\begin{parts}

\part[4]
\((2a^2 - 5b) - [(2b + 4a^2) - (2a^2 - 2b)] - 9b\)
\fillwithdottedlines{.5in}
{\footnotesize
\begin{solution}
	\(- 18b\)
\end{solution}
}
\vspace{.5cm}

\part[4]
\(2(x -1)(3x + 1) - (6x^2 + 3x + 1) + 2x(x - 1)\)
\fillwithdottedlines{.5in}
{\footnotesize
\begin{solution}
	\(2x^2 - 9x - 3\)
\end{solution}
}

\end{parts}
\vspace{.5cm}

% ---------------------------------- Esercizio 2

\addpoints
\question[6] Semplifica la seguente espressione algebrica, notando l'eventuale presenza di prodotti notevoli:\\

\([(3x -y)^3 + y^3] : 3x - (3x - y)^2\)
\fillwithdottedlines{.5in}
{\footnotesize
\begin{solution}
	\(2y^2 - 3xy\)
\end{solution}
}
\vspace{.5cm}

%% ------------------------------------- Esercizio 3
%\question
%Rispondi in modo chiaro e sintetico alle seguenti domande:\\
%\begin{parts}
%\part[4]
%Quando due o pi� monomi si dicono {\bfseries simili}?\ Puoi fare un esempio numerico?
%%\fillwithlines{0.75in}
%{\footnotesize
%\begin{solution}
%Due o pi� monomi si dicono simili quando presentano la stessa parte letterale; ad es sono simili: \(3ab^2, -6ab^2\) e \(\dfrac{1}{2}ab^2\).
%\end{solution}
%}
%\vspace{.5cm}
%\part[5]
%Che cosa sono gli {\bfseries zeri} di un polinomio?
%%\fillwithlines{.75in}
%
%{\footnotesize
%\begin{solution}
%Gli zeri di un polinomio sono quei valori della variabile per cui il valore numerico del polinomio risulta nullo.	
%\end{solution}
%}
%\end{parts}
%
%\vspace{.5cm}

% ------------------------------------- Esercizio 3
\addpoints
\question [4] Affinch� si possa eseguire una divisione tra monomi, gli esponenti della parte letterale del dividendo, rispetto a quelli del divisore, devono essere:\\

\begin{choices}
\setlength{\leftmargin}{0pt}
 \choice uguali a quelli della parte letterale del divisore. 
 \CorrectChoice maggiori o uguali a quelli della parte letterale del divisore.
 \choice maggiori di quelli della parte letterale del divisore.
 \choice minori o uguali a quelli della parte letterale del divisore.
 \choice nessuna delle precedenti
\end{choices}
\vspace{.5cm}

% ------------------------------- Esercizio 4
\addpoints
\question Esegui i seguenti prodotti notevoli:\\
 
\begin{parts}
	
\part[4] \(\left(\dfrac{2}{3}m + n^2 \right)\left(\dfrac{2}{3}m - n^2 \right)\)
\fillwithdottedlines{.5in}
{\footnotesize
\begin{solution}
	\(\left[\dfrac{4}{9}m^2 + n^4 \right]\)
\end{solution}
}
\vspace{5pt}
\part[4] \((xy^2z + 2x^2 - 3)^2\)\fillwithdottedlines{.5in}
{\footnotesize
\begin{solution}
	\([x^2y^4z^2 + 4x^4 + 9 + 4x^3y^2z - 6xy^2z - 12x^2]\)
\end{solution}
}
\vspace{5pt}

\part[4] \((3a^2 - 2ab)^2\)
\fillwithdottedlines{.5in}
{\footnotesize
\begin{solution}
	\([9a^2 - 12a^3b + 4a^2b^2]\)
\end{solution}
}
%\vspace{8pt}

\end{parts}
%\vfill
%\newpage
\vspace{2cm}

% ------------------------------- Esercizio 5 - Facoltativo (bonus points)
\noindent
\rule[1ex]{\textwidth}{0.5pt}
\addpoints
\bonusquestion[4] {\em Esercizio facoltativo:}\\

Esegui il seguente cubo di binomio:\\

 \(\left(- \dfrac {3}{4}a + 4b^2 \right)^3 \)
\fillwithdottedlines{.75in}
{\footnotesize
\begin{solution}
	\(\left[-\dfrac{27}{64}a^3 + \dfrac{27}{4}a^2b^2 - 36ab^4 + 64b^6 \right]\)
\end{solution}
}

\end{questions}

%\pagebreak
% ---------------------- Tabella punteggi
\vfill
\noindent
\rule[2ex]{\textwidth}{1pt}

\begin{center}
{\bf Tabella dei punteggi}
\vspace{10pt}

\combinedgradetable[h][questions]
\end{center}
\vspace{4pt}
\footnotesize La sufficienza � fissata a 18 punti, ma potr� subire delle modifiche in fase di correzione, al fine di garantire la validit� della prova anche nel caso in cui si riscontrassero prestazioni della classe sensibilmente lontane dalla media prevista.

\end{document}
