% ----------------------------------------------------------------------
% Template VERIFICA

% ------------------------------------------------------------------
% 2020 di d!egofantinelli at jazzmagus@gmail.com
% ----------------------------------------------------------------------

% ---------------------------------- Preambolo
\documentclass[11pt, a4paper]{exam}
\usepackage[T1]{fontenc}
\usepackage{mdframed}
%\usepackage{nicefrac}
%\usepackage[applemac]{inputenc}
%\usepackage[utf8]{inputenc}
\usepackage[italian]{babel}
\usepackage[margin=1.3in]{geometry}
\usepackage{amsmath, amssymb}
\usepackage{multicol}
\usepackage{graphicx}
\usepackage{array}
\usepackage{tikz}
\usepackage{upquote}
\usepackage{caption}
%\usepackage{fancyhdr}
\usepackage{float}
\renewcommand{\solutiontitle}{\noindent\textbf{Soluzione:}\par\noindent}
\renewcommand{\arraystretch}{3}

\newcommand{\class}{\huge {verifica di matematica}}
\newcommand{\term}{I Quadrimestre}
\newcommand{\examnum}{num.: 3}
\newcommand{\examdate}{\scriptsize 23 novembre 2020}
\newcommand{\timelimit}{50 min}

\CorrectChoiceEmphasis{\color{red}}
\SolutionEmphasis{\color{red} \footnotesize}
%\unframedsolutions
\printanswers


% ---------------------------------- Intestazione

\pagestyle{headandfoot}
\firstpageheader{\small IIS "G. A. Remondini" - Bassano del Grappa}{}{\examdate}
\runningheader{\footnotesize VERIFICA DI MATEMATICA}{\footnotesize SOLUZIONI}{Classe $3^a$ QA}
\runningheadrule

\firstpagefooter{}{}{pag. \thepage\ di \numpages}
\runningfooter{}{}{pag. \thepage\ di \numpages}
\runningfootrule

% ---------------------------------- Punteggi
\pointpoints{punto}{\em punti}
\pointformat{[{\footnotesize \thepoints}]}
\pointsinrightmargin
\setlength{\rightpointsmargin}{.2cm}
\bonuspointpoints{punto bonus}{\em punti bonus}
\bonuspointformat{[{\footnotesize \thepoints}]}
\chqword{Esercizio}
\chpword{Punti}
\chbpword{Punti Bonus}
\chsword{Punteggio}
\chtword{Totale}


% ============================================

\begin{document}

% ---------------------------------- Title Page
\begin{center}
\rule[2ex]{\textwidth}{0.5pt}\\
{\huge{\bf \class}}\\[20pt]
{\huge{ \term} -{ \examnum}}\\[8pt]
\rule[2ex]{\textwidth}{0.5pt}\\
\end{center}
\vspace{1cm}
\begin{tabular*}{\textwidth}{l @{\extracolsep{\fill}} r @{\extracolsep{6pt}} l}
\textbf{} & \textbf{Nome e Cognome:} & \makebox[2.5in]{\hrulefill}\\
\textbf{} &&\\
\textbf{} & \textbf{Classe:} & \makebox[2.5in]{\Large{\bf $3^a$ QA}}\\
%\textbf{} &&\\
\textbf{} & Tempo a disposizione: & \makebox[2.5in]{\timelimit}\\
%\textbf{} &&\\
%\textbf{} &&\\
\textbf{} &&\\
\textbf{} & {\em prof.:} & \makebox[2.5in]{\em Diego Fantinelli}
\end{tabular*}\\

\vspace{3cm}

% ---------------------------------- Avvertenze

\noindent

\textbf{Avvertenze}:
\begin{itemize}
	\item La presente Verifica - che viene somministrata in modalit� DDI - contiene \numquestions \; quesiti, per un totale di \numpoints \;punti, di cui uno facoltativo di \numbonuspoints \;punti, che verr� conteggiato soltanto se verranno svolti anche tutti i precedenti.
	\item La webcam dovr� rimanere accesa per l'intera durata della della verifica (\timelimit), salvo impossibilit� concrete di connessione; il microfono rester� spento e verr� acceso soltanto per chiarimenti e domande, che saranno consentite negli ultimi 20 min della prova.
	\item E' vietato l'utilizzo di calcolatrici scientifiche, smartphone, tablet e altri dispositivi digitali, nonch� la consultazione di testi, appunti e siti web.
	\item La verifica dovr� essere consegnata in formato digitale (pdf, jpeg, png, etc.) e dovr� essere ben leggibile; si consiglia l'inquadratura verticale.
\end{itemize}

\vfill
\newpage

% ---------------------------------- Esercizio 1
\begin{questions}
\addpoints
\question Ricordando le propriet� delle potenze, semplifica le seguenti espressioni, nell'Insieme \(\mathbb{Q}\) dei Numeri Razionali:
\begin{parts}
\part[6]  \( \left \{ \left [ - \dfrac{1}{3^2} : \left( - \dfrac{1}{3} \right)^{3} - 2 \right ]^{4}   : \left ( 2 - \dfrac{1}{3} \right )^{2} \right \} : \left ( - \dfrac{1}{5} \right )^{2} \)

\begin{solution}
	\( 9 \)
\end{solution}

\addpoints
\part[6]
\( \left [- 2^{3} : (-2)^{2} + \left( - \dfrac{1}{3} \right )^{3} : \left( -\dfrac{1}{3} \right)^{2} - \dfrac{1}{2^2} : \left( - \dfrac{1}{2} \right)^{3} \right ] : \left ( - \dfrac{1}{3} \right )^{3} \)

\begin{solution}
	\( 9 \)
\end{solution}
\end{parts}
\vspace{5pt}

% ---------------------------------- Esercizio 2
\addpoints
\question[6] Per le seguenti frazioni calcolare il numero decimale generato e indicare la tipologia di numero decimale alla quale appartiene:

\begin{oneparchoices}
	\choice \( \dfrac{37}{11} \)
	\choice \( \dfrac{15}{8} \)
	\choice \( \dfrac{2}{9}\)
	\choice \( \dfrac{19}{24}\)
	\choice \( \dfrac{7598}{100}\)
	\choice \( \dfrac{7}{15}\)
\end{oneparchoices}
\vspace{3pt}

% ---------------------------------- Esercizio 3

\addpoints
\question [4] Quale, tra le seguenti affermazioni, � falsa?

\begin{choices}
 \choice Un \emph{numero decimale periodico semplice} di periodo 9 coincide esattamente con il numero intero successivo.
 \choice Un frazione, ridotta ai minimi termini, genera un numero decimale limitato se il suo denominatore scomposto in fattori primi contiene solo fattori 2 e/o 5.
 \choice Un frazione, ridotta ai minimi termini, genera un numero periodico semplice se il suo denominatore scomposto in fattori primi non contiene fattori 2 e/o 5.
 \CorrectChoice La frazione generatrice di un \emph{numero periodico} ha al denominatore un numero le cui cifre sono tutte uguali a 9.
 \choice Un frazione, ridotta ai minimi termini, genera un numero periodico misto se il suo denominatore scomposto in fattori primi contiene fattori 2 e/o 5 e altri fattori
\end{choices}
\vspace{3pt}

% ---------------------------------- Esercizio 4

\addpoints
\question [8]Calcola la \emph{frazione generatrice} dei seguenti numeri decimali:

\begin{multicols}{2}
\begin{parts}
\part
\( 37,35\)
\vspace{3pt}

\begin{solution}
\( \dfrac{747}{20} \)
\end{solution}
\vspace{3pt}

\part
\( 0,0 \overline{32} \)
\vspace{3pt}

\begin{solution}
\( \dfrac{16}{495} \)
\end{solution}

\part
\( 0,17 \overline{2} \)
\vspace{3pt}

\begin{solution}
\( \dfrac{31}{180} \)
\end{solution}

\part
\( 0, \overline{43902} \)
\vspace{3pt}

\begin{solution}
\( \dfrac{18}{41} \)
\end{solution}

\end{parts}
\end{multicols}

%\vspace{\stretch{2}}

% ---------------------------------- Esercizio Bonus
\bonusquestion[5]{\em Esercizio facoltativo:}
Semplifica la seguente espressione in \(\mathbb{Q}\):

\( (- 0,5 - 0. \overline {3}) \cdot \left[ \dfrac{2}{5} - ( - 0,5 + 1) + 0,4 \right] : [- (1 - 2^{2})] \)

\begin{solution}
\( - \dfrac{1}{12} \)	
\end{solution}


\end{questions}
%\vspace{3pt}
\noindent
\rule[1ex]{\textwidth}{1pt}
\begin{center}
{\bf Tabella dei punteggi}\\
\vspace{1pt}
\combinedgradetable[h][questions]
\end{center}
%\vspace{2pt}
\footnotesize La sufficienza � fissata a 20 punti, ma potr� subire delle modifiche in fase di correzione, al fine di garantire la validit� della prova anche in caso di andamenti troppo scostanti della media-classe.
\end{document}
  