% ----------------------------------------------------------------------
% Template VERIFICA
% ----------------------------------------------------------------------
% 2020 di d!egofantinelli at jazzmagus@gmail.com
% ----------------------------------------------------------------------

% ---------------------------------- Preambolo
\documentclass[11pt, a4paper]{exam}
\usepackage[T1]{fontenc}
\usepackage{mdframed}
%\usepackage{nicefrac}
%\usepackage[applemac]{inputenc}
%\usepackage[utf8]{inputenc}
\usepackage[italian]{babel}
\usepackage[margin=1.3in]{geometry}
\usepackage{amsmath, amssymb}
\usepackage{multicol}
\usepackage{graphicx}
\usepackage{tikz}
\usepackage{upquote}
\usepackage{caption}
%\usepackage{fancyhdr}
\usepackage{float}
\renewcommand{\solutiontitle}{\noindent\textbf{Soluzione:}\par\noindent}
% ---------------------------------- Intestazione
\newcommand{\class}{\huge {verifica di matematica}}
\newcommand{\term}{I Quadrimestre}
\newcommand{\examnum}{num.: 1}
\newcommand{\examdate}{\scriptsize 13 novembre 2020}
\newcommand{\timelimit}{45 min}

\CorrectChoiceEmphasis{\color{red}}
\SolutionEmphasis{\color{red} \footnotesize}
%\unframedsolutions
% ---------------------------------- Intestazione

\pagestyle{headandfoot}
\firstpageheader{\small ISS "G. A. Remondini" - Bassano del Grappa}{}{\examdate}
\runningheader{\footnotesize VERIFICA DI MATEMATICA}{}{Classe $3^a$ QA}
\runningheadrule

\firstpagefooter{}{}{pag. \thepage\ di \numpages}
\runningfooter{}{}{pag. \thepage\ di \numpages}
\runningfootrule

% ---------------------------------- Punteggi
\pointpoints{punto}{\em punti}
\pointformat{[{\footnotesize \thepoints}]}
\pointsinrightmargin
\setlength{\rightpointsmargin}{.2cm}
\bonuspointpoints{punto bonus}{\em punti bonus}
\bonuspointformat{[{\footnotesize \thepoints}]}
\chqword{Esercizio}
\chpword{Punti}
\chbpword{Punti Bonus}
\chsword{Punteggio}
\chtword{Totale}

\begin{document}

% ---------------------------------- Title Page
\begin{center}
\rule[2ex]{\textwidth}{0.5pt}\\
{\huge{\bf \class}}\\[20pt]
{\huge{ \term} -{ \examnum}}\\[8pt]
\rule[2ex]{\textwidth}{0.5pt}\\
\end{center}
\vspace{3cm}
\begin{tabular*}{\textwidth}{l @{\extracolsep{\fill}} r @{\extracolsep{6pt}} l}
\textbf{} & \textbf{Nome e Cognome:} & \makebox[2.5in]{\hrulefill}\\
\textbf{} &&\\
\textbf{} & \textbf{Classe:} & \makebox[2.5in]{\Large{\bf $3^a$ QA}}\\
\textbf{} &&\\
\textbf{} & Tempo a disposizione: & \makebox[2.5in]{\timelimit}\\
\textbf{} &&\\
\textbf{} &&\\
\textbf{} &&\\
\textbf{} & {\em prof.:} & \makebox[2.5in]{\em Diego Fantinelli}
\end{tabular*}\\

\vspace{5cm}
% ---------------------------------- Avvertenze

\noindent

\textbf{Avvertenze}:
\begin{itemize}
	\item La presente Verifica - che viene somministrata in modalit� DDI - contiene \numquestions \; quesiti, per un totale di \numpoints \;punti, di cui uno  facoltativo, di \numbonuspoints \;punti, che verr� conteggiato soltanto se verranno svolti anche tutti i precedenti.
	\item La webcam dovr� rimanere accesa per tutto il tempo della verifica (\timelimit), salvo impossibilit� concrete di connessione; il microfono rester� spento e verr� acceso soltanto per chiarimenti e domande, che saranno consentite negli ultimi 20 min di prova.
	\item E' vietato l'utilizzo di calcolatrici scientifiche, smartphone, tablet e altri dispositivi digitali, nonch� la consultazione di testi, appunti e siti web.

\end{itemize}

\vfill
\newpage


% ---------------------------------- Esercizi
\begin{questions}
\addpoints
\question Risolvi i seguenti problemi, nell'Insieme \(\mathbb{N}\) dei Numeri Naturali:
\begin{parts}
\part[10]  Tre fari si accendono ad intervalli regolari.\\ Il primo si accende ogni 8 s, il secondo faro ogni 12 s, il terzo ogni 15 s. Se ad un certo istante si accendono contemporaneamente, dopo quanti secondi torneranno ad accendersi insieme?

\begin{solution}
I secondi che dovranno passare per far s� che i tre fari si accendano contemporaneamente
dovranno essere un multiplo di 8, 12 e 15, il minimo comune multiplo.
Effettuata la scomposizione in fattori primi, risulta che:\\
\begin{align*}
	8 & = 2^3\\
	12 & = 2^2 \cdot 3\\
	15 & = 3 \cdot 5
\end{align*}
per cui il \(m.c.m. = 2^3 \cdot 3 \cdot 5 = 120 \)\\
I tre fari torneranno ad accenderci contemporaneamente dopo 120 secondi.
\end{solution}

\addpoints

\part[10]	
 Gli studenti che frequentano il primo, il secondo ed il terzo anno di una scuola sono rispettivamente 140, 168 e 154.\\ Se si vogliono disporre tutti gli allievi in squadre di uguale numero di alunni, formate da alunni della stessa classe e con il numero pi� alto possibile, quanti
alunni devono essere presenti in ogni squadra e quante squadre si formeranno in totale?


\begin{solution}
Si tratta di un classico problema di M.C.D., si tratta cio� di calcolare il Massimo tra i Divisori Comuni di 140, 168 e 154.
Effettuata la scomposizione in fattori primi, risulta che:\\
\begin{align*}
	140 & = 2^2 \cdot 5 \cdot 7\\
	168 & = 2^3 \cdot 3 \cdot 7\\
	154 & = 2 \cdot 7 \cdot 11
\end{align*}
per cui il \(M.C.D. = 2 \cdot 7\)\\
In ogni squadra vi saranno pertanto 14 alunni.\\
In totale si formeranno (il totale degli studenti diviso il numero di studenti per squadra):\\ 
\((140 + 168 + 154) : 14 = 462 : 14 = 33\) squadre

\end{solution}

\end{parts}

\addpoints
\question[10]
Calcola il M.C.D. e il m.c.m. fra i seguenti numeri naturali \(\mathbb{N}\):

\begin{multicols}{2}
\begin{parts}

\part 

\(110, \quad 55, \quad 121\)


\begin{solution}

\(M.C.D.=11, \quad m.c.m.=1210\)
\end{solution}


\vspace{4pt}
\part
\(15, \quad 25, \quad 125, \quad 150\)


\begin{solution}

\(M.C.D. = 5, \quad m.c.m. = 750\)

\end{solution}

\vspace{5pt}
\end{parts}
\end{multicols}

\addpoints
\question [5]Quale, tra le seguenti definizioni, esprime meglio il procedimento di calcolo del M.C.D.?

\emph{Suggerimento:} leggere con molta attenzione il testo delle risposte perch� le differenze potrebbero essere minime.
\begin{choices}
 \choice Scomposti in fattori primi i numeri di cui si vuole calcolare il M.C.D., il M.C.D. � il quoziente dei fattori primi non comuni, presi una sola volta, con il massimo esponente.
 \choice Scomposti in fattori primi i numeri di cui si vuole calcolare il M.C.D., il M.C.D. � il prodotto dei fattori primi comuni e non comuni, presi una sola volta, con il minimo esponente.
 \CorrectChoice Scomposti in fattori primi i numeri di cui si vuole calcolare il M.C.D., il M.C.D. � il prodotto dei fattori primi comuni, presi una sola volta, con il minimo esponente.
 \choice Scomposti in fattori primi i numeri di cui si vuole calcolare il M.C.D., il M.C.D. � il prodotto dei fattori primi non comuni, presi una sola volta, con il minimo esponente.
\end{choices}
\vspace{5pt}

\addpoints
\quad
\addpoints
\question Ricordando le {\em propriet� delle potenze} e le {\em regole dei segni}, semplifica le seguenti espressioni, nell'Insieme \(\mathbb{Z}\) dei Numeri Interi:
\vspace{5pt}
\begin{parts}
\part[10]
\([(5^7)^2 : (5^5)^2 : 5^2 - 5^0] : (12^3 : 12^2)\)
\vspace{5pt}


\begin{solution}
\begin{align*}
	[ 5^{14} : 5^{10} : 5^2 - 1] : (12^1) & = [5^2 - 1] : 12\\ 
	& = [25 - 1] : 12\\
	& = 24 : 12 = 2
\end{align*}
\end{solution}
\vspace{5pt}

\addpoints
\part[15]	
\((2^{13} : 2^7)^2 : 2^{10} + (-3)^7 : (-3)^4\)
\vspace{5pt}
\footnotesize
\begin{solution}
\begin{align*}
	(2^6)^2 : 2^{10} + (-3)^3 & = 2^{12} : 2^{10} - 27\\ 
	& = 2^2 - 27\\
	& = 4 - 27 = -23
\end{align*}
\end{solution}

\end{parts}

% ------------------------------------- Domanda suddivisa in parti con bonus
%\begin{questions}
%
%\question Given the equation \(x^n + y^n = z^n\) for \(x,y,z\) and \(n\) positive
%integers.
%\begin{parts}
%\part[5] For what values of $n$ is the statement in the previous question true?
%\vspace{\stretch{1}}
%
%\part[2 \half] For $n=2$ there's a theorem with a special name. What's that name?
%\vspace{\stretch{1}}
%
%
%\bonuspart[2 \half] What famous mathematician had an elegant proof for this theorem but there was
%not enough space in the margin to write it down?
%\vspace{\stretch{1}}
%
%\end{parts}

%\droptotalpoints
%
%\question[20] Compute \[\int_{0}^{\infty} \frac{\sin(x)}{x}\]
%
%\vspace{\stretch{1}}

% ------------------------------------- Domanda Bonus
\vspace{10pt}
\bonusquestion[10]{\em Esercizio facoltativo:}

Si deve recintare un campo triangolare di lati 60, 126 e 132 metri con una rete metallica
sostenuta da paletti di cemento posti a distanze uguali tra loro ed in numero minore possibile. \\A
che distanza saranno piantati i paletti? Quanti ne serviranno?


\begin{solution}
I paletti dovranno essere distribuiti alla stessa distanza tra loro, quindi occorre cercare un divisore comune tra 42, 48 e 60. Poich� la distanza tra i paletti deve essere la massima possibile, dobbiamo cercare il Massimo Divisore Comune, quindi il M.C.D.\\
Effettuata la scomposizione in fattori primi, risulta che:\\
\begin{align*}
	42 & = 2 \cdot 3 \cdot 7\\
	48 & = 2^4 \cdot 3\\
	60 & = 2^2 \cdot 3 \cdot 5
\end{align*}
per cui il \(M.C.D. = 2 \cdot 3 = 6\)\\
I paletti andranno piantati ad una distanza di 6 metri uno dall'altro.\\ 
Poich� il perimetro del triangolo misura \(42 + 38 + 60 = 150 \, m\), dividendo questa lunghezza per 6 otterremo il numero di pali necessari: \(150 : 6 = 25\) paletti.
\end{solution}

\end{questions}
\vspace{5pt}
\noindent
\rule[2ex]{\textwidth}{1pt}
\begin{center}
{\bf Tabella dei punteggi}
\vspace{10pt}

\combinedgradetable[h][questions]
\end{center}

\end{document}
