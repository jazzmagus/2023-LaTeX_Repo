% ----------------------------------------------------------------------
% Template VERIFICA
% ----------------------------------------------------------------------
% 2020 di d!egofantinelli at jazzmagus@gmail.com
% ----------------------------------------------------------------------

% ---------------------------------- Preambolo
\documentclass[11pt, a4paper]{exam}
\usepackage[T1]{fontenc}
\usepackage{mdframed}
%\usepackage{nicefrac}
%\usepackage[applemac]{inputenc}
%\usepackage[utf8]{inputenc}
\usepackage[italian]{babel}
\usepackage[margin=1.3in]{geometry}
\usepackage{amsmath, amssymb}
\usepackage{multicol}
\usepackage{graphicx}
\usepackage{tikz}
\usepackage{upquote}
\usepackage{caption}
%\usepackage{fancyhdr}
\usepackage{float}
\renewcommand{\solutiontitle}{\noindent\textbf{Soluzione:}\par\noindent}

\newcommand{\class}{\huge {verifica di matematica}}
\newcommand{\term}{I Quadrimestre}
\newcommand{\examnum}{num.: 2}
\newcommand{\examdate}{\scriptsize 16 novembre 2020}
\newcommand{\timelimit}{45 min}

\CorrectChoiceEmphasis{\color{red}}
\SolutionEmphasis{\color{red} \footnotesize}
%\unframedsolutions
\printanswers


% ---------------------------------- Intestazione

\pagestyle{headandfoot}
\firstpageheader{\small ISS "G. A. Remondini" - Bassano del Grappa}{}{\examdate}
\runningheader{\footnotesize VERIFICA DI MATEMATICA}{\footnotesize SOLUZIONI}{Classe $3^a$ QA}
\runningheadrule

\firstpagefooter{}{}{pag. \thepage\ di \numpages}
\runningfooter{}{}{pag. \thepage\ di \numpages}
\runningfootrule

% ---------------------------------- Punteggi
\pointpoints{punto}{\em punti}
\pointformat{[{\footnotesize \thepoints}]}
\pointsinrightmargin
\setlength{\rightpointsmargin}{.2cm}
\bonuspointpoints{punto bonus}{\em punti bonus}
\bonuspointformat{[{\footnotesize \thepoints}]}
\chqword{Esercizio}
\chpword{Punti}
\chbpword{Punti Bonus}
\chsword{Punteggio}
\chtword{Totale}



\begin{document}

% ---------------------------------- Title Page
\begin{center}
\rule[2ex]{\textwidth}{0.5pt}\\
{\huge{\bf \class}}\\[20pt]
{\huge{ \term} -{ \examnum}}\\[8pt]
\rule[2ex]{\textwidth}{0.5pt}\\
\end{center}
\vspace{3cm}
\begin{tabular*}{\textwidth}{l @{\extracolsep{\fill}} r @{\extracolsep{6pt}} l}
\textbf{} & \textbf{Nome e Cognome:} & \makebox[2.5in]{\hrulefill}\\
\textbf{} &&\\
\textbf{} & \textbf{Classe:} & \makebox[2.5in]{\Large{\bf $3^a$ QA}}\\
\textbf{} &&\\
\textbf{} & Tempo a disposizione: & \makebox[2.5in]{\timelimit}\\
\textbf{} &&\\
\textbf{} &&\\
\textbf{} &&\\
\textbf{} & {\em prof.:} & \makebox[2.5in]{\em Diego Fantinelli}
\end{tabular*}\\

\vspace{5cm}

% ---------------------------------- Avvertenze

\noindent

\textbf{Avvertenze}:
\begin{itemize}
	\item La presente Verifica - che viene somministrata in modalit� DDI - contiene \numquestions \; quesiti, per un totale di \numpoints \;punti, di cui uno facoltativo del valore di \numbonuspoints \;punti, che verr� conteggiato soltanto se risulteranno svolti anche tutti i precedenti.
	\item La webcam dovr� rimanere accesa per tutto il tempo della verifica (\timelimit), salvo impossibilit� concrete di connessione; il microfono rester� spento e verr� acceso soltanto per chiarimenti e domande, che saranno consentite negli ultimi 20 min di prova.
	\item E' vietato l'utilizzo di calcolatrici scientifiche, smartphone, tablet e altri dispositivi digitali, nonch� la consultazione di testi, appunti e siti web.
	\item La verifica dovr� essere consegnata in formato digitale (pdf, jpeg, png, etc.) e dovr� essere ben leggibile; si consiglia l'inquadratura verticale.
\end{itemize}

\vfill
\newpage

% ---------------------------------- Esercizio 1
\begin{questions}
\addpoints
\question Ricordando le propriet� delle potenze, semplifica le seguenti espressioni, nell'Insieme \(\mathbb{Q}\) dei Numeri Razionali:
\begin{parts}
\part[6]  \( \left [ \left (- \dfrac{2}{5} \right )^{7} \cdot \dfrac{2}{5} \cdot \left ( - \dfrac{2}{5}  \right )^{4} \right ]^{3} : \left [ - \left ( - \dfrac{2}{5} \right )^{4} \right ]^{5} : \left [ \left ( - \dfrac{2}{5} \right )^{3} \cdot \left ( - \dfrac{2}{5} \right )^{2} \right ]^{3} \)

\begin{solution} 
\( - \dfrac{2}{5} \)

\end{solution}
%\begin{solution}
%I secondi che dovranno passare per far s� che i tre fari si accendano contemporaneamente
%dovranno essere un multiplo di 8, 12 e 15, il minimo comune multiplo.
%Effettuata la scomposizione in fattori primi, risulta che:\\
%\begin{align*}
%	8 & = 2^3\\
%	12 & = 2^2 \cdot 3\\
%	15 & = 3 \cdot 5
%\end{align*}
%per cui il \(m.c.m. = 2^3 \cdot 3 \cdot 5 = \boxed{120} \)\\
%I tre fari torneranno ad accendersi contemporaneamente dopo 120 secondi.
%\end{solution}

\addpoints

\part[6]
\( \left[ -2^{2} : \left ( 1 + \dfrac{1}{4} \right )^{2} \right ]^{2} : \left( -\dfrac{4}{5} \right)^{4}  - \left [ - 5 : \left ( 1 + \dfrac{2}{3} \right )  \right ]^{3} \cdot \left ( \dfrac{1}{3} \right )^{3}\)
\begin{solution} 
\( 17 \)

\end{solution}
%\begin{solution}
%Si tratta di un classico problema di M.C.D., si tratta cio� di calcolare il Massimo tra i Divisori Comuni di 140, 168 e 154.
%Effettuata la scomposizione in fattori primi, risulta che:\\
%\begin{align*}
%	140 & = 2^2 \cdot 5 \cdot 7\\
%	168 & = 2^3 \cdot 3 \cdot 7\\
%	154 & = 2 \cdot 7 \cdot 11
%\end{align*}
%per cui il \(M.C.D. = 2 \cdot 7\)\\
%In ogni squadra vi saranno pertanto \fbox{14} alunni.\\
%In totale si formeranno (il totale degli studenti diviso il numero di studenti per squadra):\\
%\((140 + 168 + 154) : 14 = 462 : 14 = \boxed{33}\) squadre
%
%\end{solution}

\end{parts}

% ---------------------------------- Esercizio 2
\addpoints
\question[6] Calcola le seguenti potenze nell'Insieme \(\mathbb{Q}\):

\begin{multicols}{2}
\begin{parts}

\part
\( \left \{ - [ - ( - 2 )^{-1} ]^{-2} \right \}^{-2} \)

\begin{solution} 
\( \dfrac{1}{16}\)
\end{solution}

\part
\( \left\{ \left[ - \left( - \dfrac{1}{10} \right)^{2} \right]^{-3} \right\}^{3} \)

\begin{solution} 
\( - 10^{18}\)
\end{solution}

\part
\( \left \{  \left[  \left( - \dfrac{1}{10} \right)^{-2} \right]^{-1} \right \}^{-2} \)

\begin{solution} 
\(  10^{4}\)
\end{solution}

\part
\( \{  [  ( - 3 )^{-1} ]^{2} \}^{-2} \)

\begin{solution} 
\( 81 \) oppure \( 3^{4}\)
\end{solution}

\end{parts}
\end{multicols}


% ---------------------------------- Esercizio 3

\addpoints
\question [4] Quali, tra le seguenti definizioni, sono vere?
%
%\emph{Suggerimento:} leggere con molta attenzione il testo delle risposte perch� le differenze potrebbero essere minime.
\begin{choices}
 \choice Una \emph{frazione decimale} � una frazione che ha 10 al numeratore.
 \CorrectChoice La parte intera di un numero decimale periodico semplice � quella che precede la virgola.
 \CorrectChoice Una frazione che ha 100 al denominatore genera un numero decimale finito.
 \choice La frazione generatrice di un \emph{numero periodico} ha al denominatore un numero le cui cifre sono tutte uguali a 9.
 \CorrectChoice In un numero decimale periodico misto, l'\emph{antiperiodo} appartiene alla parte decimale
\end{choices}
\vspace{3pt}

% ---------------------------------- Esercizio 4
\addpoints
\question [6]Determina la \emph{frazione generatrice} dei seguenti numeri decimali :
\vspace{3pt}
\begin{multicols}{2}
\begin{parts}
\part
\( 16,45\)
\vspace{3pt}

%\footnotesize
\begin{solution} 
\( \dfrac{1645}{100} \)
\end{solution}

%\begin{solution}
%\begin{align*}
%	[ 5^{14} : 5^{10} : 5^2 - 1] : (12^1) & = [5^2 - 1] : 12\\
%	& = [25 - 1] : 12\\
%	& = 24 : 12 = \boxed{2}
%\end{align*}
%\end{solution}
%\vspace{5pt}

\part
\( 2, \overline{9} \)
\vspace{3pt}
\begin{solution}
\( 3\)
\end{solution}

\part
\( 2, \overline{34} \)

\vspace{3pt}
\begin{solution}
\( \dfrac{232}{99}\)
\end{solution}

\part
\( 1, 2 \overline{13} \)

\vspace{3pt}
\begin{solution}
\( \dfrac{1201}{990} \)
\end{solution}

\end{parts}
\end{multicols}
% ------------------------------------- Domanda suddivisa in parti con bonus
%\begin{questions}
%
%\question Given the equation \(x^n + y^n = z^n\) for \(x,y,z\) and \(n\) positive
%integers.
%\begin{parts}
%\part[5] For what values of $n$ is the statement in the previous question true?
%\vspace{\stretch{1}}
%
%\part[2 \half] For $n=2$ there's a theorem with a special name. What's that name?
%\vspace{\stretch{1}}
%
%
%\bonuspart[2 \half] What famous mathematician had an elegant proof for this theorem but there was
%not enough space in the margin to write it down?
%\vspace{\stretch{1}}
%
%\end{parts}

%\droptotalpoints
%
%\question[20] Compute \[\int_{0}^{\infty} \frac{\sin(x)}{x}\]
%
%\vspace{\stretch{1}}


% ---------------------------------- Esercizio Bonus
%\vspace{10pt}
\bonusquestion[2]{\em Esercizio facoltativo:}

Semplifica la seguente espressione in \(\mathbb{Q}\):

\( \left \{ - [-2 \cdot (-2)^{-2} + (-2)] : \left( -\dfrac{1}{2} \right)^{-1} \right \} : \left ( - \dfrac{5}{2} + 7. \overline{7} \right ) \)

\begin{solution} 
\( - \dfrac{9}{38} \)
\end{solution}

%\begin{solution}
%I paletti dovranno essere distribuiti alla stessa distanza tra loro, quindi occorre cercare un divisore comune tra 60, 126 e 132. Poich� la distanza tra i paletti deve essere la massima possibile, dobbiamo cercare il Massimo Divisore Comune, quindi il M.C.D.\\
%Effettuata la scomposizione in fattori primi, risulta che:\\
%\begin{align*}
%	60 & = 2^2 \cdot 3 \cdot 7\\
%	126 & = 2 \cdot 3^2 \cdot 7\\
%	132 & = 2^2 \cdot 3 \cdot 11
%\end{align*}
%per cui il \(M.C.D. = 2 \cdot 3 = \boxed{6}\)\\
%
%I paletti andranno piantati ad una distanza di 6 metri uno dall'altro.\\
%
%Poich� il perimetro del triangolo misura \(60 + 126 + 132 = 318 \, m\), dividendo questa lunghezza per 6 otterremo il numero di pali necessari: \(318 : 6 = \boxed{53}\) paletti.
%\end{solution}

\end{questions}
\vspace{5pt}
\noindent
\rule[2ex]{\textwidth}{1pt}
\begin{center}
{\bf Tabella dei punteggi}
\vspace{5pt}

\combinedgradetable[h][questions]
\end{center}
\vspace{2pt}
\footnotesize La sufficienza � fissata a 18 punti, ma potrebbe subire delle modifiche in fase di correzione, al fine di garantire la validit� della prova anche in caso di andamenti troppo scostanti della media-classe.
\end{document}
  