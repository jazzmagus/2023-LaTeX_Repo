% ----------------------------------------------------------------------
% Template VERIFICA
% ----------------------------------------------------------------------
% 2020 di d!egofantinelli at jazzmagus@gmail.com
% ----------------------------------------------------------------------

% ---------------------------------- Preambolo
\documentclass[11pt, a4paper, answers]{exam}
\usepackage[T1]{fontenc}
\usepackage{mdframed}
%\usepackage{nicefrac}
%\usepackage[applemac]{inputenc}
%\usepackage[utf8]{inputenc}
\usepackage[italian]{babel}
\usepackage[margin=1.3in]{geometry}
\usepackage{amsmath, amssymb}
\usepackage{multicol}
\usepackage{graphicx}
\usepackage{tikz}
\usepackage{upquote}
\usepackage{caption}
%\usepackage{fancyhdr}
\usepackage{float}
\renewcommand{\solutiontitle}{\noindent\textbf{Soluzione:}\par\noindent}

\renewcommand{\questionshook}{%
    \setlength{\leftmargin}{0pt}%
}
\renewcommand{\choiceshook}{%
    \setlength{\leftmargin}{20pt}%
}
% ---------------------------------- Intestazione
\newcommand{\class}{\huge {Verifica di Recupero di Matematica}}
\newcommand{\term}{I Quadrimestre}
\newcommand{\examnum}{Verifica numero: 1}
\newcommand{\examdate}{25 novembre 2020}
\newcommand{\timelimit}{40 minuti}
\CorrectChoiceEmphasis{\color{red}}
\SolutionEmphasis{\color{red} \footnotesize}
\renewcommand{\solutiontitle}{\noindent\textbf{Soluzione:}\par\noindent}
% ---------------------------------- Intestazione

\pagestyle{headandfoot}
\firstpageheader{ISS "G. A. Remondini" - Bassano del Grappa (VI)}{}{\examdate}
\runningheader{\footnotesize VERIFICA di RECUPERO di MATEMATICA}{}{Classe 2\string^C}
\runningheadrule

\firstpagefooter{}{}{pag. \thepage\ di \numpages}
\runningfooter{}{}{pag. \thepage\ di \numpages}
\runningfootrule

% ---------------------------------- Punteggi
\pointpoints{punto}{\em punti}
\pointformat{[{\footnotesize \thepoints}]}
\bonuspointpoints{punto bonus}{\em punti bonus}
\bonuspointformat{[{\footnotesize \thepoints}]}
\pointsinrightmargin
\setlength{\rightpointsmargin}{.2cm}
\chqword{Esercizio}
\chpword{Punti}
\chbpword{Punti Bonus}
\chsword{Punteggio}
\chtword{Totale}

\begin{document}

% ---------------------------------- Title Page
\begin{center}
\rule[2ex]{\textwidth}{0.5pt}\\
{\huge{\bf \class}}\\[20pt]
{\huge{ \term}}\\[8pt]
\rule[2ex]{\textwidth}{0.5pt}\\
\end{center}
\vspace{3cm}
\begin{tabular*}{\textwidth}{l @{\extracolsep{\fill}} r @{\extracolsep{6pt}} l}
\textbf{} & \textbf{Nome e Cognome:} & \makebox[2.5in]{\hrulefill}\\
\textbf{} &&\\
\textbf{} & \textbf{Classe:} & \makebox[2.5in]{\Large{\bf 2 \string^ C}}\\
\textbf{} &&\\
\textbf{} & Tempo a disposizione: & \makebox[2.5in]{\timelimit}\\
\textbf{} &&\\
\textbf{} &&\\
\textbf{} &&\\
\textbf{} & {\em prof.:} & \makebox[2.5in]{\em Diego Fantinelli}
\end{tabular*}\\

\vspace{5cm}
% ---------------------------------- Avvertenze

\noindent
%\rule[2ex]{\textwidth}{0.2pt}
\textbf{Avvertenze}:
\begin{itemize}
	\item La presente Verifica - che viene somministrata in modalit� DDI - contiene \numquestions \; quesiti, per un totale di \numpoints \;punti, di cui uno facoltativo di \numbonuspoints \;punti, che verr� conteggiato soltanto se verranno svolti anche tutti i precedenti.
	\item La webcam dovr� rimanere accesa per tutto il tempo della verifica (\timelimit), salvo impossibilit� concrete di connessione; il microfono rester� spento e verr� acceso soltanto per chiarimenti e domande, che saranno consentite negli ultimi 20 min di prova.
	\item E' vietato l'utilizzo di calcolatrici scientifiche, smartphone, tablet e altri dispositivi digitali, nonch� la consultazione di testi, appunti e siti web.

\end{itemize}
%\rule[2ex]{\textwidth}{0.2pt}
\vfill
\newpage

% ---------------------------------- Esercizi
\begin{questions}

\addpoints
\question[30]
Fattorizza i seguenti polinomi con il metodo del Raccoglimento Totale:
\begin{multicols}{2}
\begin{parts}
\part
\(a^{15} + a^{10} + a^{8}\)

\begin{solution} raccoglimento totale:

	\(a^8(a^7 + a^2 + 1)\)
\end{solution} 

\vspace{.3cm}
\part
\(2x^5 + 4x^3 - 6x^2 - 2x\)

\begin{solution} raccoglimento parziale e poi totale:

	\(2x(x^4 + 2x^2 - 3x - 1) \)
\end{solution}

\vspace{.3cm}
\part
\(7x + 7 - x(x + 1)\)

\begin{solution} raccoglimento parziale e poi totale:

	\(7(x + 1) - x(x + 1) = (x + 1)(7 - x)\) 
\end{solution}

\vspace{.3cm}
\part
\(2x^6 + 2x^5 + x^3 + x^2\)

\begin{solution} raccoglimento totale e poi parziale:

	\(x^2(2x^4 + 2x^3 + x + 1) =\\ x^2[2x^3(x + 1) + (x + 1)] =\\ x^2(x + 1)(2x^3 + 1) \)
\end{solution}

\vspace{.3cm}
\part
\(36x^4y^2 - z^6\)

\begin{solution} differenza di quadrati:

	\((6x^2y + z^3)(6x^2y - z^3)\)
\end{solution}

\vspace{.3cm}
\part
\(- 12 ab + 9a^2 + 4 b^2\)

\begin{solution} quadrato di un binomio:

	\(9a^2 - 12 ab + 4 b^2 = (3a - 2b)^2\)
\end{solution}

\vspace{.3cm}
\end{parts}
\end{multicols}
\vspace{.3cm}

\question
Rispondi in modo chiaro e sintetico alle seguenti domande:\\
\begin{parts}
\part[5]
Dimostra la seguente uguaglianza: \(A^2 - 2AB + B^2 = (A - B)^2\)
%\fillwithlines{0.75in}

\begin{solution}
\begin{align*} 
(A - B)^2 & = (A - B)(A - B)\\ 
& = A^2 - AB - BA + B^2\\
& = A^2 - 2AB + B^2
\end{align*}
\end{solution}

\part[5]
Che cosa si intende con {\em Fattorizzazione} di un Polinomio?\\

\begin{solution}
{\bf Fattorizzare} un polinomio significa scriverlo come prodotto di fattori (per l'appunto) irriducibili, ovviamente di grado inferiore.\\ {\em Un polinomio � {\bf irriducibile} quando non pu� essere scritto come prodotto di due o pi� fattori di grado inferiore.}
\end{solution}

%\fillwithlines{.5in}
\end{parts}

\vspace{.5cm}

\addpoints
\question [10] \((a + b)(x - 2y)\) � la fattorizzazione di uno dei seguenti polinomi, quale?\\
\begin{choices}
\setlength{\leftmargin}{0pt}
 \choice \(x(a - b) + 2y(a + b)\) 
 \choice \(2x(x - b) + 2y(y + b)\)
 \CorrectChoice \(x(a + b) - 2y(a + b)\)
 \choice \((a - b)^2 + 2xy\)
\end{choices}
\vspace{.5cm}

\bonusquestion[10] {\em Esercizio facoltativo:}

Scomporre in fattori - se possibile - il seguente trinomio: \(- 4a^2 - 25 b^2 + 20ab\)

\begin{solution}
\begin{align*}
	- 4a^2 - 25 b^2 + 20ab & =
	-(+ 4a^2 - 20ab + 25b^2)\\ & = {\bf -(2a - 5b)^2}
\end{align*}	
\end{solution}

\end{questions}

\vspace{8pt}



\noindent
\rule[2ex]{\textwidth}{1pt}

\begin{center}
{\bf Tabella dei punteggi}
\vspace{4pt}

\combinedgradetable[h][questions]
\end{center}
\vspace{4pt}
\footnotesize La sufficienza � fissata a 35 punti, ma potr� subire delle modifiche in fase di correzione, al fine di garantire la validit� della prova anche in caso di andamenti troppo scostanti della media-classe.
\end{document}
