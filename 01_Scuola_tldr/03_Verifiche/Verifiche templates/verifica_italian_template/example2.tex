\documentclass[senzagrazie]{verifica}

\usepackage[utf8]{inputenc}
\usepackage[italian]{babel}
\usepackage[T1]{fontenc}

\begin{document}

\tipologia{verifica}
\disciplina{matematica}
\istituto{Liceo ``G. Novello'' -- Codogno}
\classe{5\tsa B}
\data{7 febbraio 2020}
\tempo{60 minuti}

\intestazione

\begin{esercizi}

\item La soluzione dell'equazione $x+1=0$ è:
\begin{test-orizz}
  \item $1$;
  \item $-1$;
  \item $0$;
  \item $+\infty$.
\end{test-orizz}
\punti{5}

\item Quale fra le seguenti è la derivata di $3x^2$? \punti{5}
\begin{test-orizz-newline}
  \item $3x$;
  \item $3$;
  \item $3x^2$;
  \item $6x$.
\end{test-orizz-newline}

\medskip

{\setlength\columnsep{2.5cm} % lo spazio tra le colonne si può modificare
\begin{multicols}{2}
\item Indicate con $p$ e $q$ due generiche proposizioni, quattro delle seguenti affermazioni sono tra loro
    logicamente equivalenti, mentre una non lo è con le altre. Quale?
\begin{test}
  \item $p$ implica $q$
  \item $q$ è condizione necessaria per $p$
  \item $p$ segue dal verificarsi di $q$
  \item $p$ solo se $q$
  \item $p$ è condizione sufficiente per~$q$\\ \punti{5}
\end{test}

\columnbreak

\item Quale delle  seguenti frasi è equivalente a

\emph{non è vero che Mario studia e ascolta la radio}

\begin{test}
  \item Mario studia e ascolta la radio
  \item Mario studia o ascolta la radio
  \item Mario non studia né ascolta la radio
  \item Mario non studia o non ascolta la radio \punti{5}
\end{test}

\end{multicols}}


\item Vero o falso? \punti[per ognuna]{2}

{\setlength\columnsep{1.3cm} % lo spazio tra le colonne si può modificare
\begin{multicols}{3}
\begin{test-verofalso}[fattorevf=.6]
  \vfitem{$1+1=2$}
  \vfitem{$3+0=0$}
  \vfitem{$3\times 3=9$}
  \vfitem{$0:3=1$}
  \vfitem{$\lim_{x\to 2}x^2=4$}
  \vfitem{$\sum_{n=0}^{1}n^2=1$}
\end{test-verofalso}
\end{multicols}}

\item Enuncia il teorema di Weierstrass.  \punti{8}

\riga{2}

\item \emph{Completa.} \doublespacing In un triangolo \dotword{rettangolo} il quadrato costruito sull'
    \dotword{ipotenusa} è uguale alla \dotword{somma} dei \dotword{quadrati} costruiti sui
    \dotword{cateti}. \punti{20} \par\singlespacing

\item Stabilisci se le seguenti affermazioni sono vere o false. \punti[per ognuna]{10}

\begin{test-verofalso}[fattorevf=.85]
  \vfitem{Il coefficiente angolare di una retta si deve \emph{sempre} indicare con la lettera $m$ e
      l'ordinata all'origine \emph{sempre} con la lettera $q$.}
  \vfitem{Se una funzione derivabile in un intervallo ha derivata positiva, allora la funzione è
      crescente in tale intervallo.}
\end{test-verofalso}

\end{esercizi}

\totpunti[/10+2]

\end{document}
