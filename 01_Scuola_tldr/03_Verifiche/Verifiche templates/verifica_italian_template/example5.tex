\documentclass[senzagrazie]{verifica}

\usepackage[utf8]{inputenc}
\usepackage[english,italian]{babel}
\usepackage[T1]{fontenc}
\usepackage{tasks}

\begin{document}

\tipologia{prova comune}
\disciplina{matematica e fisica}
\istituto{Liceo ``G. Novello'' -- Codogno}
\classe{4\tsa C}
\data{7 febbraio 2020}
\tempo{55 minuti}

\lineanome
\intestazionerighe

\begin{esercizi}

\item Data l'ellisse
\[
  9x^2+y^2+36x-4y+4=0
\]
calcola il suo centro, i valori dei semiassi, i fuochi e l'eccentricità.

\item Un aereo di linea ha una potenza sonora di circa
      $4,8\times 10^4\unit{W}$. Un operatore aeroportuale si trova
      a $25\unit{m}$ di distanza.
\begin{itemize}
  \item Calcola l'intensità e il livello sonoro percepiti dall'operatore.
  \item Calcola l'intensità e il livello sonoro a $1,5\times 10^3\unit{m}$
        di distanza.
\end{itemize}

\item Data l'equazione
\[
  \frac{x^2}{25}+\frac{y^2}{12-k}=1
\]
determina per quale valore di $k$ essa rappresenta
\begin{enumerate}[a)]
  \item un'ellisse;
  \item un'ellisse coi fuochi sull'asse $x$;
  \item un'ellisse coi fuochi sull'asse $y$;
  \item un'ellisse passante per il punto $(-2, 1)$;
  \item un'ellisse passante per il punto $(5\sqrt{2},-1)$.
\end{enumerate}

\item Risolvi le seguenti equazioni e disequazioni.
\begin{tasks}[after-item-skip=2em,column-sep=2em,item-indent=2em](2)
  \task $\dss{2\cos^2 x +3\sin^2 x=\frac{5}{2}\sin 2x}$
  \task $\dss{\frac{1 -\cos 2x}{\sqrt{3}\sin x}=\frac{\tan x}{2\cos x}}$
  \task $\dss{2\sin^2 x +3\cos x -2 \le 0}$
  \task $\dss{3\sin^2 x -2\sqrt{3}\sin x\cos x > 3\cos^2 x}$
\end{tasks}

\item \begin{otherlanguage}{english}
True or false?
\begin{test-verofalso}
  \vfitem{The speed of light in vacuum is approximately
          $3.0\times 10^8\unitx{m/s}$.}
  \vfitem{The gravitational acceleration of an object in vacuum, near
          the surface of the Earth, is denoted by $g$ and is
          approximately $9.8\unitx{m/s^2}$. At different points, $g$
          changes depending on altitude and latitude.}
\end{test-verofalso}
\end{otherlanguage}

\end{esercizi}

\vfill

% tabella per l'attribuzione dei singoli punteggi
% consultare la documentazione del pacchetto tabularx
\newcolumntype{P}[1]{>{\centering\arraybackslash}p{#1}}
\begin{flushright}
{\renewcommand{\arraystretch}{1.2}
\begin{tabularx}{\textwidth}{|X|*{12}{P{.75cm}|}}
\hline
\textbf{Esercizio} & \textbf{1} & \textbf{2} & \multicolumn{5}{c|}{\textbf{3}}
                                          & \multicolumn{4}{c|}{\textbf{4}}
                                          & \textbf{5} \\
\cline{4-12}
                   &   &   & a & b & c & d & e & a  & b  & c  & d  & \\
\hline
\hline
Punteggio          & 9 & 9 & 6 & 6 & 6 & 6 & 6 & 10 & 10 & 12 & 12 & 4 \\
\hline
Totalizzato        &   &   &   &   &   &   &   &    &    &    &    &\\
\hline
\end{tabularx}}
\end{flushright}

\vartotpunti[/12+2]

\end{document}
