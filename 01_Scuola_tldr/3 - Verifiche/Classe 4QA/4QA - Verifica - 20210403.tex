% ----------------------------------------------------------------------
% Template VERIFICA
% ----------------------------------------------------------------------
% 2020 di d!egofantinelli at jazzmagus@gmail.com
% ----------------------------------------------------------------------

% ---------------------------------- Preambolo
\documentclass[11pt, a4paper, answers]{exam}
\usepackage[T1]{fontenc}
\usepackage{mdframed}
%\usepackage{nicefrac}
%\usepackage[applemac]{inputenc}
%\usepackage[utf8]{inputenc}
\usepackage[italian]{babel}
\usepackage[margin=1.3in]{geometry}
\usepackage{amsmath,amssymb}
\usepackage{multicol}
\usepackage{graphicx}
\usepackage{tikz}
\usepackage{upquote}
\usepackage{caption}
%\usepackage{fancyhdr}
\usepackage{float}


\renewcommand{\questionshook}{%
    \setlength{\leftmargin}{0pt}%
}
\renewcommand{\choiceshook}{%
    \setlength{\leftmargin}{20pt}%
}
% ---------------------------------- Intestazione
\newcommand{\class}{\huge {Verifica di Recupero di Matematica}}
\newcommand{\term}{1� Quadrimestre}
\newcommand{\examnum}{Verifica numero: 1}
\newcommand{\examdate}{\today}
\newcommand{\timelimit}{45 minuti}
% ---------------------------------- Intestazione

\pagestyle{headandfoot}
\firstpageheader{ISS "G. A. Remondini" - Bassano del Grappa (VI)}{}{\examdate}
\runningheader{\footnotesize VERIFICA di RECUPERO di MATEMATICA}{}{Classe 4\string^QA}
\runningheadrule

\firstpagefooter{}{}{pag. \thepage\ di \numpages}
\runningfooter{}{}{pag. \thepage\ di \numpages}
\runningfootrule

% ---------------------------------- Punteggi
\pointpoints{punto}{\em punti}
\pointformat{[{\footnotesize \thepoints}]}
\bonuspointpoints{punto bonus}{\em punti bonus}
\bonuspointformat{[{\footnotesize \thepoints}]}
\pointsinrightmargin
\setlength{\rightpointsmargin}{.2cm}
\chqword{Esercizio}
\chpword{Punti}
\chbpword{Punti Bonus}
\chsword{Punteggio}
\chtword{Totale}

\begin{document}

% ---------------------------------- Title Page
\begin{center}
\rule[2ex]{\textwidth}{0.5pt}\\
{\huge{\bf \class}}\\[12pt]
{\huge - SOLUZIONI -}\\[8pt]
\rule[2ex]{\textwidth}{0.5pt}\\
\end{center}
\vspace{3cm}
\begin{tabular*}{\textwidth}{l @{\extracolsep{\fill}} r @{\extracolsep{6pt}} l}
\textbf{} & \textbf{Nome e Cognome:} & \makebox[2.5in]{\hrulefill}\\
\textbf{} &&\\
\textbf{} & \textbf{Classe:} & \makebox[2.5in]{\Large{\bf 4 \string^ QA}}\\
\textbf{} &&\\
\textbf{} & Tempo a disposizione: & \makebox[2.5in]{\timelimit}
\end{tabular*}\\[3cm]

\vspace{5cm}
% ---------------------------------- Avvertenze

\noindent
%\rule[2ex]{\textwidth}{0.2pt}
\textbf{Avvertenze}:
\begin{itemize}
	\item La presente Verifica - che viene somministrata in modalit� DDI - contiene \numquestions \; quesiti, per un totale di \numpoints \;punti, e un esercizio facoltativo, di \numbonuspoints \;punti, che verr� conteggiato soltanto se verranno svolti anche tutti i precedenti.
	\item La webcam dovr� rimanere accesa per tutto il tempo della verifica (\timelimit), salvo impossibilit� concrete di connessione; il microfono rester� spento e verr� acceso soltanto per chiarimenti e domande, che saranno consentite negli ultimi 20 min di prova.
	\item E' vietato l'utilizzo di calcolatrici scientifiche, smartphone, tablet e altri dispositivi digitali, nonch� la consultazione di testi, appunti e siti web.

\end{itemize}%\rule[2ex]{\textwidth}{0.2pt}
\vfill
\newpage

% ---------------------------------- Esercizi
\begin{questions}

\addpoints
\question
Semplifica le seguenti espressioni algebriche:\\
\begin{parts}
\part[10]
\(2xy + 3x^2 y - {\dfrac{1}{3}}xy - 4x^2y + {\dfrac{4}{3} xy}\)\\
{\footnotesize
\begin{solution}
	\(3xy - x^2y;\) \quad scrittura ottimale con fattorizzazione: \(xy(3-x)\)
\end{solution}
}
\vspace{.5cm}

%\part[15]
%\(r - {\dfrac{1}{2}}s + t - (-2r) + {\dfrac{3}{2}}s - {\dfrac{1}{2}}t - (-r) + {\dfrac{5}{2} t}\)\\
%{\footnotesize
%\begin{solution}
%	\(a(x+1)\)
%\end{solution}
%}
%\vspace{.5cm}
\part[10]
\(\left(-{\dfrac{2}{3}} a^2 b^3\right) \cdot (-9ab^2)\)\\
{\footnotesize
\begin{solution}
	\(+6a^3b^5\)
\end{solution}
}

\end{parts}
\vspace{.5cm}

%\addpoints
%\question[25]  Tre fari si accendono ad intervalli regolari.\\ Il primo si accende ogni 8 s, il secondo faro ogni 12 s, il terzo ogni 15 s. Se ad un certo istante si accendono contemporaneamente, dopo quanti secondi torneranno ad accendersi insieme?\\
%\[
%	f(x) = x^4 - 4x^2 + 4
%\]
%\emph{suggerimento:}
%
%Per determinare le radici o {\em zeri} si risolva l'equazione {\em associata}: \[x^4 - 4x^2 + 4 = 0\]
%
%\emph{osservazione:} 
%
%La determinaci\'on de la paridad (en particular) debe estar apoyada en el procedimiento algebraico que la sustenta. Dado que conoce con exactitud los puntos de inflexi\'on, los intervalos en que la curva es creciente o decreciente deben darse exactamente definidos.
%\vspace{.5cm}
%
%\addpoints
%\question[25] Dai una sintetica definizione di {\em Polinomio}:
%\fillwithlines{1in}
%\vspace{.5cm}

\question
Rispondi in modo chiaro e sintetico alle seguenti domande:\\
\begin{parts}
\part[5]
Quando due o pi� monomi si dicono {\em simili}?\ Puoi fare un esempio numerico?
%\fillwithlines{0.5in}
{\footnotesize
\begin{solution}
Due o pi� monomi si dicono simili quando presentano la stessa parte letterale; ad es sono simili: \(3ab^2, -6ab^2\) e \(\dfrac{1}{2}ab^2\).
\end{solution}
}
\vspace{.5cm}
\part[15]
Qual � la definizione di {\em Polinomio}?
%\fillwithlines{.75in}

{\footnotesize
\begin{solution}
un polinomio tipico, cio� ridotto in forma normale, � la somma algebrica di alcuni monomi non simili tra loro, vale a dire con parti letterali diverse. Ad esempio: \(x+3y-z^2\)	
\end{solution}
}
\end{parts}

\vspace{.5cm}
%\addpoints
%\question [12]Che cosa si intende con il termine {\em Polinomio}?

%\begin{oneparchoices}
% \choice Stephen Hawking 
% \choice Albert Einstein
% \choice Emmy Noether
% \choice This makes no sense
%\end{oneparchoices}

%\addpoints
%\question [20]Quali dei seguenti monomi sono {\em simili}?
%
%\begin{checkboxes}
% \choice Stephen Hawking \quad \dotfill
% \correctchoice Albert Einstein \quad \dotfill
% \choice Emmy Noether \quad \dotfill
% \choice I don't know \quad \dotfill
%\end{checkboxes}

\addpoints
\question [10]Quale delle seguenti frasi � la traduzione verbale dell'espressione \(2x^2(x^2 + y^2)\)?\\
\begin{choices}
\setlength{\leftmargin}{0pt}
 \choice Il prodotto tra il quadrato del doppio di $x$ e la somma dei quadrati di $x$ e $y$. 
 \choice Il doppio del prodotto tra il quadrato di $x$ e il quadrato della somma di $x$ con $y$.
 \choice Il prodotto tra il quadrato del doppio di $x$ e il quadrato della somma di $x$ con $y$.
 \CorrectChoice Il doppio del prodotto tra il quadrato di $x$ e la somma dei quadrati di $x$ e $y$.
\end{choices}
\vspace{.5cm}

% ------------------------------------- Domanda Bonus

\bonusquestion[5] {\em Esercizio facoltativo:}

Determina il M.C.D. e il m.c.m. fra i seguenti monomi: \[ 9a^2b^4c, \quad 3ac^4, \quad 6bc^2.\]
{\footnotesize
\begin{solution}
	\(M.C.D. = 3c\), \quad \(m.c.m. = 18a^2b^4c^4\)
\end{solution}
}
\end{questions}
%\vfill
\noindent
\rule[2ex]{\textwidth}{1pt}

\begin{center}
{\bf Tabella dei punteggi}
\vspace{10pt}

\combinedgradetable[h][questions]
\end{center}
\vspace{4pt}
\footnotesize La sufficienza � fissata a 30 punti, ma potr� subire delle modifiche in fase di correzione, al fine di garantire la validit� della prova anche nel caso in cui si riscontrino prestazioni della classe sensibilmente lontane dalla media-classe stimata.

\end{document}
