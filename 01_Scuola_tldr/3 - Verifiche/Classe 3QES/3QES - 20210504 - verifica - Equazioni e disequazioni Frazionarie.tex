% ----------------------------------------------------------------------
% Template VERIFICA
% ----------------------------------------------------------------------
% 2020 di d!egofantinelli at jazzmagus@gmail.com
% ----------------------------------------------------------------------

% ---------------------------------- Preambolo
\documentclass[11pt, a4paper]{exam}
\usepackage[T1]{fontenc}
\usepackage{mdframed}
%\usepackage{nicefrac}
%\usepackage[applemac]{inputenc}
%\usepackage[utf8]{inputenc}
\usepackage[italian]{babel}
\usepackage[margin=1.3in]{geometry}
\usepackage{amsmath,amssymb}
\usepackage{multicol}
\usepackage{graphicx}
\usepackage{tikz}
\usepackage{upquote}
\usepackage{caption}
%\usepackage{fancyhdr}
\usepackage{float}

% ---------------------------------- Command

\renewcommand{\questionshook}{%
    \setlength{\leftmargin}{0pt}%
}
\renewcommand{\choiceshook}{%
    \setlength{\leftmargin}{20pt}%
}

\newcommand{\class}{\huge {Verifica di Matematica}}
\newcommand{\term}{n. 02 | quad. 02}
%\newcommand{\examnum}{Verifica numero: 1}
\newcommand{\examdate}{04 maggio 2021}
\newcommand{\timelimit}{50 minuti}

\CorrectChoiceEmphasis{\color{red}}
\SolutionEmphasis{\color{red} \footnotesize}
\renewcommand{\solutiontitle}{\noindent\textbf{Soluzione:}\par\noindent}

% ---------------------------------- Headers and Footers

\pagestyle{headandfoot}
\firstpageheader{IIS "G. A. Remondini" - Bassano del Grappa (VI)}{}{\examdate}
\runningheader{\footnotesize VERIFICA di MATEMATICA}{}{Classe 3\string^QES}
\runningheadrule

\firstpagefooter{}{}{pag. \thepage\ di \numpages}
\runningfooter{}{}{pag. \thepage\ di \numpages}
\runningfootrule

% ---------------------------------- Punteggi
\pointpoints{punto}{\em punti}
\pointformat{[{\footnotesize \thepoints}]}
\bonuspointpoints{punto bonus}{\em punti bonus}
\bonuspointformat{[{\footnotesize \thepoints}]}
\pointsinrightmargin
\setlength{\rightpointsmargin}{.2cm}
\chqword{Esercizio}
\chpword{Punti}
\chbpword{Punti Bonus}
\chsword{Punteggio}
\chtword{Totale}

%\printanswers
\begin{document}

% ---------------------------------- Title Page
\begin{center}
\rule[2ex]{\textwidth}{0.5pt}\\
{\huge{\bf \class}}\\[12pt]
{\huge \, \term}\\[8pt]
\rule[2ex]{\textwidth}{0.5pt}\\
\end{center}
\vspace{3cm}
\begin{tabular*}{\textwidth}{l @{\extracolsep{\fill}} r @{\extracolsep{6pt}} l}
\textbf{} & \textbf{Cognome e Nome:} & \makebox[2.5in]{\hrulefill}\\
\textbf{} &&\\
\textbf{} & \textbf{Classe:} & \makebox[2.5in]{\Large{\bf 3 \string^ QES}}\\
\textbf{} &&\\
\textbf{} & Tempo a disposizione: & \makebox[2.5in]{\timelimit}
\end{tabular*}\\[3cm]
\vspace{4.5cm}

% ---------------------------------- Avvertenze

\noindent
\rule[2ex]{\textwidth}{0.2pt}
\textbf{Avvertenze}:
\begin{itemize}
	\item La presente Verifica - che viene somministrata in modalit� {\em in presenza} - contiene \numquestions \; quesiti, per un totale di \numpoints \;punti, uno dei quali facoltativo, che verr� valutato soltanto se saranno stati risolti anche tutti gli altri.
	\item Nel caso vi fossero studenti in DDI la webcam dovr� rimanere accesa per tutto il tempo della verifica (\timelimit), salvo impossibilit� concrete di connessione; il microfono rester� spento e verr� acceso soltanto per chiarimenti e domande, che saranno consentite negli ultimi 20 min di prova.
	\item E' vietato l'utilizzo di calcolatrici scientifiche, smartphone, tablet e altri dispositivi digitali, nonch� la consultazione di testi, appunti e siti web.

\end{itemize}
\vspace{5pt}
\noindent
\rule[2ex]{\textwidth}{0.2pt}
\vfill
%\newpage

% ---------------------------------- Esercizio 1
\begin{questions}

\addpoints
\question Risolvi le seguenti equazioni algebriche intere:\\
\begin{parts}

\part[4]
\((5x - 2)^2 + 1 = 5(x - 1)(x + 1) + 20x^2\)

\fillwithdottedlines{.5in}

{\footnotesize
\begin{solution}
	\(\dfrac{1}{2}\)
\end{solution}
}
\vspace{.5cm}

\part[4]
\((3 - x)(3 + x) = 3 - (x - 3)^2\)

\fillwithdottedlines{.5in}

{\footnotesize
\begin{solution}
	\(\dfrac{5}{2}\)
\end{solution}
}

\end{parts}
\vspace{.5cm}

% ---------------------------------- Esercizio 2

\addpoints
\question [6] Risolvi la seguente equazione algebrica intera:\\

%\begin{parts}
%
%\part[4]
%\(- \dfrac{1}{2} \left[ x - 2 \left(\dfrac{x - 1}{3} - \dfrac{3 - x}{2} \right)\right] - 1 = \dfrac{1}{6}\)
%
%\fillwithdottedlines{.5in}
%
%{\footnotesize
%\begin{solution}
%	\([9]\)
%\end{solution}
%}

%\vspace{.2cm}
%\part[4]
\(\dfrac{(x - 1)^2}{4} - \dfrac{(x + 1)^2}{3} = \dfrac{(1 + x )(1 - x)}{12} - \dfrac{x - 2}{6}\)

\fillwithdottedlines{.5in}

{\footnotesize
\begin{solution}
	\(\left[ -\dfrac{1}{2}\right]\)
\end{solution}
}
%
%\end{parts}

\vspace{.5cm}

% ------------------------------------- Esercizio 3
\addpoints
\question [4] Quali tra le seguenti affermazioni sono VERE?:\\

\begin{choices}
\setlength{\leftmargin}{0pt}
 \choice l'equazione \(3x = 0\) � impossibile\\
 \CorrectChoice l'equazione \(3x = 1\) ha come soluzione il {\em reciproco} di \(3\)\\
 \choice L'equazione \(x - 5 = 0\) ha come soluzione l'{\em opposto} di \(5\)\\
 \choice L'equazione \(\dfrac{3}{4}x = 0\) ha come soluzione \(x = \dfrac{4}{3}\)\\
 \choice L'equazione \(\dfrac{3}{4}x = \dfrac{5}{4}\) ha come soluzione \(x = \dfrac{1}{2}\)\\
 \CorrectChoice L'equazione \(7x = 5\) � impossibile in \(\mathbb{N}\) ma non in \(\mathbb{Q}\)\\
 \CorrectChoice L'equazione \(x = -x\) � impossibile perch� un numero non pu� essere uguale al suo opposto\\
 \CorrectChoice L'equazione \(x + 6 = x + 7\) � impossibile\\
 \choice L'equazione \((x - 1)^2 = (1 - x)^2\) � indeterminata\\
\end{choices}
\vspace{1cm}

% ------------------------------- Esercizio 4
\addpoints
\question Determina le soluzioni delle seguenti disequazioni algebriche intere:\\
 
\begin{parts}
	
\part[6] 

\((2x - 3)^2 + (1 - 3x)^2 \ge (4x - 1)(4x + 1) - 3x^2\)

\fillwithdottedlines{.5in}
\vspace{4pt}
{\footnotesize
\begin{solution}
	\(\left[ x \le \dfrac{11}{18} \right]\)
\end{solution}
}
\vspace{10pt}

\part[6] 

\( x^2 - (x + 1)^2 \ge \dfrac{x - 1}{2} - \dfrac{x + 1}{4}\)

\fillwithdottedlines{.5in}
\vspace{8pt}
{\footnotesize

\begin{solution}
	\(\left[ x \le - \dfrac{1}{9} \right]\)
\end{solution}
}
\vspace{4pt}

\end{parts}
%\vfill
%\newpage
%\vspace{2cm}

% ------------------------------- Esercizio 5 - Facoltativo (bonus points)
\noindent
\rule[1ex]{\textwidth}{0.5pt}
\addpoints
\bonusquestion[6] {\em Esercizio facoltativo:}\\

Risolvi la seguente disequazione:\\

 \( \dfrac{1}{5} \left( \dfrac{3}{2} - x \right) - \dfrac{1}{3} (x - 1) \le - \dfrac{1}{2} [x^2 - (x + 1)^2] \)

\fillwithdottedlines{.75in}
\vspace{4pt}
{\footnotesize
\begin{solution}

	\(\left[ x \ge \dfrac{2}{23}\right]\)
\end{solution}
}
\end{questions}

%\pagebreak
% ---------------------- Tabella punteggi
\vfill
\noindent
\rule[2ex]{\textwidth}{1pt}

\begin{center}
{\bf Tabella dei punteggi}
\vspace{10pt}

\combinedgradetable[h][questions]
\end{center}
\vspace{4pt}
\footnotesize La sufficienza � fissata a 18 punti, ma potr� subire delle modifiche in fase di correzione, al fine di garantire la validit� della prova anche nel caso in cui si riscontrassero prestazioni della classe sensibilmente lontane dalla media prevista.

\end{document}
