% ----------------------------------------------------------------------
% Template VERIFICA
% ----------------------------------------------------------------------
% 2020 di d!egofantinelli at jazzmagus@gmail.com
% ----------------------------------------------------------------------

% ---------------------------------- Preambolo
\documentclass[11pt, a4paper]{exam}
\usepackage[T1]{fontenc}
\usepackage{mdframed}
%\usepackage{nicefrac}
%\usepackage[applemac]{inputenc}
%\usepackage[utf8]{inputenc}
\usepackage[italian]{babel}
\usepackage[margin=1.3in]{geometry}
\usepackage{amsmath,amssymb}
\usepackage{multicol}
\usepackage{graphicx}
\usepackage{tikz}
\usepackage{upquote}
\usepackage{caption}
%\usepackage{fancyhdr}
\usepackage{float}


\printanswers
% ---------------------------------- Command
\renewcommand{\questionshook}{%
    \setlength{\leftmargin}{0pt}%
}
\renewcommand{\choiceshook}{%
    \setlength{\leftmargin}{20pt}%
}

\newcommand{\class}{\huge {Verifica di Matematica}}
\newcommand{\term}{02/01 | quad 02}
%\newcommand{\examnum}{Verifica numero: 1}
\newcommand{\examdate}{16 aplile 2021}
\newcommand{\timelimit}{50 minuti}

\CorrectChoiceEmphasis{\color{red}}
\SolutionEmphasis{\color{red} \footnotesize}
\renewcommand{\solutiontitle}{\noindent\textbf{Soluzione:}\par\noindent}
% ---------------------------------- Headers and Footers

\pagestyle{headandfoot}
\firstpageheader{IIS "G. A. Remondini" - Bassano del Grappa (VI)}{}{\examdate}
\runningheader{\footnotesize VERIFICA di Recupero di MATEMATICA}{}{Classe 2\string^C}
\runningheadrule

\firstpagefooter{}{}{pag. \thepage\ di \numpages}
\runningfooter{}{}{pag. \thepage\ di \numpages}
\runningfootrule

% ---------------------------------- Punteggi
\pointpoints{punto}{\em punti}
\pointformat{[{\footnotesize \thepoints}]}
\bonuspointpoints{punto bonus}{\em punti bonus}
\bonuspointformat{[{\footnotesize \thepoints}]}
\pointsinrightmargin
\setlength{\rightpointsmargin}{.2cm}
\chqword{Esercizio}
\chpword{Punti}
\chbpword{Punti Bonus}
\chsword{Punteggio}
\chtword{Totale}

\begin{document}

% ---------------------------------- Title Page
\begin{center}
\rule[2ex]{\textwidth}{0.5pt}\\
{\huge{\bf \class}}\\[12pt]
{\huge -\, \term \, - }\\[8pt]
\rule[2ex]{\textwidth}{0.5pt}\\
\end{center}
\vspace{3cm}
\begin{tabular*}{\textwidth}{l @{\extracolsep{\fill}} r @{\extracolsep{6pt}} l}
\textbf{} & \textbf{Cognome e Nome:} & \makebox[2.5in]{\hrulefill}\\
\textbf{} &&\\
\textbf{} & \textbf{Classe:} & \makebox[2.5in]{\Large{\bf 2 \string^ C}}\\
\textbf{} &&\\
\textbf{} & Tempo a disposizione: & \makebox[2.5in]{\timelimit}
\end{tabular*}\\[3cm]
\vspace{5cm}

% ---------------------------------- Avvertenze

\noindent
%\rule[2ex]{\textwidth}{0.2pt}
\textbf{Avvertenze}:
\begin{itemize}
	\item La presente Verifica - che viene somministrata in modalit� DDI al 50$\%$ - contiene \numquestions \;quesiti, per un totale di \numpoints \;punti, e dei quali uno facoltativo, che verr� valutato solo se saranno stati svolti prima tutti gli altri.
	\item La webcam dovr� rimanere accesa per tutto il tempo della verifica (\timelimit), salvo impossibilit� concrete di connessione; il microfono rester� spento e verr� acceso soltanto per chiarimenti e domande, che saranno consentite negli ultimi 20 min di prova.
	\item E' vietato l'utilizzo di calcolatrici scientifiche, smartphone, tablet e altri dispositivi digitali, nonch� la consultazione di testi, appunti e siti web.

\end{itemize}%\rule[2ex]{\textwidth}{0.2pt}
\vfill
\newpage

% ================================== Esercizi

% ---------------------------------- Esercizio 1
\begin{questions}
% todo completare punto a

\addpoints
\question
Semplifica le seguenti addizioni e sottrazioni con le Frazioni Algebriche e determina le Condizioni di Esistenza:\\

\begin{parts}
\part[4] \(x + \dfrac{2x + 1}{x-1}\)

\fillwithdottedlines{0.75in}

{\footnotesize
\begin{solution}

	\(\dfrac{x^2 + x + 1}{x - 1}; \, x\neq 1\)
\end{solution}
}
\vspace{.5cm}

\part[6]  
\(\dfrac{x + 2}{x + 1} - \dfrac{x + 1}{x + 2} - \dfrac{1}{x + 1}\)

\fillwithdottedlines{0.75in}

{\footnotesize
\begin{solution}
	\(\dfrac{1}{x + 2}; \, x\neq -2 \wedge x \neq -1\)
\end{solution}
}

\end{parts}
\vspace{.5cm}

% ---------------------------------- Esercizio 2

\addpoints
\question
Esegui le seguenti moltiplicazioni con le Frazioni Algebriche e determina le Condizioni di Esistenza:\\

\begin{parts}
\part[5]
 \( \left( a- \dfrac{b^2}{a}\right) : \left( 1- \dfrac{b}{a} \right)\)
\fillwithdottedlines{0.75in}
 
{\footnotesize
\begin{solution}

	\(\left[ a + b; \quad a \neq 0 \wedge a \neq b \right]\)
\end{solution}
}
\vspace{0.5cm}

\part[5] 
\( \left( \dfrac{a^2 + 4}{a + 4} - a \right) \cdot \dfrac{a + 4}{1 -a} \)

\fillwithdottedlines{0.75in}
{\footnotesize
\begin{solution}

	\([4; \, a \neq -4 \wedge a \neq  1]\)
\end{solution}
}
\vspace{0.5cm}

\end{parts}


% ---------------------------------- Esercizio 3
\question [6] Semplifica la seguente espressione e determina le condizioni di esistenza:\\

\(\left( x - \dfrac{x}{x + 1} \right) : \left( 1 - \dfrac{2x}{x - 1}\right)\cdot \left( \dfrac{1}{x^2} + \dfrac{1}{x} + 1 \right)\)

\fillwithdottedlines{0.75in}

{\footnotesize
\begin{solution}
	\(\left[ 1 - x; \quad x \neq \pm 1 \wedge x \neq 0 \right]\)
\end{solution}
}
\vspace{.5cm}

\newpage
% ---------------------------------- Esercizio 3

\addpoints
\question [4] Quale tra quelle elencate � la corretta definizione di {\em m.c.m.} tra polinomi?\\

\begin{choices}
 \choice il {\em m.c.m.} si ottiene moltiplicando i fattori non ulteriormente riducibili comuni, ciascuno preso una sola volta, col minimo esponente.
 \choice il {\em m.c.m.} si ottiene moltiplicando i fattori non ulteriormente riducibili non comuni, ciascuno preso una sola volta, col minimo esponente.
 \CorrectChoice il {\em m.c.m.} si ottiene moltiplicando i fattori non ulteriormente riducibili, comuni e non comuni, ciascuno preso una sola volta, col massimo esponente.
 \choice il {\em m.c.m.} si ottiene moltiplicando i fattori non ulteriormente riducibili, comuni e non comuni, ciascuno preso una sola volta, col minimo esponente.
 \choice nessuna delle precedenti
\end{choices}
\vspace{.5cm}

%\newpage
% ------------------------------------- Domanda Bonus
\vfill
\noindent

\rule[2ex]{\textwidth}{0.5pt}
\bonusquestion[3] {\em Esercizio facoltativo:}

Dopo aver determinato le condizioni di esistenza semplifica la seguente frazione algebrica:\\

 \[\dfrac{9 - x^2}{x^2 + 6x + 9} \cdot \dfrac{2x + 6}{2x - 6}\]
 
\fillwithdottedlines{.75in}
{\footnotesize
\begin{solution}

	\(\left[ -1 \right]\)
\end{solution}
}

\end{questions}
\vfill
%\noindent
%\rule[2ex]{\textwidth}{1pt}
%\pagebreak
\begin{center}
\rule[2ex]{\textwidth}{1pt}
\vspace{.5cm}
{\bf Tabella dei punteggi}
\vspace{10pt}

\combinedgradetable[h][questions]
\end{center}
\vspace{4pt}
\footnotesize La sufficienza � fissata a 18 punti, ma potr� subire delle modifiche in fase di correzione, al fine di garantire la validit� della prova anche nel caso in cui si riscontrassero prestazioni della classe sensibilmente lontane dalla media stimata.

\end{document}
