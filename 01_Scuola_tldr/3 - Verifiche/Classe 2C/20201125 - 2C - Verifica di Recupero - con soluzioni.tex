% ----------------------------------------------------------------------
% Template VERIFICA
% ----------------------------------------------------------------------
% 2020 di d!egofantinelli at jazzmagus@gmail.com
% ----------------------------------------------------------------------

% ---------------------------------- Preambolo
\documentclass[11pt, a4paper, answers]{exam}
\usepackage[T1]{fontenc}
\usepackage{mdframed}
%\usepackage{nicefrac}
%\usepackage[applemac]{inputenc}
%\usepackage[utf8]{inputenc}
\usepackage[italian]{babel}
\usepackage[margin=1.3in]{geometry}
\usepackage{amsmath, amssymb}
\usepackage{multicol}
\usepackage{graphicx}
\usepackage{tikz}
\usepackage{upquote}
\usepackage{caption}
%\usepackage{fancyhdr}
\usepackage{float}


%\checkboxchar{$\square$}
%\checkedchar{$\boxtimes$}
%\CorrectChoiceEmphasis{}  % solo per non avere bold nelle risposte corrette
%

%% Create a Matching question format
\newcommand*\Matching[1]{
\ifprintanswers
    \textbf{#1}
\else
    \rule{1in}{0.5pt}
\fi
}
\newlength\matchlena
\newlength\matchlenb
\settowidth\matchlena{\rule{1.1in}{0pt}}
\newcommand\MatchQuestion[2]{%
    \setlength\matchlenb{\linewidth}
    \addtolength\matchlenb{-\matchlena}
    \parbox[t]{\matchlena}{\Matching{#1}}\enspace\parbox[t]{\matchlenb}{#2}}

\renewcommand{\thequestion}{\arabic{question}}
\renewcommand{\thepartno}{\alph{partno}}
\renewcommand{\thechoice}{\alph{choice}}

\renewcommand{\checkboxeshook}{
  \setlength{\labelsep}{2.4em}
  \setlength{\leftmargin}{4em}
}

\newcommand{\ChoiceLabel}[1]{\hspace{-1.6em}\makebox[1.6em][l]{\textbf{#1.}}\ignorespaces}

\newcommand{\WrongChoice}[1]{\choice \ChoiceLabel{#1}}
\newcommand{\RightChoice}[1]{\correctchoice \ChoiceLabel{#1}}
\newcommand{\Item}[1]{\hspace{-17pt}\makebox[17pt][l]{\textbf{#1.}}\ignorespaces}


\renewcommand{\solutiontitle}{\noindent\textbf{Soluzione:}\par\noindent}

\renewcommand{\questionshook}{%
    \setlength{\leftmargin}{0pt}%
}
\renewcommand{\choiceshook}{%
    \setlength{\leftmargin}{20pt}%
}
% ---------------------------------- Intestazione
\newcommand{\class}{\huge {Verifica di Recupero di Matematica}}
\newcommand{\term}{I Quadrimestre}
\newcommand{\examnum}{Verifica numero: 1}
\newcommand{\examdate}{25 novembre 2020}
\newcommand{\timelimit}{40 minuti}
\CorrectChoiceEmphasis{\color{red}}
\SolutionEmphasis{\color{red} \footnotesize}
\renewcommand{\solutiontitle}{\noindent\textbf{Soluzione:}\par\noindent}
% ---------------------------------- Intestazione

\pagestyle{headandfoot}
\firstpageheader{ISS "G. A. Remondini" - Bassano del Grappa (VI)}{}{\examdate}
\runningheader{\footnotesize VERIFICA di RECUPERO di MATEMATICA}{}{Classe 2\string^C}
\runningheadrule

\firstpagefooter{}{}{pag. \thepage\ di \numpages}
\runningfooter{}{}{pag. \thepage\ di \numpages}
\runningfootrule

% ---------------------------------- Punteggi
\pointpoints{punto}{\em punti}
\pointformat{[{\footnotesize \thepoints}]}
\bonuspointpoints{punto bonus}{\em punti bonus}
\bonuspointformat{[{\footnotesize \thepoints}]}
\pointsinrightmargin
\setlength{\rightpointsmargin}{.2cm}
\chqword{Esercizio}
\chpword{Punti}
\chbpword{Punti Bonus}
\chsword{Punteggio}
\chtword{Totale}

\begin{document}

% ---------------------------------- Title Page
\begin{center}
\rule[2ex]{\textwidth}{0.5pt}\\
{\huge{\bf \class}}\\[20pt]
{\huge{ \term}}\\[8pt]
\rule[2ex]{\textwidth}{0.5pt}\\
\end{center}
\vspace{3cm}
\begin{tabular*}{\textwidth}{l @{\extracolsep{\fill}} r @{\extracolsep{6pt}} l}
\textbf{} & \textbf{Nome e Cognome:} & \makebox[2.5in]{\hrulefill}\\
\textbf{} &&\\
\textbf{} & \textbf{Classe:} & \makebox[2.5in]{\Large{\bf 2 \string^ C}}\\
\textbf{} &&\\
\textbf{} & Tempo a disposizione: & \makebox[2.5in]{\timelimit}\\
\textbf{} &&\\
\textbf{} &&\\
\textbf{} &&\\
\textbf{} & {\em prof.:} & \makebox[2.5in]{\em Diego Fantinelli}
\end{tabular*}\\

\vspace{5cm}
% ---------------------------------- Avvertenze

\noindent
%\rule[2ex]{\textwidth}{0.2pt}
\textbf{Avvertenze}:
\begin{itemize}
	\item La presente Verifica - che viene somministrata in modalità DDI - contiene \numquestions \; quesiti, per un totale di \numpoints \;punti, di cui uno facoltativo di \numbonuspoints \;punti, che verrà conteggiato soltanto se verranno svolti anche tutti i precedenti.
	\item La webcam dovrà rimanere accesa per tutto il tempo della verifica (\timelimit), salvo impossibilità concrete di connessione; il microfono resterà spento e verrà acceso soltanto per chiarimenti e domande, che saranno consentite negli ultimi 20 min di prova.
	\item E' vietato l'utilizzo di calcolatrici scientifiche, smartphone, tablet e altri dispositivi digitali, nonché la consultazione di testi, appunti e siti web.

\end{itemize}
%\rule[2ex]{\textwidth}{0.2pt}
\vfill
\newpage


% =========================================== VERIFICA 


% ------------------------------------------- Esercizio #1

\begin{questions}

\addpoints
\question[10]
Fattorizza i seguenti polinomi con il metodo del suggerito:
	\begin{multicols}{2}

	\begin{parts}

\part
\(a^{15} + a^{10} + a^{8}\)

{\footnotesize {\emph{Suggerimento:} Raccoglimento Totale}}

\begin{solution}

\(a^8(a^7 + a^2 + 1)\)
\end{solution} 

\vspace{.5cm}
\part
\( a^{10}x^7 + a^{8}x^5 - a^5x^3 \)

{\footnotesize {\emph{Suggerimento:} Raccoglimento Totale}}

\begin{solution} 
\( a^5x^3(a^5x^4 + a^3x^2 - 1) \)
\end{solution}

\vspace{0.3cm}
\part
\(7x + 7 - x(x + 1)\)

{\footnotesize {\emph{Suggerimento:} Raccoglimento Parziale e poi Totale}}

\begin{solution} 
\(7(x + 1) - x(x + 1) = (x + 1)(7 - x)\) 
\end{solution}

\vspace{.3cm}
\part
\(2x^6 + 2x^5 + x^2\)

{\footnotesize {\emph{Suggerimento:} Raccoglimento Totale}}

\begin{solution}
\(x^2(2x^4 + 2x^3 + 1)  \)
\end{solution}


\end{parts}
\end{multicols}


% ------------------------------------------- Esercizio #2

\addpoints
\question [8] Quali delle seguenti fattorizzazioni sono corrette?

\begin{multicols}{2}
\begin{choices}
\choice
\(48x^2 - 4 = (6x + 2) \cdot (6x - 2)\)
\vspace{0.5\baselineskip}

\choice
\((x^4 - 9y^2) \cdot (x^4 - 9y^2) = (x^8 -49y)^4\)
\vspace{0.5\baselineskip}

\choice
\(5x - 15 + x(3 - x) = (x + 3) \cdot (x - 5)\)
\vspace{0.5\baselineskip}

\CorrectChoice
\((5a - 2b)^2 = 25a^2 - 20ab + 4b^2\)

\end{choices}
\end{multicols}


% ------------------------------------------- Esercizio #3 

\addpoints
\question Completa le seguenti definizioni:

\begin{parts}
\vspace{.3cm}
\part [6]

{\emph{Fattorizzare}} un polinomio, che è generalmente espresso come \fillin[somma][40pt] algebrica di monomi, nel \fillin[prodotto][45pt]  di altri polinomi di grado \fillin[inferiore][45pt] a quello del polinomio assegnato inizialmente.

\vspace{.5cm}

\part [6]

Il Quadrato di un Binomio è un {\fillin[Prodotto][45pt]} Notevole e la sua espressione è la seguente:\\ 

\( (a + b )^2 = (\) \fillin[\(a + b \)][30pt] \() \cdot (a + b) = a^2 \, + \) \fillin[\( 2ab\)][20pt] \( + \, b^2 \)

\end{parts}
\vspace{.6cm}

% ------------------------------------------- Esercizio #4 - Facoltativo (bonus points)

\addpoints
\bonusquestion[8] {\em Esercizio facoltativo:}

Esegui il seguente Prodotto Notevole: \[ \left(- \dfrac {3}{4}a + 4b^2 \right)^2 \]

\begin{solution}
\[ \left( \dfrac {9}{16}a^2 - 6ab^2 + 16b^4 \right)^2 \]
\end{solution}

\end{questions}

\noindent
\rule[2ex]{\textwidth}{1pt}

\begin{center}
{\bf Tabella dei punteggi}
\vspace{10pt}

\combinedgradetable[h][questions]
\end{center}
\vspace{4pt}
\footnotesize La sufficienza è fissata a 18 punti, ma potrà subire delle modifiche in fase di correzione, al fine di garantire la validità  della prova anche in caso di andamenti troppo scostanti della media-classe.
\end{document}
