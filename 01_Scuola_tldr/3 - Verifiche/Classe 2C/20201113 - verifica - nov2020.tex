% ----------------------------------------------------------------------
% Template VERIFICA
% ----------------------------------------------------------------------
% 2020 di d!egofantinelli at jazzmagus@gmail.com
% ----------------------------------------------------------------------

% ---------------------------------- Preambolo
\documentclass[11pt, a4paper]{exam}
\usepackage[T1]{fontenc}
\usepackage{mdframed}
%\usepackage{nicefrac}
%\usepackage[applemac]{inputenc}
%\usepackage[utf8]{inputenc}
\usepackage[italian]{babel}
\usepackage[margin=1.3in]{geometry}
\usepackage{amsmath, amssymb}
\usepackage{multicol}
\usepackage{graphicx}
\usepackage{tikz}
\usepackage{upquote}
\usepackage{caption}
%\usepackage{fancyhdr}
\usepackage{float}


\renewcommand{\questionshook}{%
    \setlength{\leftmargin}{0pt}%
}
\renewcommand{\choiceshook}{%
    \setlength{\leftmargin}{20pt}%
}
% ---------------------------------- Intestazione
\newcommand{\class}{\huge {Verifica di Matematica}}
\newcommand{\term}{I Quadrimestre}
\newcommand{\examnum}{Verifica numero: 1}
\newcommand{\examdate}{13 novembre 2020}
\newcommand{\timelimit}{50 minuti}
% ---------------------------------- Intestazione

\pagestyle{headandfoot}
\firstpageheader{ISS "G. A. Remondini" - Bassano del Grappa (VI)}{}{\examdate}
\runningheader{\footnotesize VERIFICA di MATEMATICA}{}{Classe 2\string^C}
\runningheadrule

\firstpagefooter{}{}{pag. \thepage\ di \numpages}
\runningfooter{}{}{pag. \thepage\ di \numpages}
\runningfootrule

% ---------------------------------- Punteggi
\pointpoints{punto}{\em punti}
\pointformat{[{\footnotesize \thepoints}]}
\bonuspointpoints{punto bonus}{\em punti bonus}
\bonuspointformat{[{\footnotesize \thepoints}]}
\pointsinrightmargin
\setlength{\rightpointsmargin}{.2cm}
\chqword{Esercizio}
\chpword{Punti}
\chbpword{Punti Bonus}
\chsword{Punteggio}
\chtword{Totale}

\begin{document}

% ---------------------------------- Title Page
\begin{center}
\rule[2ex]{\textwidth}{0.5pt}\\
{\huge{\bf \class}}\\[20pt]
{\huge{ \term} -{ \examnum}}\\[8pt]
\rule[2ex]{\textwidth}{0.5pt}\\
\end{center}
\vspace{3cm}
\begin{tabular*}{\textwidth}{l @{\extracolsep{\fill}} r @{\extracolsep{6pt}} l}
\textbf{} & \textbf{Nome e Cognome:} & \makebox[2.5in]{\hrulefill}\\
\textbf{} &&\\
\textbf{} & \textbf{Classe:} & \makebox[2.5in]{\Large{\bf 2 \string^ C}}\\
\textbf{} &&\\
\textbf{} & Tempo a disposizione: & \makebox[2.5in]{\timelimit}
\end{tabular*}\\[3cm]

\vspace{5cm}
% ---------------------------------- Avvertenze

\noindent
%\rule[2ex]{\textwidth}{0.2pt}
\textbf{Avvertenze}:
\begin{itemize}
	\item La presente Verifica - che viene somministrata in modalit� DDI - contiene \numquestions \; quesiti, per un totale di \numpoints \;punti, e un esercizio facoltativo, di \numbonuspoints \;punti, che verr� conteggiato soltanto se verranno svolti anche tutti i precedenti.
	\item La webcam dovr� rimanere accesa per tutto il tempo della verifica (\timelimit), salvo impossibilit� concrete di connessione; il microfono rester� spento e verr� acceso soltanto per chiarimenti e domande, che saranno consentite negli ultimi 20 min di prova.
	\item E' vietato l'utilizzo di calcolatrici scientifiche, smartphone, tablet e altri dispositivi digitali, nonch� la consultazione di testi, appunti e siti web.

\end{itemize}
%\rule[2ex]{\textwidth}{0.2pt}
\vfill
\newpage

% ---------------------------------- Esercizi
\begin{questions}

\addpoints
\question[30]
Fattorizza i seguenti polinomi con il metodo che ritieni pi� opportuno:
\begin{multicols}{2}
\begin{parts}
\part
\(ax + x + a + 1\)
{\footnotesize
\begin{solution}
	\(a(x+1)\)
\end{solution}
}
\vspace{.3cm}
\part
\(2a^3 + a^2 - 6a -3\)
{\footnotesize
\begin{solution}
	\(a(x+1)\)
\end{solution}
}
\vspace{.3cm}
\part
\(7x + 7 - x(x + 1)\)
{\footnotesize
\begin{solution}
	\(a(x+1)\)
\end{solution}
}
\vspace{.3cm}
\part
\(2x^6 + 2x^5 + x^3 + x^2\)
{\footnotesize
\begin{solution}
	\(a(x+1)\)
\end{solution}
}
\vspace{.3cm}
\part
\(36x^4y^2 - z^6\)
{\footnotesize
\begin{solution}
	\(a(x+1)\)
\end{solution}
}
\vspace{.3cm}
\part
\(- 12 ab + 9a^2 + 4 b^2\)
{\footnotesize
\begin{solution}
	\(a(x+1)\)
\end{solution}
}
\vspace{.3cm}
\end{parts}
\end{multicols}
\vspace{.3cm}

%\addpoints
%\question[25]  Tre fari si accendono ad intervalli regolari.\\ Il primo si accende ogni 8 s, il secondo faro ogni 12 s, il terzo ogni 15 s. Se ad un certo istante si accendono contemporaneamente, dopo quanti secondi torneranno ad accendersi insieme?\\
%\[
%	f(x) = x^4 - 4x^2 + 4
%\]
%\emph{suggerimento:}
%
%Per determinare le radici o {\em zeri} si risolva l'equazione {\em associata}: \[x^4 - 4x^2 + 4 = 0\]
%
%\emph{osservazione:} 
%
%La determinaci\'on de la paridad (en particular) debe estar apoyada en el procedimiento algebraico que la sustenta. Dado que conoce con exactitud los puntos de inflexi\'on, los intervalos en que la curva es creciente o decreciente deben darse exactamente definidos.
%\vspace{.5cm}
%
%\addpoints
%\question[25] Dai una sintetica definizione di {\em Polinomio}:
%\fillwithlines{1in}
%\vspace{.5cm}

\question
Rispondi in modo chiaro e sintetico alle seguenti domande:\\
\begin{parts}
\part[5]
Dimostra la seguente uguaglianza: \(A^2 - 2AB + B^2 = (A - B)^2\)
\fillwithlines{0.75in}

{\footnotesize
\begin{solution}
\begin{align*}
(A - B)^2 & = (A - B)(A - B)\\ 
& = A^2 - AB - BA + B^2\\
& = A^2 - 2AB + B^2
\end{align*}
\end{solution}
}
\vspace{.5cm}
\part[5]
Che cosa si intende con {\em Fattorizzazione} di un Polinomio?\\
{\footnotesize
\begin{solution}
{\bf Fattorizzare} un polinomio significa scriverlo come prodotto di fattori (per l'appunto) irriducibili, ovviamente di grado inferiore.\\ Un polinomio � {\bf irriducibile} quando non pu� essere scritto come prodotto di due o pi� fattori di grado inferiore.	
\end{solution}
}
\fillwithlines{.5in}
\end{parts}

\vspace{.5cm}
%\addpoints
%\question [12]Che cosa si intende con il termine {\em Polinomio}?

%\begin{oneparchoices}
% \choice Stephen Hawking 
% \choice Albert Einstein
% \choice Emmy Noether
% \choice This makes no sense
%\end{oneparchoices}

%\addpoints
%\question [20]Quali dei seguenti monomi sono {\em simili}?
%
%\begin{checkboxes}
% \choice Stephen Hawking \quad \dotfill
% \correctchoice Albert Einstein \quad \dotfill
% \choice Emmy Noether \quad \dotfill
% \choice I don't know \quad \dotfill
%\end{checkboxes}

\addpoints
\question [10] \((a + b)(x - 2y)\) � la fattorizzazione di uno dei seguenti polinomi, quale?\\
\begin{choices}
\setlength{\leftmargin}{0pt}
 \choice \(x(a - b) + 2y(a + b)\) 
 \choice \(2x(x - b) + 2y(y + b)\)
 \CorrectChoice \(x(a + b) - 2y(a + b)\)
 \choice \((a - b)^2 + 2xy\)
\end{choices}
\vspace{.5cm}

%\addpoints
%\question[25] 
%En la funci\'on dada se garantiza que hay tres puntos de inflexi\'on ubicados en las raices o ceros y el $y_i$. Determine: Dominio, Rango, Tipo de funci\'on (inyectiva, sobreyectiva o biyectiva), paridad e intervalos en los que es creciente o decreciente y construya una aproximaci\'on gr\'afica del lugar geom\'etrico de la funci\'on.
%\[
%	f(x) = x^4 - 4x^2 + 4
%\]
%\emph{Suggerimento:} Para determinar las raices o ceros resuelva la ecuaci\'on $$x^4 - 4x^2 + 4 = 0$$
%
%\emph{Osservazione:} La determinaci\'on de la paridad (en particular) debe estar apoyada en el procedimiento algebraico que la sustenta. Dado que conoce con exactitud los puntos de inflexi\'on, los intervalos en que la curva es creciente o decreciente deben darse exactamente definidos.
%
%\addpoints
%\question[25]  En la funci\'on dada se garantiza que hay tres puntos de inflexi\'on ubicados en las raices o ceros y el $y_i$. Determine: Dominio, Rango, Tipo de funci\'on (inyectiva, sobreyectiva o biyectiva), paridad e intervalos en los que es creciente o decreciente y construya una aproximaci\'on gr\'afica del lugar geom\'etrico de la funci\'on.
%\[
%	f(x) = x^4 - 4x^2 + 4
%\]
%\emph{Sugerencia:} Para determinar las raices o ceros resuelva la ecuaci\'on $x^4 - 4x^2 + 4 = 0$
%
%\emph{Observaci\'on:} La determinaci\'on de la paridad (en particular) debe estar apoyada en el procedimiento algebraico que la sustenta. Dado que conoce con exactitud los puntos de inflexi\'on, los intervalos en que la curva es creciente o decreciente deben darse exactamente definidos.
%
%\addpoints
%\question[25] Semplifica la seguente espressione algebrica:
%	\[
%		\lim _{x \to 1} \frac{x^3 - 1}{x^2 - 1}
%	\]
%
%\addpoints
%\question[25] Determine, si los hay, los n\'umeros en los que la funci\'on dada es discontinua.
%	\[
%		f(x) = (x^2 - 9x + 18)^{-1}
%	\]
%\vspace{.5cm}
% ------------------------------------- Domanda suddivisa in parti con bonus
%\begin{questions}
%
%\question Given the equation \(x^n + y^n = z^n\) for \(x,y,z\) and \(n\) positive
%integers.
%\begin{parts}
%\part[5] For what values of $n$ is the statement in the previous question true?
%\vspace{\stretch{1}}
%
%\part[2 \half] For $n=2$ there's a theorem with a special name. What's that name?
%\vspace{\stretch{1}}
%
%
%\bonuspart[2 \half] What famous mathematician had an elegant proof for this theorem but there was
%not enough space in the margin to write it down?
%\vspace{\stretch{1}}
%
%\end{parts}

%\droptotalpoints
%
%\question[20] Compute \[\int_{0}^{\infty} \frac{\sin(x)}{x}\]
%
%\vspace{\stretch{1}}

% ------------------------------------- Domanda Bonus

%\bonusquestion[20] {\em Esercizio facoltativo:}\\Si deve recintare un campo triangolare di lati 60, 126 e 132 metri con una rete metallica
%sostenuta da paletti di cemento posti a distanze uguali tra loro ed in numero minore possibile. A
%che distanza saranno piantati i paletti? Quanti ne saranno?
%\end{questions}

\bonusquestion[10] {\em Esercizio facoltativo:}

Scomporre in fattori - se possibile - il seguente trinomio: \(- 4a^2 - 25 b^2 + 20ab\)
{\footnotesize
\begin{solution}
\begin{align*}
	- 4a^2 - 25 b^2 + 20ab & =
	-(+ 4a^2 - 20ab + 25b^2)\\ & = {\bf -(2a - 5b)^2}
\end{align*}	
\end{solution}
}
\end{questions}

\vspace{8pt}
%\addpoints
%\question[25] Encuentre $D_xy$
%\[
%	y = \frac{(x + 1)^2}{3x - 4}
%\]
%
%\end{questions}

%\begin{table}[h]
%\centering
%	\caption{Relazione tra $f$ e $f'$.}
%	\def\arraystretch{1.5}
%	\begin{tabular}{c|c|r} % con |c| si definiscono le colonne e poi si separano le righe tra loro con \hline
%
%	{$f(x)$} & {$f'(x)$}\\ \hline
%	$x>0$ & La funzione $f(x)$ è \emph{crescente}. & prova 1\\
%	\hline
%	$x<0$ & La funzione $f(x)$ è \emph{decrescente}. & prova 2\\
%	\hline	
%	$x<0$ & La funzione $f(x)$ è \emph{costante}. & prova 3\\	
%	\end{tabular}
%\end{table}

%\vfill
%% ================================= Bloque de instrucciones
%\noindent
%\rule[2ex]{\textwidth}{0.5pt}
%\textbf{Istruzioni}: Este examen contiene \numquestions \;planteamientos que corresponde a \numpoints \;puntos de la valoración final. Tenga presente que no esta autorizada la comunicación con sus compañeros, ni el uso de ayudas computacionales (calculadora, celular, etc) y que resolver el pliego a l\'apiz implica renunciar a cualquier reclamación después de entregados los resultados.\\

% Tabla de calificaciones
%\vspace{8pt}

\noindent
\rule[2ex]{\textwidth}{1pt}

\begin{center}
{\bf Tabella dei punteggi}
\vspace{4pt}

\combinedgradetable[h][questions]
\end{center}
\vspace{4pt}
\footnotesize La sufficienza � fissata a 30 punti, ma potr� subire delle modifiche in fase di correzione, al fine di garantire la validit� della prova anche nel caso in cui si riscontrino prestazioni della classe sensibilmente lontane dalla media-classe stimata.
\end{document}
