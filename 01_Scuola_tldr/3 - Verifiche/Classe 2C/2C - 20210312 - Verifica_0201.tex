% ----------------------------------------------------------------------
% Template VERIFICA
% ----------------------------------------------------------------------
% 2020 di d!egofantinelli at jazzmagus@gmail.com
% ----------------------------------------------------------------------

% ---------------------------------- Preambolo
\documentclass[11pt, a4paper]{exam}
\usepackage[T1]{fontenc}
\usepackage{mdframed}
%\usepackage{nicefrac}
%\usepackage[applemac]{inputenc}
%\usepackage[utf8]{inputenc}
\usepackage[italian]{babel}
\usepackage[margin=1.3in]{geometry}
\usepackage{amsmath,amssymb}
\usepackage{multicol}
\usepackage{graphicx}
\usepackage{tikz}
\usepackage{upquote}
\usepackage{caption}
%\usepackage{fancyhdr}
\usepackage{float}


\printanswers
% ---------------------------------- Command
\renewcommand{\questionshook}{%
    \setlength{\leftmargin}{0pt}%
}
\renewcommand{\choiceshook}{%
    \setlength{\leftmargin}{20pt}%
}

\newcommand{\class}{\huge {Verifica di Matematica}}
\newcommand{\term}{\#01 | 1 quadrimestre}
%\newcommand{\examnum}{Verifica numero: 1}
\newcommand{\examdate}{30 settembre 2021}
\newcommand{\timelimit}{60 minuti}

\CorrectChoiceEmphasis{\color{orange}}
\SolutionEmphasis{\color{orange} \footnotesize}
\renewcommand{\solutiontitle}{\noindent\textbf{Soluzione:}\par\noindent}
% ---------------------------------- Headers and Footers

\pagestyle{headandfoot}
\firstpageheader{ITIS "E. Fermi" - Bassano del Grappa (VI)}{}{\examdate}
\runningheader{\footnotesize VERIFICA di MATEMATICA}{}{Classe 2\string^C}
\runningheadrule

\firstpagefooter{}{}{pag. \thepage\ di \numpages}
\runningfooter{}{}{pag. \thepage\ di \numpages}
\runningfootrule

% ---------------------------------- Punteggi
\pointpoints{punto}{\em punti}
\pointformat{[{\footnotesize \thepoints}]}
\bonuspointpoints{punto bonus}{\em punti bonus}
\bonuspointformat{[{\footnotesize \thepoints}]}
\pointsinrightmargin
\setlength{\rightpointsmargin}{.2cm}
\chqword{Esercizio}
\chpword{Punti}
\chbpword{Punti Bonus}
\chsword{Punteggio}
\chtword{Totale}

\begin{document}

% ---------------------------------- Title Page
\begin{center}
\rule[2ex]{\textwidth}{0.5pt}\\
{\huge{\bf \class}}\\[12pt]
{\huge -\, \term \, - }\\[8pt]
\rule[2ex]{\textwidth}{0.5pt}\\
\end{center}
\vspace{3cm}
\begin{tabular*}{\textwidth}{l @{\extracolsep{\fill}} r @{\extracolsep{6pt}} l}
\textbf{} & \textbf{Cognome e Nome:} & \makebox[2.5in]{\hrulefill}\\
\textbf{} &&\\
\textbf{} & \textbf{Classe:} & \makebox[2.5in]{\Large{\bf 2 \string^ C}}\\
\textbf{} &&\\
\textbf{} & Tempo a disposizione: & \makebox[2.5in]{\timelimit}
\end{tabular*}\\[3cm]
\vspace{5cm}

% ---------------------------------- Avvertenze

\noindent
%\rule[2ex]{\textwidth}{0.2pt}
\textbf{Avvertenze}:
\begin{itemize}
	\item La presente Verifica di Recupero - che viene somministrata in modalit� DDI - contiene \numquestions \;quesiti, per un totale di \numpoints \;punti.
	\item La webcam dovr� rimanere accesa per tutto il tempo della verifica (\timelimit), salvo impossibilit� concrete di connessione; il microfono rester� spento e verr� acceso soltanto per chiarimenti e domande, che saranno consentite negli ultimi 20 min di prova.
	\item E' vietato l'utilizzo di calcolatrici scientifiche, smartphone, tablet e altri dispositivi digitali, nonch� la consultazione di testi, appunti e siti web.

\end{itemize}%\rule[2ex]{\textwidth}{0.2pt}
\vfill
\newpage

% ================================== Esercizi ==============================

% ---------------------------------- Esercizio 1
\begin{questions}
% todo completare punto a

\addpoints
\question
Determina le condizioni di esistenza delle seguenti Frazioni Algebriche:\\

\begin{parts}
\part[5] \(\dfrac{35a^{2}xy}{10ax^2}\)

%\fillwithlines{0.5in}
{\footnotesize
\begin{solution}
	\(\dfrac{7a}{2x}\)
\end{solution}
}
\vspace{.5cm}

\part[5]  
\(\dfrac{2}{4x + 10}\)

%\fillwithlines{0.5in}

{\footnotesize
\begin{solution}
	\(\nexists \, x \in \mathbb{R}\)
\end{solution}
}
\vspace{.5cm}

\part[5]  
\(\dfrac{a^3 - 8}{a^3 + 2a^2 + 4a}\)

%\fillwithlines{0.5in}

{\footnotesize
\begin{solution}
	\(\dfrac{a - 2}{a}\)
\end{solution}
}


\end{parts}
\vspace{.5cm}

% ---------------------------------- Esercizio 2

\addpoints
\question
determina le {\em condizioni di esistenza} e, ove possibile, semplifica le seguenti frazioni algebriche utilizzando i metodi di fattorizzazione;\\
\begin{parts}
\part[2]
 \(\dfrac{a^2 + 2ab + b^2}{a^2 - b^2}\)
% \fillwithlines{.5in}
{\footnotesize
\begin{solution}

	\(\dfrac{a + b}{a - b}\)\\
	
	\(C.E. = \{a \neq b\}\)
\end{solution}
}
\vspace{0.5cm}

\part[3] 
\(\dfrac{4x^3 + 8x^2 + 4x}{4x^2 - 4x}\)
%\fillwithlines{.5in}
{\footnotesize
\begin{solution}

	\(\dfrac{x + 1}{x - 1}\)\\
	
	\(C.E. = \{x \neq 1\}\)
\end{solution}
}
\vspace{0.5cm}
\part[3] 
\(\dfrac{t^5 - 5t^4 + t - 5}{t^2 - 25}\)
% \fillwithlines{.5in}
{\footnotesize
\begin{solution}

	\(\dfrac{t^4 + 1}{t + 5}\)\\
	
	\(C.E. = \{t \neq -5\}\)
\end{solution}
}
\vspace{.5cm}
\end{parts}


% ---------------------------------- Esercizio 3
\question [4] Verifica se le seguenti razioni algebriche sono {\em equivalenti}?
%\fillwithlines{0.5in}

\begin{parts}
\part[3] \(\dfrac{a - b}{a - b} = \quad \ldots \quad = \dfrac{a^2 - ab}{a^2 - ab}\)
%{\footnotesize
%\begin{solution}
%	\(36x^2 - 25y^2\)
%\end{solution}
%}
%\fillwithlines{.5in}

\vspace{.5cm}

\part[3] \(\dfrac{2xb^2}{4x - 2xb} = \quad \ldots \quad = \dfrac{b^2}{2 - b}\)
%{\footnotesize
%\begin{solution}
%	\((4a^2 - 12ab + 9b^2)\)
%\end{solution}
%}
%\fillwithlines{.5in}
\end{parts}
\vspace{.5cm}


% ---------------------------------- Esercizio 3

\addpoints
\question [2] Quali tra quelle elencate non sono {\em Frazioni algebriche}?\\

\begin{oneparchoices}
 \CorrectChoice \(\dfrac{x^3 - 4x^2 -x +3}{27}\)
 \choice \(\dfrac{4bcde}{a}\)
 \CorrectChoice \(\dfrac{x^3 - 4}{24}\)
 \choice \(\dfrac{1}{x}\)
 \choice nessuna delle precedenti
\end{oneparchoices}
\vspace{.5cm}

% ---------------------------------- Esercizio 3

%\addpoints
%\question
%Esegui i seguenti {\em Prodotti Notevoli}:\\
%\begin{parts}
%\part[3] \((6x - 5y) \cdot (6x + 5y)\)\\
%{\footnotesize
%\begin{solution}
%	\(36x^2 - 25y^2\)
%\end{solution}
%}
%\vspace{.5cm}
%
%\part[3] \((2a - 3b)^2\)\\
%{\footnotesize
%\begin{solution}
%	\((4a^2 - 12ab + 9b^2)\)
%\end{solution}
%}
%
%\end{parts}
%\vspace{.5cm}
%
%\addpoints
%\question [5] Qual � il significato di calcolare le condizioni di esistenza - C.E. - di una frazione algebrica?
%
%\begin{choices}
%\setlength{\leftmargin}{0pt}
% \choice per poterla semplificare meglio. 
% \choice per stabilire per quali valori dell'incognita il numeratore della frazione si annulla.
% \choice per verificare la presenza di prodotti notevoli
% \CorrectChoice per stabilire per quali valori dell'incognita il denominatore della frazione si annulla.
% \choice nessuna delle precedenti.
%\end{choices}
%\vspace{.5cm}

% ---------------------------------- Esercizio 4
%
%\addpoints
%\question [7]Esegui la seguente divisione utilizzando la Regola di Ruffini, scrivendo alla fine la scomposizione ottenuta: \(A(x) = B(x) \cdot Q(x) + R\)
%\[ (2x^4 - 5x^3 + 2) : (x - 1)\]
%\fillwithlines{0.5in}
%
%{\footnotesize
%\begin{solution}
%	\(Q(x) = 2x^3 - 3x^2 - 3x - 3, \quad R = -1\).
%\end{solution}
%}
% ------------------------------------- Domanda Bonus
\vfill
\noindent
\rule[2ex]{\textwidth}{1pt}

\bonusquestion[5] {\em Esercizio facoltativo:}

Dopo aver determinato le condizioni di esistenza semplifica la seguente frazione algebrica:\\
 \[\dfrac{x^3 - 2x^2 + x}{x^3 - 3x^2 + 3x - 1}\]
%\fillwithlines{.5in}
{\footnotesize
\begin{solution}
	\(C.E.: x \neq 1\), \quad \(\dfrac{x}{x - 1}\)
\end{solution}
}

\end{questions}
%\vfill
%\noindent
%\rule[2ex]{\textwidth}{1pt}
\pagebreak
\begin{center}
{\bf Tabella dei punteggi}
\vspace{10pt}

\combinedgradetable[h][questions]
\end{center}
\vspace{4pt}
\footnotesize La sufficienza � fissata a 18 punti, ma potr� subire delle modifiche in fase di correzione, al fine di garantire la validit� della prova anche nel caso in cui si riscontrino prestazioni della classe sensibilmente lontane dalla media-classe stimata.

\end{document}
