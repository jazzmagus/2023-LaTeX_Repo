\documentclass[12pt]{exam}
\usepackage{amsmath}
\usepackage[italian]{babel}
\addpoints
\begin{document}
\begin{center}
\textbf{Il questionario è formato
da \numquestions\ domande,
per un totale di \numpoints\
punti.}
\end{center}
\vspace{6pt}
	\begin{questions}
		\question
			Cos’è l’insieme vuoto?
		\question
			Dato un insieme avente N oggetti, qual è
			la cardinalità dell’insieme delle parti?
		\question
			Calcola $\displaystyle\int_0^1 x^2 \, dx$.
\end{questions}

\begin{questions}
	\question
		Cos’è l’insieme vuoto?
	\question
		Sia dato un insieme con N oggetti.
	\begin{parts}
		\part
			Definire cosa si intende con insieme
			delle parti.
		\part
			Determinare la cardinalità dell’insieme
			delle parti.
		\end{parts}
	\question
	\begin{parts}
		\part
			Calcola $\displaystyle\int_0^1 x^2 \, dx$.
		\part
			Calcola $\displaystyle\int_0^1 (x^2+1) \, dx$.
	\end{parts}
\end{questions}
\newpage
\pointpoints{punto}{punti}
\begin{questions}
\question[1]
Cos’è l’insieme vuoto?
\question[4]
Sia dato un insieme con N oggetti.
\begin{parts}
\part
Definire cosa si intende con insieme
delle parti.
\part
Determinare la cardinalità dell’insieme delle parti.
\end{parts}
\question
\begin{parts}
\part[15]
Calcola
$\displaystyle\int_0^1 x^2 \, dx$.
\part[15]
Calcola
$\displaystyle\int_0^1 (x^2+1) \, dx$.
\end{parts}
\end{questions}


\bracketedpoints
\begin{questions}
\question[1]
Cos’è l’insieme vuoto?
\end{questions}

\pointpoints{punto}{punti}
\begin{questions}
\question[25]
Cos’è l’insieme vuoto?
\end{questions}
La domanda numero \ref{dom:a} è
facilissima. La domanda
\ref{dom:b} è facile.
La domanda \ref{dom:c} è formata
da due parti (parte \ref{part:a}
e parte \ref{part:b}).
\begin{questions}
\question[1]
\label{dom:a}
Cos’è l’insieme vuoto?
\question[4]
\label{dom:b}
Sia dato un insieme con N oggetti.
\begin{parts}
\part
Definire cosa si intende con
insieme delle parti.
\part
Determinare la cardinalità dell’insieme delle parti.
\end{parts}
\question
\label{dom:c}
\begin{parts}
\part[15]
\label{part:a}
Calcola
$\displaystyle\int_0^1 x^2\, dx$.
\part[15]
\label{part:b}
Calcola
$\displaystyle\int_0^1 (x^2+1)\, dx$.
\end{parts}
\end{questions}

\begin{questions}
\uplevel{Per rispondere alle domande
\ref{dom:c} si utilizzino i teoremi
sugli integrali.}
\question[1]
\label{dom:a}
Cos’è l’insieme vuoto?
\question[4]
\label{dom:b}
Sia dato un insieme con N oggetti.
\begin{parts}
\uplevel{Le due domande seguenti
richiedono di fornire due definizioni}
\part
Definire cosa si intende con
insieme delle parti.
\part
Determinare la cardinalità
dell’insieme delle parti.
\end{parts}
\question
\label{dom:c}
\begin{parts}
\part[15]
Calcola$\displaystyle\int_0^1 x^2\, dx$.
\part[15]
Calcola
$\displaystyle\int_0^1 (x^2+1)\, dx$.
\end{parts}
\end{questions}
\fullwidth{Alla fine della prova, prima
di riconsegnare, verificare nuovamente
le risposte.}


\pagebreak
\begin{questions}
\fullwidth{\large\bf Parte generale}
\question Domanda uno...
\question Domanda due...
\fullwidth{\large\bf Parte specifica}
\question Domanda tre...
\question Domanda quattro...
\end{questions}


\begin{questions}
\question
Dato l’insieme formato dai numeri
dispari divisori di 30, quale dei
seguenti numeri naturali è
un suo elemento?
\begin{choices}
\choice
10
\choice
7
\choice
5
\choice
9
\choice
25
\end{choices}
\end{questions}

\begin{questions}
\question
Dato l’insieme formato dai numeri
dispari divisori di 30, quale dei
seguenti numeri naturali è
un suo elemento?
\begin{oneparchoices}
\choice
10
\choice
7
\choice
5
\choice
9
\choice
25
\end{oneparchoices}
\end{questions}

...
%\noprintanswers
\printanswers
...
\begin{questions}
\question
Cosa si intende per intersezione
di due insiemi?
\begin{solution}[3cm]
Si intende quell’insieme
formato dai soli elementi
comuni ai due insiemi
dati.
\end{solution}
...
\end{questions}
% tabella dei punteggi
\vspace{2cm}
\begin{center}
\gradetable[h][questions]
\end{center}

\end{document}