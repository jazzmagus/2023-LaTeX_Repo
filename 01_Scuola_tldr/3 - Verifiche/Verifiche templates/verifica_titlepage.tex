\documentclass[10pt,a4paper]{exam}
\usepackage[utf8]{inputenc}
\usepackage[italian]{babel}
\usepackage[T1]{fontenc}
\usepackage{amsmath}
\usepackage{amsfonts}
\usepackage{amssymb}
\usepackage{graphicx}
\usepackage{lmodern}
\usepackage{physics}
\usepackage[left=2.5cm,right=2.5cm,top=3cm,bottom=3cm]{geometry}
\usepackage{siunitx}
%\usepackage{fancyhdr}
\usepackage{enumerate}
\usepackage{mhchem}
\usepackage{mathtools}
\usepackage{graphicx}
\usepackage{setspace}
\usepackage{float}
\usepackage{xcolor}
\usepackage{mdframed}
\usepackage{changepage}
\usepackage{csquotes}

%\pagestyle{fancy}
%\fancyhead{}
%\fancyfoot{}
%\fancyhead[L]{\small {\MakeUppercase{verifica di matematica}}}
%\fancyhead[C]{\small {\MakeUppercase {Cognome e Nome:}}}
%\fancyfoot[L]{\em Liceo Scientifico "J. da Ponte"}
%\fancyfoot[C]{A.S. 2020/'21}
%\fancyfoot[R]{pag. \thepage}
%\renewcommand{\headrulewidth}{0.5pt}
%\renewcommand{\footrulewidth}{0.3pt}

%\sisetup{locale=DE}
%\sisetup{per-mode = symbol-or-fraction}
%\DeclareSIUnit\year{a}
%\DeclareSIUnit\clight{c}
%\mdfdefinestyle{exercise}{
%	backgroundcolor=black!8,roundcorner=8pt,hidealllines=true,nobreak
%}

\begin{document}
\begin{titlepage}

\begin{center}
	\Large{Liceo Scientifico "B. Brocchi" - Bassano del Grappa}\\[6pt]
	\large{- CLASSE 1 C -}\\
	\vspace{4cm}
	\line(1,0){400}\\[3.5mm]
	{\huge{\textbf{VERIFICA DI MATEMATICA}}}\\[3mm]
	\textit{\today}\\[1mm]	
	\line(1,0){400}\\
	\vspace{5cm}
\end{center}

\begin{adjustwidth}{1.5cm}{1.5cm}
\textbf{Avvertenze:}\\[6pt]
	\begin{itemize}
		\item 
			Si avere una funzione $f$ di variabile reale a valori reali definita nell'intervallo $[a, b]$, come nell'immagine. Supponiamo che essa sia continua e che in ogni punto del suo grafico - esclusi $(a, f(a))$ $(b, f(b))-$ sia ben definita la retta tangente, quest'ultima non parallela all'asse delle ordinate (supponiamo cioè che la funzione $f$ sia derivabile in $] a, b[$\\
			\item 
			Supponiamo che essa sia continua e che in ogni punto del suo grafico - esclusi $(a, f(a))$ $(b, f(b))-$ sia ben definita la retta tangente, quest'ultima non parallela\\
			\item 
			Supponiamo che essa sia continua e che in ogni punto del suo grafico - esclusi $(a, f(a))$ $(b, f(b))-$ sia ben definita la retta tangente, quest'ultima non parallela\\
		\end{itemize}
\end{adjustwidth}

\vfill
	\begin{center}
	\vspace{18pt}
	{\Large{\tt COGNOME e NOME:}}\quad \underline{\hspace{2.8in}}\\
	\end{center}
\end{titlepage}


\end{document}