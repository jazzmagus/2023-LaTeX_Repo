\documentclass[10pt,a4paper]{exam}
\usepackage[utf8]{inputenc}
\usepackage[italian]{babel}
\usepackage[T1]{fontenc}
\usepackage{amsmath}
\usepackage{amsfonts}
\usepackage{amssymb}
\usepackage{graphicx}
\usepackage{lmodern}
\usepackage{physics}
\usepackage[left=2cm,right=2cm,top=2cm,bottom=2cm]{geometry}
\usepackage{siunitx}
\usepackage{fancyhdr}
\usepackage{enumerate}
\usepackage{mhchem}
\usepackage{mathtools}
\usepackage{graphicx}
\usepackage{setspace}
\usepackage{float}
\usepackage{xcolor}
\usepackage{mdframed}
\usepackage{changepage}
\usepackage{csquotes}

\pagestyle{fancy}
\fancyhead{}
\fancyfoot{}
\fancyhead[L]{\small {\MakeUppercase{verifica di matematica}}}
\fancyhead[C]{\small {\MakeUppercase {Cognome e Nome:}}}
\fancyfoot[L]{\em Liceo Scientifico "J. da Ponte"}
\fancyfoot[C]{A.S. 2020/'21}
\fancyfoot[R]{pag. \thepage}

\renewcommand{\headrulewidth}{0.5pt}
\renewcommand{\footrulewidth}{0.3pt}

\sisetup{locale=DE}
\sisetup{per-mode = symbol-or-fraction}
\DeclareSIUnit\year{a}
\DeclareSIUnit\clight{c}
\mdfdefinestyle{exercise}{
	backgroundcolor=black!8,roundcorner=8pt,hidealllines=true,nobreak
}
\begin{document}
\begin{titlepage}

\begin{center}
	\Large{Liceo Scientifico "B. Brocchi" - Bassano del Grappa}\\[6pt]
	\large{- Sottotitolo -}\\
	\vspace{4cm}
	\line(1,0){400}\\[3.5mm]
	{\huge{\textbf{VERIFICA DI MATEMATICA}}}\\[3mm]
	\textit{\today}\\[1mm]	
	\line(1,0){400}\\
	\vspace{5cm}
\end{center}

\begin{adjustwidth}{1.5cm}{1.5cm}
\textbf{Avvertenze:}\\[6pt]
	\begin{itemize}
		\item 
			Si avere una funzione $f$ di variabile reale a valori reali definita nell'intervallo $[a, b]$, come nell'immagine. Supponiamo che essa sia continua e che in ogni punto del suo grafico - esclusi $(a, f(a))$ $(b, f(b))-$ sia ben definita la retta tangente, quest'ultima non parallela all'asse delle ordinate (supponiamo cioè che la funzione $f$ sia derivabile in $] a, b[$\\
			\item 
			Supponiamo che essa sia continua e che in ogni punto del suo grafico - esclusi $(a, f(a))$ $(b, f(b))-$ sia ben definita la retta tangente, quest'ultima non parallela\\
			\item 
			Supponiamo che essa sia continua e che in ogni punto del suo grafico - esclusi $(a, f(a))$ $(b, f(b))-$ sia ben definita la retta tangente, quest'ultima non parallela\\
		\end{itemize}
\end{adjustwidth}

\vfill
	\begin{center}
	\vspace{18pt}
	{\Large{\tt COGNOME e NOME:}}\quad \underline{\hspace{2.8in}}\\
	\end{center}
\end{titlepage}

	\begin{enumerate}
    	\setcounter{enumi}{0}
        \item \textbf{Polinomi e Prodotti Notevoli:}
        \begin{mdframed}[style=exercise]
        Verwenden Sie die Gleichung $m\qty(\dv{v}{t}+\frac{v}{\tau})=-eE$ für die Driftgeschwindigkeit $v$ der Elektronen um zu zeigen, dass die Leitfähigkeit bei der Frequenz $\omega$ gleich
        \begin{align*}
			 \sigma (\omega) &= \sigma(0)\qty(\frac{1+i\omega\tau}{1+\qty(\omega\tau)^2})
		\end{align*}
        ist, wobei $\sigma (0)=ne^2\tau/m$.\\
        \textit{Hinweis:} Verwenden Sie $v=e^{-i\omega t}$
        \end{mdframed}
        \begin{align*}
			\dv{v}{t} &= -i\omega\cdot v\\
            m\qty(-i\omega\cdot v + \frac{v}{\tau}) &= -eE\\
            \Leftrightarrow v &= \frac{-eE/m}{-i\omega+\frac{1}{\tau}}=\tau\frac{-eE/m}{-i\omega\tau+1}\stackrel{\text{c.c.}}{=}\tau\frac{-eE/m}{\qty(\omega\tau)^2+1}\qty(i\omega\tau+1)\\
            j&=-nev\qquad\sigma=\frac{j}{E}\\
            \Rightarrow\sigma(\omega) &= \underbrace{\frac{-ne}{E}\tau\qty(-eE/m)}_{=ne^2\tau/m=\sigma (0)}\frac{1+i\omega\tau}{1+\qty(\omega\tau)^2}
		\end{align*}
        \item \textbf{Teorema del Resto e Teorema di Ruffini}
        \begin{mdframed}[style=exercise]
			Supponiamo di avere una funzione $f$ di variabile reale a valori reali definita nell'intervallo $[a, b]$, come nell'immagine. Supponiamo che essa sia continua e che in ogni punto del suo grafico - esclusi $(a, f(a))$ $(b, f(b))-$ sia ben definita la retta tangente, quest'ultima non parallela all'asse delle ordinate (supponiamo cioè che la funzione $f$ sia derivabile in $] a, b[$ ). Tracciamo la retta secante il grafico, passante per i punti $(a, f(a))$ e $(b, f(b))$
II teorema di Lagrange afferma che sotto le ipotesi di regolarità sopra enunciate esiste almeno un punto $c \in] a, b[,$ come nell'esempio, tale che la tangente al grafico di $f$ nel punto $(c, f(c))$ abbia la stessa pendenza della retta passante per i punti $(a, f(a))$ e $(b, f(b))$?
		\end{mdframed}
        \begin{itemize}
			\item Gitteranteil
            \begin{align*}
				T\gg\Theta : \qquad C_{\mathrm{V,Gitter}} &= 3nk_B = \SI{3.5}{\joule\per\centi\metre\cubed\per\kelvin}\\
                T\ll\Theta:\qquad C_{\mathrm{V,Gitter}} &= 3nk_B\frac{4\pi^4}{5}\qty(\frac{T}{\Theta})^3=\SI{3.5}{\joule\per\centi\metre\cubed\per\kelvin}\cdot\num{77.93}\qty(\frac{T}{\SI{343}{\kelvin}})^3
            \end{align*}
            \item elektronischer Anteil
            \begin{align*}
				C_{\mathrm{V,e}} &= \frac{\pi^2}{2}k_Bn\frac{T}{T_{\mathrm{F,Cu}}}=\SI{5.75}{\joule\per\centi\metre\cubed\per\kelvin}\cdot\qty(\frac{T}{\SI{8.16e4}{\kelvin}})
			\end{align*}
		\end{itemize}
        \begin{align*}
			\Rightarrow C_V(T\gg \Theta) &= \SI{3.5}{\joule\per\centi\metre\cubed\per\kelvin} + \SI{5.75}{\joule\per\centi\metre\cubed\per\kelvin}\cdot\qty(\frac{T}{\SI{8.16e4}{\kelvin}})\\
            \Rightarrow C_V(T\ll\Theta) &= \SI{5.75}{\joule\per\centi\metre\cubed\per\kelvin}\cdot\qty(\frac{T}{\SI{8.16e4}{\kelvin}}) + \SI{5.75}{\joule\per\centi\metre\cubed\per\kelvin}\cdot\qty(\frac{T}{\SI{8.16e4}{\kelvin}})
        \end{align*}
        \begin{align*}
            &T\gg\Theta\qquad \SI{3.5}{\joule\per\centi\metre\cubed\per\kelvin}\stackrel{!}{=}\SI{5.75}{\joule\per\centi\metre\cubed\per\kelvin}\cdot\qty(\frac{T}{\SI{8.16e4}{\kelvin}})\\
            &\Rightarrow T = \SI{5.0e4}{\kelvin}\\
            &T\ll\Theta\qquad \SI{3.5}{\joule\per\centi\metre\cubed\per\kelvin}\cdot\num{77.93}\qty(\frac{T}{\SI{343}{\kelvin}})^3\stackrel{!}{=}\SI{5.75}{\joule\per\centi\metre\cubed\per\kelvin}\cdot\qty(\frac{T}{\SI{8.16e4}{\kelvin}})\\
            & \Rightarrow T^2=\SI{10.43}{\kelvin\squared} \Leftrightarrow T=\SI{3.23}{\kelvin}
		\end{align*}
%\newpage
%        \item \textbf{Photoemissionsspektren}
%        \begin{mdframed}[style=exercise]
%			Für ein eindimensionales freies Elektronengas ergibt sich nachstehende Dispersion $E(k)$ (Abb.
%1), wenn der unterste \enquote{gebundene} Zustand $\SI{9}{\electronvolt}$ unterhalb der Fermikante $E_F$ liegt. Nehmen Sie an, dass in einem Photoemissions-Experiment in senkrechter Emission bei den angegebenen Photonenenergien $h\nu$ nachstehende Spektren gemessen wurden. Bestimmen Sie graphisch den Verlauf $E_i(k)$ der (beiden) besetzten Bänder relativ zu $E_f$ (d.h. $E_i < 0$) unter der Annahme, dass die Endzustände $E_f(k)$ in ihrer Dispersion der gezeichneten Form freier Elektronen folgen.\\
%\textit{Hinweis:} $E_f(k)=E_i(k)+h\nu$, wobei $E_i$ der Bindungsenergie relativ zur Fermikante entspricht.
%		\end{mdframed}
%        \begin{tabular}{| l | r | r | r | r | r |}
%        \hline
%        	 $h\nu$ & \SI{36}{\electronvolt} & \SI{24}{\electronvolt} & \SI{18}{\electronvolt} & \SI{9}{\electronvolt} & \SI{6}{\electronvolt}\\ \hline
%			1. $E_i$ & -8 & -6 & -5 & \num{-3.5} & -3\\ \hline
%            2. $E_i$ & \num{-0.25} & \num{-0.5} & -1 & -2 & \num{-2.5}\\ \hline
%            1. $E_f$ & 28 & 18 & 13 & \num{5.5} & 3\\ \hline
%            2. $E_f$ & 35 & \num{23.5} & 17 & 7 & \num{3.5} \\ \hline
%		\end{tabular}
  \end{enumerate}
\end{document}