\documentclass[10pt]{article}
%\usepackage{blindtext}
\usepackage{fancyhdr}
\usepackage{changepage}

%\usepackage[utf8]{inputenc}
\usepackage[T1]{fontenc}
\usepackage{lmodern}
\usepackage[italian]{babel}
\usepackage{epigraph}
\usepackage{hyperref}
\hypersetup{
   colorlinks=true,
   linkcolor=blue,
   filecolor=magenta,      
	urlcolor=cyan,
}
\urlstyle{same}
\setlength{\parindent}{0em}
\setlength{\parskip}{1em}
\renewcommand{\baselinestretch}{1.5}

\usepackage{setspace}
\usepackage{quoting}
\quotingsetup{font=small}
\usepackage{fancyhdr}

\begin{document}
\pagestyle{empty}
\onehalfspacing
\emph{Bassano del Grappa,\qquad 29 settembre 2020}\\
\line(1,0){200}\\

Carissimi,

Come promesso vi lascio un paio di righe per comunicarvi che, ahim�, quest'anno non sar� io il vostro prof. di matematica.

Ho fatto tutto ci� che era legalmente possibile per tornare ad Asiago, ma sono rimasto fregato dai meccanismi poco matematici delle convocazioni.

Vi basti sapere che quest'anno insegner� {\em Maths by Night}: sar� infatti in un Istituto Professionale ad insegnare matematica ai Corsi Serali; pensate, avr� ben 2 (DUE) classi, entrambe con lo stesso programma: Polinomi!!!\\
La vivr� come una sorta di punizione, di Karma, per essermi permesso di scherzare con Paolino Ruffini: � stramaledettamente permaloso!\\ Ora dovr� sorbirmi le sue "fattorizzazioni capricciose" tutto l'anno!

Vi confesso che non l'ho presa benissimo; mi ci vorr� un po' per ingoiare questo brutto rospo - un {\em Incilius Alvarius} almeno \dots -.

Non mi rimane quindi che augurarvi uno splendido anno scolastico e ringraziarvi per quello appena trascorso insieme.\\
Grazie per la fiducia.\\
Grazie per il rispetto con cui mi avete trattato.\\
Grazie per i vostri sorrisi e la vostra pazienza.\\
Grazie infinite per avermi concesso l'onore di poter lavorare con le vostre meravigliose menti.

Vorrei poter ricambiare in maniera altrettanto generosa, tipo spedirvi con Amazon Prime uno scatolone di Felicit�: in realt� l'ho gi� messa nel carrello, ma al momento non � disponibile, pertanto vi riporto qui sotto la sua {\em formula matematica}, che esiste veramente, cio� vera-veramente c'� stato qualcuno (pi� di uno in realt� \dots) che ha sprecato tempo ed energie per qualcosa che non serve assolutamente a nulla.
{\em People are so weird!}. (traduzione: � pieno di scemi l� fuori!).\\

Vi servir� a ricordare che le emozioni presentano un grado di complessit� elevatissimo, a causa del numero di variabili coinvolte, spesso incommensurabili (non misurabili).\\
E' cos� fratell�!

	{\em $\textbf{Happiness(t)}=w_{0}+w_{1} \cdot\displaystyle{\sum_{j=1}^{t}} \gamma^{t-j} C R_{j}+w_{2} \cdot\sum_{j=1}^{t} \gamma^{t-j} E V_{j}+w_{3} \cdot\sum_{j=1}^{t} \gamma^{t-j} R P E_{j}$}
	
	\begin{quoting}{\textbf{\small Formula della Felicit�:} {\em La felicit� � funzione del tempo $t$; $w_0$, $w_1$, $w_2$ e $w_3$ sono costanti che indicano l'influenza dei diversi tipi di eventi; $\gamma$ � un "forgetting factor" (fattore dimenticando) che rende gli eventi degli studi pi� recenti pi� influenti rispetto a quelli precedenti; $CRj$ � la gratificazione ottenuta dalla scelta su un processo $j$; $EVj$ � la valutazione del rischio su di un processo $j$; $RPEj$ rappresenta la differenza tra la ricompensa desiderata e quella effettivamente ottenuta dal processo $j$.\\
Questo modello spiega le fluttuazioni della felicit� momento dopo momento, mettendo in luce quanto un evento recente sia valutato da ogni individuo pi� importante di uno precedente.\\
Inoltre, fattore veramente rilevante � l'aspettativa: solitamente tendiamo a caricarci di aspettative e nel momento in cui una circostanza si verifica ci ritroviamo delusi in quanto l'avvenimento non � all'altezza di ci� che avevamo immaginato.}}
\end{quoting}

%\vspace{10pt}
Vi abbraccio tutti, ma proprio tutti, uno ad uno, cos� non si creano assembramenti\dots\\

Con disordinato affetto,\\

D!ego

\epigraph{Vien dietro a me, e lascia dir le genti:
sta come torre ferma, che non crolla
gi� mai la cima per soffiar di venti}{\textit {Dante Alighieri, \\ Divina Commedia - Canto V}}
\end{document}