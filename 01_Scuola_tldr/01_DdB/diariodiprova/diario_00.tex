\documentclass[10pt, twoside, notitlepage, notoc, justified]{tufte-handout}
\usepackage{lipsum}
\usepackage[utf8]{inputenc}
\usepackage[italian]{babel}
\usepackage{multicol}
\usepackage{caption}
\captionsetup{justification   = raggedright,
              singlelinecheck = false}
\setlength{\columnsep}{1cm}
%\usepackage[margin=1in]{geometry}
\usepackage{amsfonts, amsmath, amssymb}
\usepackage[none]{hyphenat}
\usepackage{fancyhdr}
\usepackage{graphics}
\usepackage{float}
\usepackage[nottoc, notlot, notlof]{tocbibind}
\usepackage{pgf,tikz,pgfplots} 
%\pgfplotsset{compat=1.15}
\usepackage{mathrsfs}
\usetikzlibrary{arrows}

\pagestyle{fancy}
\fancyhead{}
\fancyfoot{}
\fancyhead[L]{\small \MakeUppercase{lezioni di matematica}}
\fancyhead[R]{\small \emph{prof. Diego Fantinelli}}
\fancyfoot[C]{\thepage}
\renewcommand{\headrulewidth}{0.15pt}
\renewcommand{\footrulewidth}{0.1pt}

\parindent 0ex
%\setlength{\parindent}{2em}
%\setlength{\parskip}{1em}
%\renewcommand{\baselinestretch}{1.5}



%%%%%%%%%%%%%%%%%%%%%%%%%%%%%%%%%%%%%%%%%%%% title page

\begin{document}
\noindent
{\LARGE{\textbf{Diario di Classe - Prima C}}}\\[3mm]
begin{titlepage}
begin{center}
vspace{1cm}
\Large{\Large{\textit {Matematica - Lezioni per il Liceo}}}\\[4mm]
\large{\textbf{Anno Scolastico 2023-24}}\\
\vfill
\line(1,0){400}\\[.5mm]
{\Huge{\textbf{Diario di Bordo 2021}}}
\Large{\textbf{- Sottotitolo:  -}}\\[1mm]
\\
\line(1,0){400}\\
\tableofcontents
%vfill
%scriptsize By Student Name}\\
%scriptsize Candidate \#} \\
%scriptsize \today} \\
end{center}

end{titlepage}

%%%%%%%%%%%%%%%%%%%%%%%%%%%%%%%%%%%%%%%%%%%% introduzione

\thispagestyle{empty}
\clearpage

\setcounter{page}{1}


\newenvironment{loggentry}[2]% date, heading
{\noindent{\ttfamily {#2}\marginnote{#1}}\\}{\vspace{0.cm}}

\newpage
%------------------------------------- OTTOBRE 2020

\section{Ottobre - 2021}
	\subsection{Classe 1 C}
\begin{loggentry}{11/10/2021}{Teoria degli Insiemi}
Mauris ut leo. Cras viverra metus rhoncus sem. et  vestibulum ur na fringilla ultrices. 
Phasellus eu tellus sit amet tortor gravida placerat. Integer sapien est, iaculis in, pretium quis, viverra ac, nunc. Praesent eget sem vel leo ultrices bibendum.

\subsection{Classe 1 B}
\end{loggentry}

\begin{loggentry}{22/10/2021}{Titolo della LEZIONE}
Mauris ut leo. Cras viverra metus rhoncus sem. et  vestibulum ur na fringilla ultrices. Phasellus eu tellus sit amet tortor gravida placerat. Integer sapien est, iaculis in, pretium quis, viverra ac, nunc. Praesent eget sem vel leo ultrices bibendum.\\

\end{loggentry}

\begin{loggentry}{31/10/2021}{Le Derivate: definizione e Rappresentazione Grafica}
%\vspace{5pt}
Mauris ut leo. Cras viverra metus rhoncus sem. 
\marginnote{ces bibendum. Mauris ut leo. Cras viverra metus rhoncus sem. Nulla et lectus vestibulum ur}Nulla et ur na fringilla ultrices. Phasellus eu tellus sit amet tortor gravida placerat. Integer sapien est, iaculis in, pretium quis, viverra ac, nunc. Praesent eget sem vel leo ultrices bibendum.\\
\subsection{attenzione} 
Mauris ut leo. Cras viverra metus rhoncus sem.  et vestibulum ur na fringilla ultrices. Phasellus eu tellus sit amet tortor gravida placerat. Integer sapien est, iaculis in, pretium quis, viverra ac, nunc. Praesent eget sem vel leo ultrices bibendum.\\
\end{loggentry}

\begin{loggentry}{31/10/2021}{Sistemi Lineari}
\begin{itemize}
	\item Mauris ut leo. Cras viverra metus rhoncus sem. Nulla et  ur na fringilla ultrices. Phasellus eu tellus sit amet tortor gravida placerat. 
	\item Mauris ut leo. Cras viverra metus rhoncus sem. Nulla et  ur na fringilla ultrices. Phasellus eu tellus sit amet tortor gravida placerat. 
	\item Mauris ut leo. Cras viverra metus rhoncus sem.
	\item Nulla et  ur na fringilla ultrices. Phasellus eu tellus sit amet tortor gravida placerat. 
\end{itemize}

\subsection{comportamento}
Integer sapien est, iaculis in, pretium quis, viverra ac, nunc. Praesent eget sem vel leo ultrices bibendum. Mauris ut leo. Cras viverra metus rhoncus sem. 

\subsection{attenzione}
\newthought{In his later books}, Tufte starts \dots\\Nulla et lectus vestibulum ur na fringilla ultrices. ellus eu tellus sit amet tortor gravida placerat. Integer sapien est, iaculis in, pretium quis, viverra ac, nunc. Praesent eget sem vel leo ultrices bibendum.
\end{loggentry}
\vspace{6pt}\\
\line(1,0){310}\\
%------------------------------------- NOVEMBRE 2020
\section{Novembre - 2021}

 \begin{loggentry}{15/11/2011}{Integrali Indefiniti}
 	\subsection{Classe 1 C}
 	 Integer sapien est, iaculis in, \(a x^2 - 3 x + 2 = 0\)pretium quis, viverra ac, nunc. Praesent eget sem vel leo ultrices bibendum.
 	 \subsection{Classe 4 C}
 	 \newthought{Segnalazioni} Il di Michele Micheletto impertinente
 	 \subsection{Classe 3 C}
 	 Nulla et ur na fringilla ultrices. Phasellus eu tellus sit amet tortor gravida placerat.
 	 \begin{enumerate}
 	 	\item equazioni lineari
 	 	\item eq. elementari \(a x^2 - 3 x + 2 = 0\)
 	 	\item Eq. riconducibili ad elementari
 	 	\item esercizi
 	 \end{enumerate}
 \end{loggentry}
  
\begin{loggentry}{2010-Dez-31}{Water of Life}
\lipsum[9] 
\end{loggentry}

 \begin{loggentry}{2012-Aug-24}{Sunrise}
\lipsum[6]
\end{loggentry}

\end{document}
