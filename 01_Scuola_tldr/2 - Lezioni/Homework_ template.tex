\documentclass[10pt, A4paper, landscape]{exam}

%%%%%%%%%%%%%%%%%%%%%%%%%%%%%%%%%%%%%%%%%%%% packages
%\usepackage[T1]{fontenc}
\usepackage[latin1]{inputenc}
\usepackage[italian]{babel}
\usepackage{multicol}
\usepackage{amsthm}
\usepackage{caption}
\captionsetup{justification=raggedright, singlelinecheck = false}
\setlength{\columnsep}{1cm}

\usepackage[margin=1in]{geometry}
%\usepackage{amsfonts, amsmath, amssymb}
%\usepackage[none]{hyphenat}
%\usepackage{fancyhdr}
%\usepackage{graphics}
\usepackage{lipsum}
%\usepackage{float}
%\usepackage[nottoc, notlot, notlof]{tocbibind}
%\usepackage{pgf, tikz, pgfplots} 
%\pgfplotsset{compat=1.15}
%\usepackage{mathrsfs}
%\usetikzlibrary{arrows, calc}

\newtheorem{theorem}{Teorema}[section]
\newtheorem{corollary}{Corollario}[theorem]
\newtheorem{lemma}[theorem]{Lemma}
\theoremstyle{definition}
\newtheorem{definition}{Definizione}[section]
\theoremstyle{remark}
\newtheorem*{remark}{Osservazioni}

\begin{document}
	
	
%\begin{titlepage}
\begin{center}
%\vspace*{1cm}

\Large{\textbf{ISS "G. A. Remondini}}\\
\Large{\textbf{Dipartimento di Matematica}}\\
\large{\em{- prof. diego fantinelli -}}\\
\vspace{0,5in} 
\line(1,0){400}\\[.5mm]
\huge{\textbf{Esercizi di Matematica}}\\
%\Large{\textbf{- Sottotitolo:  -}}\\[1mm]
\line(1,0){400}\\[.5mm]
%\line(1,0){400}\\
\vspace{0.5in} 
%{\scriptsize By Student Name}\\
%{\scriptsize Candidate \#} \\
%{\scriptsize \today} \\

\end{center}
%\end{titlepage}
%-------------------------title page

%\tableofcontents
%\thispagestyle{empty}
%\clearpage

\setcounter{page}{1}
\begin{multicols}{2}
\section*{Introduzione}
Il presente documento contiene le principali soluzioni per la formattazione di un testo scientifico, con particolare riferimento ai testi matematici, comprensivi di \emph{formule}, e caratteri speciali

\subsection*{Come recuperare l'autostima}
Sed fringilla, neque sit amet maximus luctus, neque eros fermentum ipsum, nec hendrerit leo urna id urna. Pellentesque vel odio lobortis diam placerat porttitor non auctor leo.\\

\end{multicols}
\end{document}