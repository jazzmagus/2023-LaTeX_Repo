% -------------------------------------------------------------
% Template HOMEWORK
% -------------------------------------------------------------
% 2020 by d!egofantinelli at jazzmagus@gmail.com
% -------------------------------------------------------------

% ---------------------------------- Preambolo
\documentclass[12pt, a4paper, landscape]{exam}
\usepackage[T1]{fontenc}
\usepackage{mdframed}
%\usepackage{nicefrac}
%\usepackage[applemac]{inputenc}
%\usepackage[utf8]{inputenc}
\usepackage[italian]{babel}
\usepackage[margin=1in]{geometry}
\usepackage{amsfonts, amsthm, amsmath, amssymb}
\usepackage{multicol}
\usepackage{mathrsfs}
\usepackage[none]{hyphenat}
\usepackage{bbm}
\usepackage{graphicx}
\usepackage{tikz}
%\usepackage[dvipsnames]
%\usepackage{upquote}
\usepackage{caption}
\usepackage{float}

%\printanswers

% ---------------------------------- General Math Setup
\newcommand{\numberset}{\mathbb}
\newcommand{\N}{\numberset{N}}
\newcommand{\Z}{\numberset{Z}}
\newcommand{\Q}{\numberset{Q}}
\newcommand{\R}{\numberset{R}}

\newcommand{\ChoiceLabel}[1]{\hspace{-1.6em}\makebox[1.6em][l]{\textbf{#1.}}\ignorespaces}

\newcommand{\WrongChoice}[1]{\choice \ChoiceLabel{#1}}
\newcommand{\RightChoice}[1]{\correctchoice \ChoiceLabel{#1}}
\newcommand{\Item}[1]{\hspace{-17pt}\makebox[17pt][l]{\textbf{#1.}}\ignorespaces}

\renewcommand{\solutiontitle}{\noindent\textbf{Soluzione:}\par\noindent}

\renewcommand{\questionshook}{%
    \setlength{\leftmargin}{0pt}%
}
\renewcommand{\choiceshook}{%
    \setlength{\leftmargin}{20pt}%
}

% ---------------------------------- Intestazione
\newcommand{\class}{\LARGE {Esercizi proposti}}
%%\newcommand{\term}{I Quadrimestre}
%\newcommand{\examnum}{Verifica numero: 1}
%\newcommand{\examdate}{11 dicembre 2020}
%\newcommand{\timelimit}{40 minuti}
\CorrectChoiceEmphasis{\color{red}}
\SolutionEmphasis{\color{red} \footnotesize}
\renewcommand{\solutiontitle}{\noindent\textbf{Soluzione:}\par\noindent}


% ---------------------------------- Intestazione
\pagestyle{headandfoot}
\footrule
\headrule
\lhead{MATEMATICA}
\rhead{Classe 4\string^QA}
\chead{IIS "G. A. Remondini" - Bassano del Grappa (VI)}
\rfoot{\em{\color{orange} scadenza consegna: mercoled� 24 marzo 2021, ore 19:30}}
\lfoot{pag. \thepage\ of \numpages}

% ---------------------------------- Punteggi
%\pointpoints{punto}{\em punti}
%\pointformat{[{\footnotesize \thepoints}]}
%\bonuspointpoints{punto bonus}{\em punti bonus}
%\bonuspointformat{[{\footnotesize \thepoints}]}
%\pointsinrightmargin
%\setlength{\rightpointsmargin}{.2cm}
%\chqword{Esercizio}
%\chpword{Punti}
%\chbpword{Punti Bonus}
%\chsword{Punteggio}
%\chtword{Totale}

\begin{document}

% ---------------------------------- Title Page
%\vspace{2cm}
\noindent
{\huge{\bf \class}}\\[20pt]
\vspace{10pt}
{\Large{Disequazioni Razionali Frazionarie di secondo grado}}:\\
\hrule

%{\large {\(y = f(x) = \dfrac {x - 1}{x^2 - x - 6}\)}}\\


% ====================== VERIFICA 

\begin{questions}

\begin{multicols}{2}
% ------------------------------------------- Esercizio #1
%\addpoints
\question
Studiare il Segno della seguente disequazione:\\

{\large {\[\dfrac {x - 1}{x^2 - x - 6} \le 0\]}}\\

\fillwithdottedlines{1in}

%\vspace{0.3cm}

% ------------------------------------------- Esercizio #2
\question
Studiare il Segno della seguente disequazione, dopo averla opportunamente riportata nella forma \(\dfrac{N(x)}{D(x)} \ge 0\), oppure\(\dfrac{N(x)}{D(x)} \le 0\)\\

{\large {\[\dfrac {1}{x^2 - 1} + \dfrac {1}{2x - 2} > \dfrac {1}{6} + \dfrac {1}{2x - 2}  \]}}\\

\fillwithdottedlines{1in}

\begin{solution}
La differenza sta nel fatto che i Numero Razionali - che appartiene cio� all'Insieme \( \Q \)	- per definizione hanno un SEGNO, in quanto composti da numeri Interi Relativi \( \Z \); si pu� anche dire, pi� precisamente, che una \emph{frazione} � il rapporto tra due numeri Naturali \( \N \), mentre un numero \emph{razionale} � il rapporto tra due numeri Interi Relativi dell'insieme \( \Z \).
\end{solution}

\vspace{.8cm}


\end{multicols}
\end{questions}

\end{document}
