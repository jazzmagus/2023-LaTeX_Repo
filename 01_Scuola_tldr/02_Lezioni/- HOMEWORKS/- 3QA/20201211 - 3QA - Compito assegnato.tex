% -------------------------------------------------------------
% Template HOMEWORK
% -------------------------------------------------------------
% 2020 by d!egofantinelli at jazzmagus@gmail.com
% -------------------------------------------------------------

% ---------------------------------- Preambolo
\documentclass[12pt, a4paper, landscape]{exam}
\usepackage[T1]{fontenc}
\usepackage{mdframed}
%\usepackage{nicefrac}
%\usepackage[applemac]{inputenc}
%\usepackage[utf8]{inputenc}
\usepackage[italian]{babel}
\usepackage[margin=1in]{geometry}
\usepackage{amsfonts, amsthm, amsmath, amssymb}
\usepackage{multicol}
\usepackage{mathrsfs}
\usepackage[none]{hyphenat}
\usepackage{bbm}
\usepackage{graphicx}
\usepackage{tikz}
%\usepackage[dvipsnames]
%\usepackage{upquote}
\usepackage{caption}
\usepackage{float}

%\printanswers


% ---------------------------------- Checkbox Answer Setup
%\checkboxchar{$\square$}
%\checkedchar{$\boxtimes$}
%\CorrectChoiceEmphasis{}  % solo per non avere bold nelle risposte corrette
%

%% Create a Matching question format
%\newcommand*\Matching[1]{
%\ifprintanswers
%    \textbf{#1}
%\else
%    \rule{1in}{0.5pt}
%\fi
%}
%\newlength\matchlena
%\newlength\matchlenb
%\settowidth\matchlena{\rule{1.1in}{0pt}}
%\newcommand\MatchQuestion[2]{%
%    \setlength\matchlenb{\linewidth}
%    \addtolength\matchlenb{-\matchlena}
%    \parbox[t]{\matchlena}{\Matching{#1}}\enspace\parbox[t]{\matchlenb}{#2}}
%
%\renewcommand{\thequestion}{\arabic{question}}
%\renewcommand{\thepartno}{\alph{partno}}
%\renewcommand{\thechoice}{\alph{choice}}
%
%\renewcommand{\checkboxeshook}{
%  \setlength{\labelsep}{2.4em}
%  \setlength{\leftmargin}{4em}
%}

% ---------------------------------- General Math Setup
\newcommand{\numberset}{\mathbb}
\newcommand{\N}{\numberset{N}}
\newcommand{\Z}{\numberset{Z}}
\newcommand{\Q}{\numberset{Q}}
\newcommand{\R}{\numberset{R}}

\newcommand{\ChoiceLabel}[1]{\hspace{-1.6em}\makebox[1.6em][l]{\textbf{#1.}}\ignorespaces}

\newcommand{\WrongChoice}[1]{\choice \ChoiceLabel{#1}}
\newcommand{\RightChoice}[1]{\correctchoice \ChoiceLabel{#1}}
\newcommand{\Item}[1]{\hspace{-17pt}\makebox[17pt][l]{\textbf{#1.}}\ignorespaces}

\renewcommand{\solutiontitle}{\noindent\textbf{Soluzione:}\par\noindent}

\renewcommand{\questionshook}{%
    \setlength{\leftmargin}{0pt}%
}
\renewcommand{\choiceshook}{%
    \setlength{\leftmargin}{20pt}%
}

% ---------------------------------- Intestazione
\newcommand{\class}{\LARGE {Esercizi assegnati}}
%%\newcommand{\term}{I Quadrimestre}
%\newcommand{\examnum}{Verifica numero: 1}
%\newcommand{\examdate}{11 dicembre 2020}
%\newcommand{\timelimit}{40 minuti}
\CorrectChoiceEmphasis{\color{red}}
\SolutionEmphasis{\color{red} \footnotesize}
\renewcommand{\solutiontitle}{\noindent\textbf{Soluzione:}\par\noindent}


% ---------------------------------- Intestazione
\pagestyle{headandfoot}
\footrule
\headrule
\lhead{MATEMATICA}
\rhead{Classe 3\string^QA}
\chead{IIS "G. A. Remondini" - Bassano del Grappa (VI)}
\rfoot{{\emph {\color{red} {\bfseries scadenza consegna:} venerd� 11 dicembre 2020, ore 22:00}}}
\lfoot{pag. \thepage\ of \numpages}

% ---------------------------------- Punteggi
%\pointpoints{punto}{\em punti}
%\pointformat{[{\footnotesize \thepoints}]}
%\bonuspointpoints{punto bonus}{\em punti bonus}
%\bonuspointformat{[{\footnotesize \thepoints}]}
%\pointsinrightmargin
%\setlength{\rightpointsmargin}{.2cm}
%\chqword{Esercizio}
%\chpword{Punti}
%\chbpword{Punti Bonus}
%\chsword{Punteggio}
%\chtword{Totale}

\begin{document}

% ---------------------------------- Title Page
%\vspace{2cm}
\noindent
{\huge{\bf \class}}\\

%\vspace{3cm}
%\begin{tabular*}{\textwidth}{l @{\extracolsep{\fill}} r @{\extracolsep{6pt}} l}
%\textbf{} & \textbf{Nome e Cognome:} & \makebox[2.5in]{\hrulefill}\\
%\textbf{} &&\\
%\textbf{} & \textbf{Classe:} & \makebox[2.5in]{\Large{\bf 2 \string^ C}}\\
%\textbf{} &&\\
%\textbf{} & Tempo a disposizione: & \makebox[2.5in]{\timelimit}\\
%\textbf{} &&\\
%\textbf{} &&\\
%\textbf{} &&\\
%\textbf{} & {\em prof.:} & \makebox[2.5in]{\em Diego Fantinelli}
%\end{tabular*}\\
%
%\vspace{5cm}
% ---------------------------------- Avvertenze

%\noindent
%%\rule[2ex]{\textwidth}{0.2pt}
%\textbf{Avvertenze}:
%\begin{itemize}
%	\item La presente Verifica - che viene somministrata in modalit� DDI - contiene \numquestions \; quesiti, per un totale di \numpoints \;punti, di cui uno facoltativo di \numbonuspoints \;punti, che verr� conteggiato soltanto se verranno svolti anche tutti i precedenti.
%	\item La webcam dovr� rimanere accesa per tutto il tempo della verifica (\timelimit), salvo impossibilit� concrete di connessione; il microfono rester� spento e verr� acceso soltanto per chiarimenti e domande, che saranno consentite negli ultimi 20 min di prova.
%	\item E' vietato l'utilizzo di calcolatrici scientifiche, smartphone, tablet e altri dispositivi digitali, nonch� la consultazione di testi, appunti e siti web.
%
%\end{itemize}
%%\rule[2ex]{\textwidth}{0.2pt}
%\vfill
%\newpage
%

% =========================================== VERIFICA 

\begin{questions}

\begin{multicols}{2}
% ------------------------------------------- Esercizio #1
%\addpoints
\question
Semplifica le seguenti espressioni con le frazioni:
 
%\( (A + B) \cdot (A - B) = A^2 - B^2\)\\

\begin{parts}

%\part
%\( (2a - 3b) \cdot (2a + 3b) \);
%
%%{\footnotesize {\emph{Suggerimento:} Raccoglimento Totale}}
%
%\begin{solution}
%
%\( [- 115] \)
%\end{solution} 
%
%\vspace{.3cm}
%\part
%\( (3x^2 - 5y^3) \cdot (3x^2 + 5y^3) \);
%
%%{\footnotesize {\emph{Suggerimento:} Raccoglimento Totale}}
%
%\begin{solution} 
%\( [+ 17] \)
%\end{solution}
%
\vspace{0.3cm}
\part

\(\left\{\dfrac{3}{20} \cdot\left[\left(\dfrac{4}{9}-\dfrac{1}{3}\right): 5+\left(\dfrac{3}{7}-\dfrac{2}{5}\right): \dfrac{1}{14}+\dfrac{1}{5} \cdot \dfrac{1}{9}\right]+\dfrac{2}{15}\right\}: 2 \);

%{\footnotesize {\emph{Suggerimento:} Raccoglimento Parziale e poi Totale}}

\begin{solution} 
\begin{align*}
	\left\{\dfrac{3}{20} \cdot\left[\left(\dfrac{4}{9}-\dfrac{1}{3}\right): 5+\left(\dfrac{3}{7}-\dfrac{2}{5}\right): \dfrac{1}{14}+\dfrac{1}{5} \cdot \dfrac{1}{9}\right]+\dfrac{2}{15}\right\}: 2=\\
=\left\{\dfrac{3}{20} \cdot\left[\left(\dfrac{4-3}{9}\right): 5+\left(\dfrac{15-14}{35}\right): \dfrac{1}{14}+\dfrac{1}{45}\right]+\dfrac{2}{15}\right\}: 2 \\
=\left[\dfrac{3}{20} \cdot\left(\dfrac{1}{9}: 5+\dfrac{1}{35}: \dfrac{1}{14}+\dfrac{1}{45}\right)+\dfrac{2}{15}\right]: 2 \\
=\left[\dfrac{3}{20} \cdot\left(\dfrac{1}{9} \cdot \dfrac{1}{5}+\dfrac{1}{35} \cdot \dfrac{14}{1}+\dfrac{1}{45}\right)+\dfrac{2}{15}\right]: 2\\
=\left[\dfrac{3}{20} \cdot\left(\dfrac{1}{45}+\dfrac{1}{7 \cdot 5} \cdot \dfrac{7 \cdot 2}{1}+\dfrac{1}{45}\right)+\dfrac{2}{15}\right]: 2\\
=\left[\dfrac{3}{20} \cdot\left(\dfrac{1}{45}+\dfrac{2}{5}+\dfrac{1}{45}\right)+\dfrac{2}{15}\right]: 2\\
=\left[\dfrac{3}{20} \cdot\left(\dfrac{1+18+1}{45}\right)+\dfrac{2}{15}\right]: 2 \\
=\left(\dfrac{3}{20} \cdot \dfrac{20}{45}+\dfrac{2}{15}\right): 2 \\
=\left(\dfrac{3}{20} \cdot \dfrac{4}{9}+\dfrac{2}{15}\right): 2 \\
=\left(\dfrac{1}{5} \cdot \dfrac{1}{3}+\dfrac{2}{15}\right): 2\\
=\left(\dfrac{1}{15}+\dfrac{2}{15}\right): 2  \\
=\dfrac{3}{15}: 2 \\
=\dfrac{1}{5} \cdot \dfrac{1}{2} \\
=\dfrac{1}{10} \qed
\end{align*}

\end{solution}

\vspace{.3cm}
\part
\(
	 \left[\dfrac{13}{5}:\left(3+\dfrac{9}{10}\right)+\dfrac{7}{8}+\left(\dfrac{13}{4}-2\right) \cdot \dfrac{4}{15}-\dfrac{7}{8}\right] \cdot \dfrac{11}{3}:\left(6-\dfrac{1}{2}\right)
\);


%{\footnotesize {\emph{Suggerimento:} Raccoglimento Totale}}

\begin{solution}
\begin{align*}
	\left[\begin{array}{l}\left.\dfrac{13}{5}:\left(3+\dfrac{9}{10}\right)+\dfrac{7}{8}+\left(\dfrac{13}{4}-2\right) \cdot \dfrac{4}{15}-\dfrac{7}{8}\right] \cdot \dfrac{11}{3}:\left(6-\dfrac{1}{2}\right)= \\ =\left[\dfrac{13}{5}:\left(\dfrac{30+9}{10}\right)+\dfrac{7}{8}+\left(\dfrac{13-8}{4}\right) \cdot \dfrac{4}{15}-\dfrac{7}{8}\right] \cdot \dfrac{11}{3}:\left(\dfrac{12-1}{2}\right) \\ =\left(\dfrac{13}{5}: \dfrac{39}{10}+\dfrac{7}{8}+\dfrac{5}{4} \cdot \dfrac{4}{15}-\dfrac{7}{8}\right) \cdot \dfrac{11}{3}: \dfrac{11}{2} \\ =\left(\dfrac{13}{5}: \dfrac{39}{10}+\dfrac{7}{8}+\dfrac{1}{3}-\dfrac{7}{8}\right) \cdot \dfrac{11}{3} \cdot \dfrac{2}{11} \\ =\left(\dfrac{2}{3}+\dfrac{7}{8}+\dfrac{1}{3}-\dfrac{7}{8}\right) \cdot \dfrac{11}{3} \cdot \dfrac{2}{11} \\ =\left(\dfrac{2}{3}+\dfrac{7}{8}+\dfrac{1}{3}-\dfrac{7}{8}\right) \cdot \dfrac{2}{3} \\ =\left(\dfrac{2}{3}+\dfrac{1}{3}\right) \cdot \dfrac{2}{3} \\ =1 \cdot \dfrac{2}{3} \\ =\dfrac{2}{3}\end{array}\right. 
\end{align*} 
\end{solution}

\end{parts}

\vspace{0.3cm}

% ------------------------------------------- Esercizio #2
\question
Utilizzando le \emph{propriet� delle potenze} nell'Insieme \( \Q \), \\ semplifica le seguenti espressioni: \\

\begin{parts}

\part
\(  - \left( \dfrac{5}{6}  \right)^{-1} \); \quad  \( \left[ \left( - \dfrac{3}{5}  \right)^{-2} \right]^{2}\); \quad  \( - \left[ - \left( - \dfrac{3}{5}  \right)^{-2} \right]^{2}\); 

%{\footnotesize {\emph{Suggerimento:} Raccoglimento Totale}}

\begin{solution}

\( [+ 2] \)
\end{solution} 

\vspace{.3cm}

\part
\( \left( - \dfrac{3}{5}  \right)^{0} \); \quad  \( \left( - \dfrac{2}{3}  \right)^{-2} \); \quad  \( \left\{ \left[ \left( - \dfrac{2}{3}  \right)^{-2} \right]^{0} \right\}^{-3} \); 

%{\footnotesize {\emph{Suggerimento:} Raccoglimento Totale}}

\begin{solution}

\( [+ 2] \)
\end{solution} 

\vspace{.3cm}

\part
\( \left\{ - \left[ - \left( - 1  \right)^{-2} \right]^{3} \right\}^{-1} \); \quad  \( \left( - \dfrac{3}{2}  \right)^{-2} \cdot \left( - \dfrac{3}{2}  \right)^{-3} \); \quad  \( \left[ \left( - \dfrac{5}{4}  \right)^{-2}\right]^{-1}  \); 

%{\footnotesize {\emph{Suggerimento:} Raccoglimento Totale}}

\begin{solution} 
\( \left[ - \dfrac{1}{16} \right] \)
\end{solution}

%\vspace{0.3cm}
%\part
%\( ( x - x^2 + 1)^2 \);
%
%%{\footnotesize {\emph{Suggerimento:} Raccoglimento Parziale e poi Totale}}
%
%\begin{solution} 
%\( [- 3] \) 
%\end{solution}
%
%\vspace{.3cm}
%\part
%\(  \left( 3x^2 + \dfrac{1}{2}y^2 - \dfrac{3}{4} \right)^{2} \);
%
%%{\footnotesize {\emph{Suggerimento:} Raccoglimento Totale}}
%
%\begin{solution}
%\( \left[ + \dfrac{8}{9} \right] \)
%\end{solution}
\end{parts}

%\fillwithlines{1in}

\begin{solution}
La differenza sta nel fatto che i Numero Razionali - che appartiene cio� all'Insieme \( \Q \)	- per definizione hanno un SEGNO, in quanto composti da numeri Interi Relativi \( \Z \); si pu� anche dire, pi� precisamente, che una \emph{frazione} � il rapporto tra due numeri Naturali \( \N \), mentre un numero \emph{razionale} � il rapporto tra due numeri Interi Relativi dell'insieme \( \Z \).
\end{solution}

\vspace{.8cm}

% ------------------------------------------- Esercizio #3

%\addpoints
%\question Riconosci quali dei seguenti polinomi sono \emph{cubi di binomi}:\\
%
%\begin{choices}
%\CorrectChoice
%\quad \( -a^3 - 3 a^2b + 3ab^2 + b^3\)
%
%\vspace{0.2\baselineskip}
%
%\choice
%\quad \( a^9 - 6a^4b - 12a^2b^2 - 8b^3 \)
%\vspace{0.2\baselineskip}
%
%\choice
%\quad \( 8a^9 - b^3 - 6b^2a^3 + 12a^6b \)
%\vspace{0.2\baselineskip}
%
%\CorrectChoice
%\quad \( \dfrac{1}{27} a^6 - 8b^3 + 4a^2b^2 - \dfrac{2}{3} a^4b \)
%
%\end{choices}
%
%\vspace{.8cm}

% ------------------------------------------- Esercizio #34
%\addpoints
\question Semplifica le seguenti espressioni in \( \Q \):

\begin{parts}
\vspace{.3cm}

%\part
%
%\( (a + b) \cdot (a - b) - (a + b)^2 \);
%
%\vspace{.3cm}
\part

\( \left[ \dfrac{2}{3} - \left( - \dfrac{1}{4} + \dfrac{2}{5} \right) \right] - \left[ \dfrac{3}{5} - \left( \dfrac{3}{4} - \dfrac{1}{3} \right) \right] \);

%{\emph{Fattorizzare}} un polinomio, che � generalmente espresso come \fillin[somma][40pt] algebrica di monomi, nel \fillin[prodotto][45pt]  di altri polinomi di grado \fillin[inferiore][45pt] a quello del polinomio assegnato inizialmente.

\part

\( \left\{ \left( \dfrac{2}{5} \right)^{4} \cdot \left[ \left( \dfrac{2}{5} \right)^{8}  : \left( \dfrac{2}{5} \right)^{3} \right]^{2} \right\}^{2} : \left[ \left( \dfrac{2}{5} \right)^{3} \cdot \dfrac{2}{5} \cdot \left( \dfrac{2}{5} \right)^{3} \right]^{4} \);

%Il Quadrato di un Binomio � un {\fillin[Prodotto][45pt]} Notevole e la sua espressione � la seguente:\\ 
%
%\( (a + b )^2 = (\) \fillin[\(a + b \)][30pt] \() \cdot (a + b) = a^2 \, + \) \fillin[\( 2ab\)][20pt] \( + \, b^2 \)

\part

\( \left( 2 + \dfrac{1}{2} \right)^{2} : \left( 2 - \dfrac{1}{2} \right)^{-2} + \left[ \left( 2 + \dfrac{1}{3} \right) \cdot \left( \dfrac{7}{3} \right)^{-2} \right]^{-1} \)
\end{parts}

\vspace{0.3cm}
\question Risolvi i seguenti problemi utilizzando le frazioni:
\begin{parts}

\part
Luigi ha 18 anni, cio� i \( \dfrac{3}{7} \) dell'et� di sua madre che a sua volta ha i \( \dfrac{4}{5} \) dell'et� del marito. Quali sono l'et� della madre e del padre di Luigi?
\fillwithlines{1in}

\vspace{0.3cm}

\part
Un televisore a \( \dfrac{16}{9} \) ha la base di 18 pollici. Quanti pollici misura l'altezza?
\fillwithlines{1in}

\end{parts}
\end{multicols}
\end{questions}


%\vspace{.6cm}
%
%% ------------------------------------------- Esercizio #4 - Facoltativo (bonus points)
%
%\addpoints
%\bonusquestion[8] {\em Esercizio facoltativo:}
%
%Esegui il seguente Prodotto Notevole: \[ \left(- \dfrac {3}{4}a + 4b^2 \right)^2 \]
%
%\begin{solution}
%\[ \left(- \dfrac {9}{16}a^2 - 6ab^2 + 16b^4 \right)^2 \]
%\end{solution}
%

%
%\noindent
%\rule[2ex]{\textwidth}{1pt}
%
%\begin{center}
%{\bf Tabella dei punteggi}
%\vspace{10pt}
%
%\combinedgradetable[h][questions]
%\end{center}
%\vspace{4pt}
%\footnotesize La sufficienza � fissata a 18 punti, ma potr� subire delle modifiche in fase di correzione, al fine di garantire la validit� della prova anche in caso di andamenti troppo scostanti della media-classe.

%\begin{flushright}
%	\emph{tempo medio stimato: 40 minuti}
%\end{flushright}
\end{document}
