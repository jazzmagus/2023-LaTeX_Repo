% ----------------------------------------------------------------------
% Template VERIFICA
% ----------------------------------------------------------------------
% 2020 di d!egofantinelli at jazzmagus@gmail.com
% ----------------------------------------------------------------------

% ---------------------------------- Preambolo
\documentclass[11pt, a4paper, landscape, answers]{exam}
\usepackage[T1]{fontenc}
\usepackage{mdframed}
%\usepackage{nicefrac}
%\usepackage[applemac]{inputenc}
%\usepackage[utf8]{inputenc}
\usepackage[italian]{babel}
\usepackage[margin=1.3in]{geometry}
\usepackage{amsfonts, amsthm, amsmath, amssymb}
\usepackage{multicol}
\usepackage{mathrsfs}
\usepackage[none]{hyphenat}
\usepackage{bbm}
\usepackage{graphicx}
\usepackage{tikz}

\usepackage{upquote}
\usepackage{caption}
%\usepackage{fancyhdr}
\usepackage{float}


%\checkboxchar{$\square$}
%\checkedchar{$\boxtimes$}
%\CorrectChoiceEmphasis{}  % solo per non avere bold nelle risposte corrette
%

%% Create a Matching question format
\newcommand*\Matching[1]{
\ifprintanswers
    \textbf{#1}
\else
    \rule{1in}{0.5pt}
\fi
}
\newlength\matchlena
\newlength\matchlenb
\settowidth\matchlena{\rule{1.1in}{0pt}}
\newcommand\MatchQuestion[2]{%
    \setlength\matchlenb{\linewidth}
    \addtolength\matchlenb{-\matchlena}
    \parbox[t]{\matchlena}{\Matching{#1}}\enspace\parbox[t]{\matchlenb}{#2}}

\renewcommand{\thequestion}{\arabic{question}}
\renewcommand{\thepartno}{\alph{partno}}
\renewcommand{\thechoice}{\alph{choice}}

\renewcommand{\checkboxeshook}{
  \setlength{\labelsep}{2.4em}
  \setlength{\leftmargin}{4em}
}


\newcommand{\numberset}{\mathbb}
\newcommand{\N}{\numberset{N}}
\newcommand{\Z}{\numberset{Z}}
\newcommand{\Q}{\numberset{Q}}
\newcommand{\R}{\numberset{R}}

\newcommand{\ChoiceLabel}[1]{\hspace{-1.6em}\makebox[1.6em][l]{\textbf{#1.}}\ignorespaces}

\newcommand{\WrongChoice}[1]{\choice \ChoiceLabel{#1}}
\newcommand{\RightChoice}[1]{\correctchoice \ChoiceLabel{#1}}
\newcommand{\Item}[1]{\hspace{-17pt}\makebox[17pt][l]{\textbf{#1.}}\ignorespaces}


\renewcommand{\solutiontitle}{\noindent\textbf{Soluzione:}\par\noindent}

\renewcommand{\questionshook}{%
    \setlength{\leftmargin}{0pt}%
}
\renewcommand{\choiceshook}{%
    \setlength{\leftmargin}{20pt}%
}

% ---------------------------------- Intestazione
\newcommand{\class}{\LARGE {Esercizi proposti}}
%\newcommand{\term}{I Quadrimestre}
\newcommand{\examnum}{Verifica numero: 1}
\newcommand{\examdate}{\today}
\newcommand{\timelimit}{40 minuti}
\CorrectChoiceEmphasis{\color{red}}
\SolutionEmphasis{\color{red} \footnotesize}
\renewcommand{\solutiontitle}{\noindent\textbf{Soluzione:}\par\noindent}
% ---------------------------------- Intestazione

\pagestyle{headandfoot}
\firstpageheader{ISS "G. A. Remondini" - Bassano del Grappa (VI)}{}{\examdate}
\runningheader{\footnotesize VERIFICA di RECUPERO di MATEMATICA}{}{Classe 3\string^QA}
\runningheadrule

\firstpagefooter{}{}{pag. \thepage\ di \numpages}
\runningfooter{}{}{pag. \thepage\ di \numpages}
\runningfootrule

% ---------------------------------- Punteggi
\pointpoints{punto}{\em punti}
\pointformat{[{\footnotesize \thepoints}]}
\bonuspointpoints{punto bonus}{\em punti bonus}
\bonuspointformat{[{\footnotesize \thepoints}]}
\pointsinrightmargin
\setlength{\rightpointsmargin}{.2cm}
\chqword{Esercizio}
\chpword{Punti}
\chbpword{Punti Bonus}
\chsword{Punteggio}
\chtword{Totale}

\begin{document}

% ---------------------------------- Title Page
\noindent
\rule[2ex]{\textwidth}{0.5pt}\\
{\huge{\bf \class}}\\[20pt]
%{\huge{ \term}}\\[8pt]
%\rule[2ex]{\textwidth}{0.5pt}\\

%\vspace{3cm}
%\begin{tabular*}{\textwidth}{l @{\extracolsep{\fill}} r @{\extracolsep{6pt}} l}
%\textbf{} & \textbf{Nome e Cognome:} & \makebox[2.5in]{\hrulefill}\\
%\textbf{} &&\\
%\textbf{} & \textbf{Classe:} & \makebox[2.5in]{\Large{\bf 2 \string^ C}}\\
%\textbf{} &&\\
%\textbf{} & Tempo a disposizione: & \makebox[2.5in]{\timelimit}\\
%\textbf{} &&\\
%\textbf{} &&\\
%\textbf{} &&\\
%\textbf{} & {\em prof.:} & \makebox[2.5in]{\em Diego Fantinelli}
%\end{tabular*}\\
%
%\vspace{5cm}
% ---------------------------------- Avvertenze

%\noindent
%%\rule[2ex]{\textwidth}{0.2pt}
%\textbf{Avvertenze}:
%\begin{itemize}
%	\item La presente Verifica - che viene somministrata in modalit� DDI - contiene \numquestions \; quesiti, per un totale di \numpoints \;punti, di cui uno facoltativo di \numbonuspoints \;punti, che verr� conteggiato soltanto se verranno svolti anche tutti i precedenti.
%	\item La webcam dovr� rimanere accesa per tutto il tempo della verifica (\timelimit), salvo impossibilit� concrete di connessione; il microfono rester� spento e verr� acceso soltanto per chiarimenti e domande, che saranno consentite negli ultimi 20 min di prova.
%	\item E' vietato l'utilizzo di calcolatrici scientifiche, smartphone, tablet e altri dispositivi digitali, nonch� la consultazione di testi, appunti e siti web.
%
%\end{itemize}
%%\rule[2ex]{\textwidth}{0.2pt}
%\vfill
%\newpage
%

% =========================================== VERIFICA 


% ------------------------------------------- Esercizio #1

\begin{questions}

\begin{multicols}{2}
%\addpoints


\question
Calcola il valore delle seguenti espressioni nell'Insieme \( \Z \):\\


\begin{parts}

\part
\( -5 \cdot (12 - 3 + 4) - 2 \cdot [3 - 16 : (- 2 + 4)]^{2} \);

%{\footnotesize {\emph{Suggerimento:} Raccoglimento Totale}}

\begin{solution}

\( [- 115] \)
\end{solution} 

\vspace{.3cm}
\part
\( [- 3 + (- 5) \cdot (-1)]^{3} + [- 4 - (1 - 2)]^{2} \);

%{\footnotesize {\emph{Suggerimento:} Raccoglimento Totale}}

\begin{solution} 
\( [+ 17] \)
\end{solution}

\vspace{0.3cm}
\part
\( [2 \cdot (- 3)^{2} + 2 \cdot (- 3) \cdot (- 2)]^{2} : [2^4 - 3 \cdot (+6)]^{2} \);

%{\footnotesize {\emph{Suggerimento:} Raccoglimento Parziale e poi Totale}}

\begin{solution} 
\( [+ 225]\) 
\end{solution}

\vspace{.3cm}
\part
\( (- 3)^{2} \cdot(4 -1)^{5} : [(- 4)^{3} : (2^{5}) - 3^{3} : (- 3)^{3}] \);

%{\footnotesize {\emph{Suggerimento:} Raccoglimento Totale}}

\begin{solution}
\( [- 1] \)
\end{solution}

\end{parts}

\vspace{2cm}


\question
Calcola il valore delle seguenti espressioni negli Insiemi \( \Z \) e \( \Q \):\\

\begin{parts}

\part
\( \left[+ \dfrac{2}{5} + \left( - \dfrac{4}{5} \cdot  \dfrac{15}{12}\right)^{2} \right] + \dfrac{3}{5} \);

%{\footnotesize {\emph{Suggerimento:} Raccoglimento Totale}}

\begin{solution}

\( [+ 2] \)
\end{solution} 

\vspace{.3cm}
\part
\( \left[ \left( - \dfrac{3}{8} - 2 \right) + \dfrac{5}{4} \right] \cdot \left( 1 - \dfrac{17}{18}  \right) \);

%{\footnotesize {\emph{Suggerimento:} Raccoglimento Totale}}

\begin{solution} 
\( \left[ - \dfrac{1}{16} \right] \)
\end{solution}

\vspace{0.3cm}
\part
\( 3 - \dfrac{2}{5} \cdot \left[ \left( - \dfrac{18}{5} \right)^{2} \cdot \left( + \dfrac{25}{9} \right)^{2} \right] : \dfrac{20}{3} \);

%{\footnotesize {\emph{Suggerimento:} Raccoglimento Parziale e poi Totale}}

\begin{solution} 
\( [- 3] \) 
\end{solution}

\vspace{.3cm}
\part
\( + \dfrac{4}{3} - \left[ \left( + \dfrac{15}{22} \right) : \left( + \dfrac{45}{44} \right) \right]^{4} : \left( + \dfrac{20}{3}  \right)^{2} \);

%{\footnotesize {\emph{Suggerimento:} Raccoglimento Totale}}

\begin{solution}
\( \left[ + \dfrac{8}{9} \right] \)
\end{solution}
\end{parts}

\end{multicols}

\question
Qual � la differenza tra una Frazione e un Numero Razionale?\\
%\fillwithlines{1in}

\begin{solution}
La differenza sta nel fatto che i Numero Razionali - che appartiene cio� all'Insieme \( \Q \)	- per definizione hanno un SEGNO, in quanto composti da numeri Interi Relativi \( \Z \); si pu� anche dire, pi� precisamente, che una \emph{frazione} � il rapporto tra due numeri Naturali \( \N \), mentre un numero \emph{razionale} � il rapporto tra due numeri Interi Relativi dell'insieme \( \Z \).
\end{solution}

\end{questions}


% ------------------------------------------- Esercizio #2

%\addpoints
%\question [8] Quali delle seguenti fattorizzazioni sono corrette?
%
%\begin{multicols}{2}
%\begin{choices}
%\choice
%\(48x^2 - 4 = (6x + 2) \cdot (6x - 2)\)
%\vspace{0.5\baselineskip}
%
%\CorrectChoice
%\((x^4 - 9y^2) \cdot (x^4 - 9y^2) = (x^8 -49y)^4\)
%\vspace{0.5\baselineskip}
%
%\choice
%\(5x - 15 + x(3 - x) = (x + 3) \cdot (x - 5)\)
%\vspace{0.5\baselineskip}
%
%\CorrectChoice
%\((5a - 2b)^2 = 25a^2 - 20ab + 4b^2\)
%
%\end{choices}
%\end{multicols}


% ------------------------------------------- Esercizio #3 

%\addpoints
%\question Completa le seguenti definizioni:
%
%\begin{parts}
%\vspace{.3cm}
%\part [6]
%
%{\emph{Fattorizzare}} un polinomio, che � generalmente espresso come \fillin[somma][40pt] algebrica di monomi, nel \fillin[prodotto][45pt]  di altri polinomi di grado \fillin[inferiore][45pt] a quello del polinomio assegnato inizialmente.
%
%\vspace{.5cm}
%
%\part [6]
%
%Il Quadrato di un Binomio � un {\fillin[Prodotto][45pt]} Notevole e la sua espressione � la seguente:\\ 
%
%\( (a + b )^2 = (\) \fillin[\(a + b \)][30pt] \() \cdot (a + b) = a^2 \, + \) \fillin[\( 2ab\)][20pt] \( + \, b^2 \)
%
%\end{parts}
%\vspace{.6cm}
%
%% ------------------------------------------- Esercizio #4 - Facoltativo (bonus points)
%
%\addpoints
%\bonusquestion[8] {\em Esercizio facoltativo:}
%
%Esegui il seguente Prodotto Notevole: \[ \left(- \dfrac {3}{4}a + 4b^2 \right)^2 \]
%
%\begin{solution}
%\[ \left(- \dfrac {9}{16}a^2 - 6ab^2 + 16b^4 \right)^2 \]
%\end{solution}
%

%
%\noindent
%\rule[2ex]{\textwidth}{1pt}
%
%\begin{center}
%{\bf Tabella dei punteggi}
%\vspace{10pt}
%
%\combinedgradetable[h][questions]
%\end{center}
%\vspace{4pt}
%\footnotesize La sufficienza � fissata a 18 punti, ma potr� subire delle modifiche in fase di correzione, al fine di garantire la validit� della prova anche in caso di andamenti troppo scostanti della media-classe.
\end{document}
