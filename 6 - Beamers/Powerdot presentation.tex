\documentclass[
    13pt,
    style=klope, %powerdot style
    display=slidesnotes, %whether the notes will be displayed or not
    paper=smartboard, %Wide screen paper size
    orient=landscape, %Paper orientation
    ]{powerdot}

\pdsetup{
    trans=Split,     %Transition effect
    palette=Spring,  %Colour palette
}

\usepackage[utf8]{inputenc}

%This code sets the listings style
%------------------------------------------
\usepackage{listings}
\lstnewenvironment{code}{%
\lstset{frame=single,escapeinside=`',
%backgroundcolor=\color{green!20},
basicstyle=\footnotesize \ttfamily}
}{}
%------------------------------------------


% Presentation metadata
\title{Powerdot Presentation}
\author{Overleaf}
\date{\today}

\begin{document}
  % title slide
  \maketitle
  
  % section: title takes up a full slide
  \section{First section}
   
\begin{slide}{Slide Title}
You can see a list of items below. \pause \\ %First overlay
There are commands to make them appear sequentially
    \begin{itemize}[type=1]
      \item<2> This is an item
      \item<3> Second item
      \item<4> Third item
    \end{itemize}
\end{slide}
%Note corresponding to the previous slide
%-----------------------------------------------------
\begin{note}{About items}
    Mention that lists of items can be customised.
\end{note}
%-----------------------------------------------------
  
\begin{slide}[method=direct]{Slide 2} 
%Extra options in this slide to include verbatim text
Steps 1 and 2:
\begin{code}

compute a;

compute b;
\end{code}
\end{slide}
  
\begin{slide}{Slide N 2}
  This is the content of slide 2.
  Math $x=2\pi r$.
\end{slide}
\begin{note}{about the circle}
  The mathematical formula in this slide is the one used to compute the area of a circle
\end{note}
  
\end{document}