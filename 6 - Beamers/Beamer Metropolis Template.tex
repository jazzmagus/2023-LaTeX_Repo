\documentclass{beamer}
\usetheme{metropolis} % Use metropolis theme


\title{Lezione di Matematica: Derivate}
\date{\today}
\author{Diego Fantinelli}
\institute{Matematica per il Liceo}

\begin{document}
\maketitle
\section{Le Derivate}

\begin{frame}{\emph{Derivate}}
\begin{itemize}
	\item primo elemento della lista
	\item secondo elemento della lista
	\item terzo elemento della lista
\end{frame}

\begin{frame}{Domande}
In questa lezione vedremo come analizzare la scrittura: $a^2+b^2=c^2$, e capiremo come cambiare la prospettiva con cui si osserva una formula matematica\\

\[
-Delta u=f, quad \quad x\in\Omega
\]
\end{frame}

\end{document}