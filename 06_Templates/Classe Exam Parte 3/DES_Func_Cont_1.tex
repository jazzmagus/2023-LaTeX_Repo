\question Seja \(g\) a função, de domínio \(\mathds{R}\), definida por
		
		\[
		g(x)=
		\begin{cases}
		\dfrac{1-x^2}{1-e^{x-1}} 			& \text{se } x<1\\
		2 									& \text{se } x=1\\
		3+\dfrac{\sin\left(x-1\right)}{1-x} & \text{se } x>1
		\end{cases}
		\]
		
		Resolva os itens \ref{part:continuidade} e \ref{part:equacao} recorrendo a métodos analíticos, sem utilizar a calculadora. 
		
		\begin{parts}
			\part[15]\label{part:continuidade} Estude a função \(g\) quanto à continuidade no ponto 1.
				
				\begin{solutionorgrid}[5cm]
					Aqui será colocada a resposta.
				\end{solutionorgrid}
			
			\part[15]\label{part:equacao} Resolva, no intervalo \(\big]4,5\big[\), a equação \(g(x)=3\).
		\end{parts}