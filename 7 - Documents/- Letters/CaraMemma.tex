\documentclass[11pt]{article}
%\usepackage{blindtext}
\usepackage{fancyhdr}
\usepackage{changepage}

\usepackage[utf8]{inputenc}
\usepackage[italian]{babel}
\usepackage{epigraph}
\usepackage{hyperref}
\hypersetup{
   colorlinks=true,
   linkcolor=blue,
   filecolor=magenta,      
	urlcolor=cyan,
}
\urlstyle{same}
\setlength{\parindent}{0em}
\setlength{\parskip}{1em}
\renewcommand{\baselinestretch}{1.5}

\usepackage{setspace}
\usepackage{quoting}
\quotingsetup{font=small}
\usepackage{fancyhdr}

\begin{document}
\pagestyle{empty}
\onehalfspacing
\emph{Bassano del Grappa,\qquad 17 agosto 2020} \\
\line(1,0){200} \\

Amatissima Memma,

Ti lascio qualcosa da leggere durante il tuo viaggio in una delle più pazzesche città del mondo: la mia preferita, senza dubbio.

Non so bene da dove partire, perché ammetto che mi emoziona molto scriverti, pertanto proverò ad improvvisare; non sarà difficile perchè si parlerà della Memma.\\
Non preoccuparti, non ho intenzione di propinarti un pistolotto infarcito di retorica sul significato di avere diciotto anni, anche se mi commuove rendermene conto e anch'io, come la tua {\em maman}, mi sono chiesto: {\em "Ma com'è successo?!?"}.\\
Mi sono risposto così: {\em “E’ successo un po’ alla volta, con la lentezza con cui accadono le cose importanti, quelle che durano di più: in continua impercettibile mutazione; un’evoluzione senza soluzione di continuità: la ricetta della resilienza autentica”.}

Prendo spunto da una frase della tua prof., (della quale non ricordo il nome): {\em “Gemma può scegliere di fare ciò che vuole\dots”}.\\ Ecco, quando qualcuno ti dice che puoi fare ciò che vuoi per me significa che puoi fare ciò che più ti piace/soddisfa/gratifica: è una libertà di scelta immensa e te la sei guadagnata giorno dopo giorno; ma significa anche che la società adulta ti ritiene talmente responsabile che si fida ciecamente di te. Non è poco.\\
Un mio prof. di matematica una volta a lezione disse una frase che uso tutt'ora con i miei studenti: {\em "La vita non è altro che la sommatoria, dalla nascita fino alla dipartita, di tutte le scelte fatte - e non fatte - che si traducono nella vostra, personalissima definizione di Felicità".}\\
Mi piace pensare alla crescita personale come a una sommatoria di scelte, più o meno importanti, più o meno difficili, più o meno definitive, più o meno discutibili, più o meno rischiose, etc.\\
La mia sensazione è che tu abbia compiuto delle scelte che lasciano trasparire ciò che hai scelto di essere, come hai scelto di vivere senza limitarti ad esistere e il risultato,  {\em so far}, è spettacolare!\\
Se è vero, come credo, che la maturità di una persona non ha un'età precisa, credo che lo stesso valga per la saggezza.\\ E' questa l'idea che mi sono fatto della Memma: un essere umano capace di cogliere l'essenza delle cose, perché ne ha la sensibilità e anche perché sa estrarre dalle sue esperienze il sapere necessario per proseguire, anche ad occhi chiusi se necessario, verso la propria realizzazione personale.

E' una sorta di superpotere.\\
Mi spiego meglio.\\
Quand’ero più giovane, (che frase da {\em Boomer!}), le serie TV si chiamavano Telefilm, come “Un Posto al Sole” per intenderci; uno di questi si intitolava {\em Ralph SuperMaxiEroe}, e raccontava di un prof. di un College americano che, assieme al suo amico Max dell’FBI, riceveva dagli UFO un costume da Supereroe, per sistemare un po’ di cose sulla Terra. I due però, nella concitazione, perdono il libretto delle istruzioni del costume e Ralph è costretto ad imparare da autodidatta.\\
Morale della favola? {\em Nemmeno un Supereroe è un Supereroe finché non diventa consapevole dei propri superpoteri.}
\begin{adjustwidth}{2cm}{}
{\small A questo \href{https://youtu.be/7JeyT7WvcOM}{\em link} trovi la sigla originale, che riassume tutta la trama in una specie di {\em spoilerone}\dots \, a me faceva riderissimo.}
\end{adjustwidth}
Ecco perché quando dico che per me {\em sei speciale} intendo che ho già visto il tuo costume da Supereroe piegato e stirato nell’armadio.\\
Lungi da me caricarti delle mie aspettative, perché non ne ho: sono già felice così.\\
Ti faccio un esempio banale, con {\em "i numeri"}. Ora la Memma può votare se non erro. Ecco, io sono felice di sapere che d’ora in avanti la Memma, con le sue idee, potrà contribuire anche istituzionalmente al futuro del posto in cui vivo.
Mi dà un senso di tranquillità e di speranza.\\ Capisci cosa fai senza "fare niente"?

La tua permanenza al Liceo Classico, (ho adorato questa tua scelta\dots ), è stata esemplare e ora, che da qualche anno frequento anch'io le aule insegnanti, mi rendo conto di cosa significhi avere in una classe anche solo uno studente come te; non è una sviolinata, ma una profonda manifestazione di stima.

Permettimi di spendere due parole anche sul tuo futuro, dopo il Liceo.\\ 
Sì sì, lo so che c'è ancora tutto un anno scolastico, non lo do per scontato, ma sarà un anno particolare, molto diverso dagli altri anni, non solo a causa della pandemia: da gennaio somiglierà molto più a una gara di immersione, tutta in apnea, fino alla liberatoria riemersione finale con l'Esame di Stato.

Se al Liceo \emph{cavarsela da soli} è una dote che si acquisisce un po’ alla volta, all’Università l’imperativo è \emph{cavarsela}.\\
Lo studio, a partire da luglio, assumerà una connotazione tutta nuova, gli insegnanti avranno tutto un altro ruolo  ma, soprattutto, sarai tu a gestire il tuo tempo e, di  conseguenza, i ritmi di studio; l'obiettivo principale sarà: arrivare preparata al prossimo esame, "con qualsiasi mezzo". Avrai tutta la libertà per prepararti come preferisci, servendoti degli strumenti che riterrai più efficaci per raggiungere lo scopo.

Tra le altre cose, ti potrebbe far comunque comodo in certi momenti avere dei riferimenti, dei piccoli faretti che, alla bisogna, ti possano illuminare la strada quel tanto che basta per non allontanarti troppo dal sentiero; hai presente quelle mini-torce di Decathlon che si ricaricano a mano? Ecco, mi piacerebbe essere una di quelle.
Non sarà più un "chiedere un aiuto", (che crea quel fastidioso "effetto imbarazzo" che si innesca quando si chiede un favore), bensì una {\em collaborazione}: due adulti che si {\em confrontano}, e che si trovano a diversi livelli di competenze, per uno scambio di conoscenze che è \underline{sempre} bidirezionale, dunque è sempre una crescita.

Ti regalo il mio tempo, tutto quello che ti servirà, nel momento in cui ne avrai bisogno perché, alle persone che si amano, i regali si fanno {\em per sempre}.

Veniamo ora al mio regalo più "pratico", che prevede alcuni passaggi tecnici che scoprirai al tuo rientro.\\
Ancora una volta ho preso spunto dalle edificanti parole della tua prof. di cui sopra che si è offerta di contribuire concretamenteai tuoi studi; credimi, non ho mai sentito una tale manifestazione concreta di stima da parte di un insegnante, fin qui almeno.
Ho cercato quindi un modo per realizzare praticamente questa cosa.\\
L’idea è quella di istituire il Fondo: {\em “Il Capitalismo non fermerà la sete di sapere della Memma”}, ove tu possa far confluire ogni introito, da lavoro, mance, regali, etc.\\
Sarà il tuo primo vero Conto Corrente con tanto di Bancomat e/o Carta Prepagata (che ti consiglio) per i tuoi studi, acquisti, viaggi, {\em quellochevorrai}, e gestire le tue finanze dall’app della banca.\\
Il mio vero regalo sarà il primo bonifico a tuo favore: se vuoi vedila un po’ come quando si regala un portafogli e, per scaramanzia o buon auspicio, si lascia una moneta all’interno.

Se potessi ti regalerei direttamente la Felicità, ma dato che pare essere una cosa caratterizzata da forte soggettività, ti lascio la sua {\em formula matematica}. Non è uno scherzo, ci hanno lavorato molti cervelli per parecchio tempo, (le persone fanno cose strane), eccola qua:

	{\em $\textbf{Happiness(t)}=w_{0}+w_{1} \cdot\displaystyle{\sum_{j=1}^{t}} \gamma^{t-j} C R_{j}+w_{2} \cdot\sum_{j=1}^{t} \gamma^{t-j} E V_{j}+w_{3} \cdot\sum_{j=1}^{t} \gamma^{t-j} R P E_{j}$}
	
	\begin{quoting}{\textbf{\small Formula della Felicità:} {\em La felicità è in funzione del tempo $t$; $w_0$, $w_1$, $w_2$ e $w_3$ sono costanti che indicano l’influenza dei diversi tipi di eventi; $\gamma$ è un “forgetting factor” (fattore dimenticando) che rende gli eventi degli studi più recenti più influenti rispetto a quelli precedenti; $CRj$ è la gratificazione ottenuta dalla scelta su un processo $j$; $EVj$ è la valutazione del rischio su di un processo $j$; $RPEj$ rappresenta la differenza tra la ricompensa desiderata e quella effettivamente ottenuta dal processo $j$.\\
Questo modello spiega le fluttuazioni della felicità momento dopo momento, mettendo in luce quanto un evento recente sia valutato da ogni individuo più importante di uno precedente.\\
Inoltre, fattore veramente rilevante è l’aspettativa: solitamente tendiamo a caricarci di aspettative e nel momento in cui una circostanza si verifica ci ritroviamo delusi in quanto l’avvenimento non è all’altezza di ciò che avevamo immaginato. "E' così, fratellì!"}}
\end{quoting}

Battute a parte, per te vorrei il meglio che c’è in questo posto pazzesco dove, per una serie di strane coincidenze, ci è stato fatto dono di vivere un’esperienza esistenziale davvero incredibile. \\
Ciò che ti auguro è una vita come quella che troverai descritta dettagliatamente nell'altro mini-regalino che sta già viaggiando verso Schio e che troverai al tuo rientro.\\
Non è il puzzle di un matematico pazzo; è che con te non c'è bisogno di scendere nel dettaglio, basta darti una traccia, uno spunto, ché tu poi ci costruisci meraviglie, {\em meraviglia}.

\epigraph{“Vien dietro a me, e lascia dir le genti:
sta come torre ferma, che non crolla
già mai la cima per soffiar di venti”}{\textit {Dante Alighieri, \\ Divina Commedia - Canto V}}
%\vspace{4pt}
Tuo, {\em per sempre}.

E!o

\vspace{18pt}
P.S.: un piccolo consiglio per il tuo viaggio a Roma: ci ho vissuto un anno e ti assicuro che vedere i Fori Imperiali di notte è un'esperienza quasi estatica.
\end{document}