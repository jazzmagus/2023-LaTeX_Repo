%% Overleaf: The \marks command was renamed to 
%% \Qmarks for correct compilation (30 September 2016)

\documentclass[12pt,fleqn]{exampaper}

\def\semester{SEMESTER 1}
\def\acadyear{2006-2007}
\def\examdate{August/September 2006}
\def\subjectcode{EExxxx}
\def\subjecttitle{The `exampaper' LaTeX Class File}
\def\timeallowed{$2.5$}

\begin{document}

% Instructions to candidates
\begin{enumerate}
  \item This paper contains FOUR (4) questions and comprises TWO (2) pages.
  \item Answer all FOUR (4) questions.
  \item All questions carry equal marks.
\end{enumerate}

\vspace{10mm} \hrule \vspace{10mm}

% The questions begins

\begin{enumerate}
  \item This is question 1. It consists of three parts
  \begin{enumerate}
    \item This is Part (a)
    \Qmarks{5}
    \item This is Part (b), and
    \Qmarks{10}
    \item This is Part (c)
    \Qmarks{10}
  \end{enumerate}

  \item
  \begin{enumerate}
    \item This is Part (a)
    \Qmarks{5}
    \item This is Part (b), and
    \Qmarks{10}
    \item This is Part (c)
    \Qmarks{10}
  \end{enumerate}

\newpage

  \item This is question 3. It consists of three parts
  \begin{enumerate}
    \item This is Part (a)
    \Qmarks{5}
    \item This is Part (b), and
    \Qmarks{10}
    \item This is Part (c)
    \Qmarks{10}
  \end{enumerate}

  \item This is question 4. It consists of three parts
  \begin{enumerate}
    \item This is Part (a)
    \Qmarks{5}
    \item This is Part (b), and
    \Qmarks{10}
    \item This is Part (c)
    \Qmarks{10}
  \end{enumerate}

\end{enumerate}

\end{document}
