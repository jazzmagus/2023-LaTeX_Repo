\documentclass[addpoints]{exam}

%--------------------------------------------------------------------------------------------------------
% PACKAGES USADOS
%--------------------------------------------------------------------------------------------------------

\usepackage[portuguese]{babel}
\usepackage[utf8]{inputenc}
\usepackage{geometry}
\usepackage{xpatch}
\usepackage{dsfont}
\usepackage{mathtools}
%--------------------------------------------------------------------------------------------------------
% OUTRAS CONFIGURAÇÕES
%--------------------------------------------------------------------------------------------------------

\geometry{a4paper,
	top=1cm,	
	bottom=2cm,	
	left=1.5cm,
	right=3cm,
}

%--------------------------------------------------------------------------------------------------------
% CONFIGURAÇÕES DA CLASSE EXAM
%--------------------------------------------------------------------------------------------------------

\renewcommand{\thequestion}{\bf \arabic{question}}
\renewcommand{\choicelabel}{{\bf (\thechoice)}}
\renewcommand{\partlabel}{\bf \thequestion.\arabic{partno}}
\pointpoints{ponto}{pontos}
\pointformat{[\bf \thepoints]}
\pointsinrightmargin
\setlength{\rightpointsmargin}{.5cm}
\hqword{Questão:}
\hpword{Cotação:}

\xpatchcmd{\oneparchoices}{\penalty -50\hskip 1em plus 1em\relax}{\hfill}

%--------------------------------------------------------------------------------------------------------
% INÍCIO DO DOCUMENTO
%--------------------------------------------------------------------------------------------------------

\begin{document}
	\begin{questions}
		\question[5] Considere todos os números naturais de quatro algarismos que se podem formar com os algarismos de 1 a 9.
		Destes números, quantos são múltiplos de 5?
		
		\begin{oneparchoices}
			\choice 729
			\choice 1458
			\choice 3645
			\choice 6561
		\end{oneparchoices}
	
		\question Seja \(g\) a função, de domínio \(\mathds{R}\), definida por
		
		\[
		g(x)=
		\begin{cases}
		\dfrac{1-x^2}{1-e^{x-1}} 			& \text{se } x<1\\
		2 									& \text{se } x=1\\
		3+\dfrac{\sin\left(x-1\right)}{1-x} & \text{se } x>1
		\end{cases}
		\]
		
			\begin{parts}
				\part[15] Estude a função \(g\) quanto à continuidade no ponto 1.
				
				\part[15] Resolva, no intervalo \(\big]4,5\big[\), a equação \(g(x)=3\).
			\end{parts}
		
		\vfill
		\center
		\pointtable[h]
	\end{questions}
\end{document}