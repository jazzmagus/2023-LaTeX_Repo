%%% ====================================================================
%%% @LaTeX-file{
%%%   filename  = "amsthdoc.tex",
%%%   version   = "2.04",
%%%   date      = "2000/06/06",
%%%   time      = "11:42:29 EDT",
%%%   checksum  = "45213 291 1500 12328",
%%%   author    = "American Mathematical Society",
%%%   address   = "American Mathematical Society,
%%%                Technical Support,
%%%                Electronic Products and Services,
%%%                P. O. Box 6248,
%%%                Providence, RI 02940,
%%%                USA",
%%%   telephone = "401-455-4080 or (in the USA and Canada)
%%%                800-321-4AMS (321-4267)",
%%%   FAX       = "401-331-3842",
%%%   email     = "tech-support@ams.org (Internet)",
%%%   copyright = "Copyright 1999, 2000 American Mathematical Society,
%%%                all rights reserved.  Copying of this file is
%%%                authorized only if either:
%%%                (1) you make absolutely no changes to your copy,
%%%                including name; OR
%%%                (2) if you do make changes, you first rename it
%%%                to some other name.",
%%%   supported = "yes",
%%%   keywords  = "latex, theorem, proof, amsthm",
%%%   abstract  = "This file is the User's Guide for the amsthm
%%%                package.",
%%%   docstring = "The checksum field above contains a CRC-16
%%%                checksum as the first value, followed by the
%%%                equivalent of the standard UNIX wc (word
%%%                count) utility output of lines, words, and
%%%                characters.  This is produced by Robert
%%%                Solovay's checksum utility.",
%%% }
%%% ====================================================================
%%% 
%%% Traduzione di Onofrio de Bari <onodebari@gmail.com>
%%%
%%% Aggiornata al 13 febbraio 2001
%%%
\documentclass[a4paper]{article}
\pagestyle{myheadings}
\usepackage[italian]{babel}
\usepackage{url,guit}
\title{Uso del pacchetto \pkg{amsthm}}
\author{Versione 2.07, 02/06/2000\\ American Mathematical Society\\
\scriptsize{Traduzione di Onofrio de Bari (onodebari@gmail.com)}}
\date{}

\providecommand{\qq}[1]{\textquotedblleft#1\textquotedblright}
\providecommand{\mdash}{\textemdash\penalty\hyphenpenalty}

% Embedded \index commands are a legacy from the time when this
% documentation was part of amsldoc. Since they don't hurt anything,
% let's leave them in. Maybe they will become useful again in the
% future. [mjd,2000/06/06]

\chardef\bslchar=`\\ % p. 424, TeXbook
\newcommand{\ntt}{%
  \fontfamily\ttdefault \fontseries\mddefault \fontshape\updefault
  \selectfont
}
\DeclareRobustCommand{\cn}[1]{{\ntt\bslchar#1}}
\DeclareRobustCommand{\cls}[1]{{\ntt#1}}
\DeclareRobustCommand{\pkg}[1]{{\ntt#1}}
\DeclareRobustCommand{\opt}[1]{{\ntt#1}}
\DeclareRobustCommand{\env}[1]{{\ntt#1}}
\DeclareRobustCommand{\fn}[1]{{\ntt#1}}

\providecommand{\qedsymbol}{\leavevmode
  \hbox to.77778em{%
  \hfil\vrule
  \vbox to.675em{\hrule width.6em\vfil\hrule}%
  \vrule\hfil}}

\hfuzz4pt \vbadness9999 \hbadness5000
\def\latex/{{\protect\LaTeX}}

\setlength{\textwidth}{210mm}\addtolength{\textwidth}{-2in}
\setlength{\oddsidemargin}{39pt}
\addtolength{\textwidth}{-2\oddsidemargin}

\begin{document}
\maketitle
\markboth{USO DEL PACCHETTO \pkg{amsthm}}{USO DEL PACCHETTO \pkg{amsthm}}
%%% Nota del traduttore
\subsubsection*{Nota alla traduzione italiana}

Una copia di questo documento e altre traduzioni in italiano di
manuali su \LaTeX\ sono reperibili presso
\begin{itemize}
\item\url{http://guild.prato.linux.it}
\item\url{ftp://lorien.prato.linux.it/pub/guild}
\item\url{ftp://ftp.unina.it/pub/TeX/info/italian}
\end{itemize}
e su ogni sito CTAN --- per esempio \url{ftp://ftp.tex.ac.uk/tex-archive} --- nella
directory \url{/info/italian}.

Dal giugno 2003 \`e attivo il \guittext (\guit), \TeX\ user group
ufficiale in Italia. \guit\ \`e raggiungibile all'indirizzo web
\begin{center}
  \guiturl
\end{center}
Non essendo pi\`u attivo il gruppo Gilda/Guild, potete rivolgervi al
\guit\ per ogni informazione riguardante \TeX, \LaTeX\ e la presente documentazione.
\begin{flushright}
  Onofrio de Bari (\guit)
\end{flushright}
%%%%%%%%%%%%%%%%%%%%%%%%%%%%%%%%%%%%%%%%%%%%%%%%%%%%%%%%%%%%%%%%%%%%%%%%
\section{Introduzione}

Il pacchetto \pkg{amsthm} offre una versione migliorata del comando
 \cn{newtheorem} di \latex/ per definire ambienti simili a teoremi. La
versione del comando \cn{newtheorem} nel pacchetto \pkg{amsthm} riconosce
 una specifico stile
\cn{theoremstyle} (come avviene nel pacchetto \pkg{theorem} di
 Mittelbach) ed \`e provvista di una forma \verb'*' per definire ambienti
 non numerati. Il pacchetto
\pkg{amsthm} definisce altres\`\i\ un ambiente \env{proof} che aggiunge
 automaticamente un simbolo Q.E.D. in coda. Le classi di documento AMS
 comprendono il pacchetto \pkg{amsthm}, quindi ogni situazione qui
 descritta \`e ad esse applicabile; alcuni esempi sono forniti nel file
\fn{thmtest.tex}.

\section{Il comando \cn{newtheorem}}

In articoli e libri relativi a ricerche matematiche, teoremi\index{teoremi}
e dimostrazioni\index{dimostrazioni} sono tra gli elementi pi\`u
comuni, ma gli autori ne usano anche molti altri che ricadono nella
stessa generica classe: lemmi, proposizioni, assiomi, corollari,
congetture, definizioni, note, casi, passi e cos\`\i\ via. Poich\'e questi
elementi formano una porzione del flusso di testo dai contorni ben
delimitati, sono naturalmente trattati in
\latex/ come ambienti. Le classi di documento \latex/ di norma non
forniscono ambienti predefiniti per gli elementi di tipo teorema perch\'e
(a)~ci\`o renderebbe difficile agli autori esercitare il necessario
controllo sulla numerazione automatica e (b)~la variet\`a di tali
elementi \`e cos\`\i\ ampia da rendere impossibile per una
classe di documento fornire ogni elemento che sarebbe richiesto. Esiste
invece un comando
\cn{newtheorem}, simile nell'effetto a \cn{newenvironment}, che facilita
gli autori nell'impostare gli elementi richiesti per un particolare documento.

Il comando \cn{newtheorem} ha due argomenti obbligatori; il primo \`e il
nome dell'ambiente che l'autore desidera usare per questo elemento; il
secondo \`e il testo che compare. Per esempio
\begin{verbatim}
\newtheorem{lem}{Lemma}
\end{verbatim}
significa che le istanze nel documento consistenti in
\begin{verbatim}
\begin{lem} Testo testo ... \end{lem}
\end{verbatim}
daranno luogo a
\[\makebox[.8\columnwidth]{%
  \textbf{Lemma 1.} \textit{Testo testo \dots}\hfill}\]
in cui l'intestazione \`e costituita dal testo specificato \qq{Lemma} e da
punteggiatura e numerazione automaticamente generati.

Se si usa \cn{newtheorem*} al posto di \cn{newtheorem} nell'esempio
sopra, non sar\`a generata la numerazione automatica per nessuno dei
lemmi nel documento . Questa forma del comando pu\`o essere utile se si
ha un solo lemma e non si vuole che sia numerato; pi\`u spesso,
comunque, \`e usata per produrre una variante di uno dei comuni tipi di
teoremi che sia dotata di un nome; se ad esempio si avesse un lemma il cui
nome dovrebbe essere \qq{Lemma di Klein} invece di \qq{Lemma} +
il numero, allora la dichiarazione
\begin{verbatim}
\newtheorem*{KL}{Lemma di Klein}
\end{verbatim}
permetterebbe di scrivere
\begin{verbatim}
\begin{KL} Testo testo ... \end{KL}
\end{verbatim}
e di ottenere l'\emph{output} richiesto.

\section{Modifiche alla numerazione}

In aggiunta ai due argomenti obbligatori, \cn{newtheorem} ha due
argomenti opzionali che si escludono vicendevolmente; questi riguardano
la sequenza\index{teeoremi!numerazione} e la gerarchia della numerazione.

Come impostazione predefinita, ogni genere di ambiente di tipo teorema
\`e numerato in maniera indipendente e pertanto, se si hanno tre lemmi e
due teoremi frapposti, essi risulteranno numerati nel seguente modo: Lemma 1, Lemma
2, Teorema 1, Lemma 3, Teorema 2. Se si desidera che i lemmi e i teoremi
condividano la stessa sequenza di numerazione\mdash Lemma 1, Lemma 2, Teorema 3, Lemma
4, Teorema 5\mdash si deve allora indicare la relazione desiderata nel
modo seguente:
\begin{verbatim}
\newtheorem{thm}{Teorema}
\newtheorem{lem}[thm]{Lemma}
\end{verbatim}
L'argomento opzionale \verb'[thm]' nella seconda dichiarazione fa in
modo che l'ambiente \texttt{lem} condivida la sequenza di numerazione di
\texttt{thm} invece di essere dotato di una sua sequenza indipendente.

Al fine di avere un ambiente teorema numerato in maniera subordinata a
un'unit\`a di sezionamento\mdash ad esempio per ottenere delle
proposizioni numerate come Proposizione 2.1,
Proposizione 2.2 e cos\`\i\ via nel Paragrafo 2\mdash si inserir\`a il nome
dell'unit\`a genitrice in parentesi quadre alla fine:
\begin{verbatim}
\newtheorem{prop}{Proposizione}[section]
\end{verbatim}
Con l'argomento opzionale \verb'[section]', il contatore \verb'prop'
sar\`a riportato a zero non appena verr\`a incrementato il contatore genitore \verb'section'.

\section{Cambiare gli stili in ambienti di tipo teorema}

\subsection{Il comando \cn{theoremstyle}}

Nel pacchetto \pkg{amsthm} \`e disponibile il concetto di stile di
teorema corrente,
che determina cosa sar\`a restituito da un dato comando \cn{newtheorem}.
I tre stili di teorema forniti\mdash
\verb'plain',\index{plain theo@\texttt{plain} stile di teorema}\relax
\index{definition theo@\texttt{definition} stile di teorema}\relax
\index{remark theo@\texttt{remark} stile di teorema} \verb'definition' e
\verb'remark'\mdash subiscono un differente trattamento tipografico che
ad essi fornisce un'enfasi visiva associato alla rispettiva
importanza. I dettagli di tale trattamento tipografico possono variare
in base alla classe di documento, ma in generale lo stile \verb'plain'
produce testo in corsivo, mentre gli altri due forniscono testo in tondo.

Per creare nuovi ambienti di tipo teorema in differenti stili,
si dividano i comandi \cn{new\-theorem} in gruppi e si premetta ad ogni
gruppo l'appropriato \cn{theo\-rem\-style}; se non si d\`a alcun comando \cn{theoremstyle}
, lo stile usato sar\`a \texttt{plain}. Alcuni esempi:
\begin{verbatim}
\theoremstyle{plain}% default
\newtheorem{thm}{Teorema}[section]
\newtheorem{lem}[thm]{Lemma}
\newtheorem{prop}[thm]{Proposizione}
\newtheorem*{cor}{Corollario}
\newtheorem*{KL}{Lemma di Klein}

\theoremstyle{definition}
\newtheorem{defn}{Definizione}[section]
\newtheorem{conj}{Congettura}[section]
\newtheorem{exmp}{Esempio}[section]

\theoremstyle{remark}
\newtheorem*{comm}{Commento}
\newtheorem*{note}{Nota}
\newtheorem{caso}{Caso}
\end{verbatim}

\subsection{Spostamento dei numeri}

Una frequente variazione di stile per i titoli di teoremi consiste
nell'avere il numero del teorema sulla sinistra, all'inizio del titolo,
invece che sulla destra. Poich\'e tale variazione \`e di solito eseguita
in maniera generalizzata non badando ai singoli cambi con \cn{theoremstyle},
lo spostamento dei numeri \`e effettuato posizionando un
comando\cn{swapnumbers} all'inizio della lista delle dichiarazioni
\cn{newtheorem} che devono essere modificate. Esempio:
\begin{verbatim}
\swapnumbers
\theoremstyle{plain}
\newtheorem{thm}{Teorema}
\theoremstyle{remark}
\newtheorem{rem}{Nota}
\end{verbatim}
Quando il pacchetto \pkg{amsthm} \`e usato con una delle generiche
classi di documento \latex/
come \cls{article} o \cls{book}, l'effetto delle dichiarazioni sopra
riportate consister\`a nell'avere i titoli dei teoremi e delle note
stampati nella forma \textbf{1.4~Teorema}, \textit{9.1~Nota}; con altre
classi di documento il risultato potrebbe essere diverso.

\subsection{Ulteriori possibilit\`a di personalizzazione}

Altre possibilit\`a di personalizzazione sono fornite dal pacchetto
\pkg{amsthm} nella forma del comando \cn{newtheoremstyle} e un metodo
per usare le opzioni del pacchetto per caricare definizioni
personalizzate dello stile dei teoremi; poich\'e queste caratteristiche
sono in certo qual modo oltre i bisogni dell'utente medio, la
discussione dei dettagli \`e rimandata al file di esempio
\fn{thmtest.tex} e al commento in \fn{amsclass.dtx}.

\section{Dimostrazioni}

Un ambiente \env{proof} predefinito che \`e fornito dal pacchetto \pkg{amsthm}
produce l'intestazione \qq{Proof} con interpunzione e spaziatura
appropriate.
%Qui vorrei mettere una n.d.t. che spieghi che se si usa babel con
%l'opzione italian risultera' scritto 'dimostrazione' invece di
%'proof'. Che ne dici?
L'ambiente \env{proof} \`e principalmente inteso per
dimostrazioni brevi, che non occupino pi\`u di una pagina o due;
dimostrazioni pi\`u consistenti sono in generale meglio realizzate
separatamente come \cn{section} o \cn{subsection} nel documento.

Un argomento opzionale dell'ambiente \env{proof} permette di utilizzare un
nome differente al posto di \qq{Proof}; se, ad esempio, si vuole che
l'intestazione della dimostrazione sia \qq{Dimostrazione del teorema
principale}, si scriver\`a
\begin{verbatim}
\begin{proof}[Dimostrazione del teorema principale]
\end{verbatim}

Un simbolo \qq{QED}, \qedsymbol, viene automaticamente aggiunto alla
fine dell'ambiente \env{proof}; per sostiturlo con un diverso simbolo di
fine dimostrazione, utilizzare \cn{renewcom\-mand} per ridefinire il comando \cn{qedsymbol}.
Per una lunga dimostrazione realizzata come sottoparagrafo o paragrafo
invece che con l'ambiente
\env{proof}, si pu\`o ottenere il simbolo e l'usuale spaziatura che lo
precede servendosi di \cn{qed}.

Il posizionamento del simbolo QED pu\`o essere problematico se l'ultima
parte di un ambiente \env{proof} \`e un'equazione in modalit\`a
\emph{display} o un ambiente lista o qualcosa di questo genere; in questo caso basta
posizionare un comando \cn{qedhere} nel punto in cui dovr\`a comparire
il simbolo QED.
\begin{verbatim}
\begin{proof}
...
\begin{equation}
G(t)=L\gamma!\,t^{-\gamma}+t^{-\delta}\eta(t) \qedhere
\end{equation}
\end{proof}
\end{verbatim}
Se il pacchetto \pkg{amsthm} viene utilizzato con una classe di documento non
AMS e con il pacchetto \pkg{amsmath}, il pacchetto \pkg{amsthm} deve
essere caricato
\emph{dopo} \pkg{amsmath} e non prima.\footnote{Il posizionamento sul
margine destro eseguito da \cn{qedhere} nelle equazioni in modalit\`a
\emph{display} funziona solo con la versione 2 del pacchetto
\pkg{amsmath} e non con le precedenti.}
Se \cn{qedhere} provoca un messaggio d'errore in un'equazione, si provi
a usare al suo posto \verb'\mbox{\qedhere}'.
\end{document}

