%% Author_tex.tex
%% V1.0
%% 2016/09/02
%% developed by Techset
%%
%% This file describes the coding for cta-author.cls

\documentclass{cta-author}%%%%where cta-author is the template name

\newtheorem{theorem}{Theorem}{}
\newtheorem{corollary}{Corollary}{}
\newtheorem{remark}{Remark}{}

%The authors can define any packages after the \documentclass{cta-author} command.

%\usepackage{amsmath} for dealing with mathematics,
%\usepackage{amsthm} for dealing with theorem environments,
%\usepackage{cite} for dealing with citations
%\usepackage{hyperref} for linking the cross references
%\usepackage{graphics} for dealing with figures.
%\usepackage{algorithmic} for describing algorithms
%\usepackage{subfig} for getting the subfigures e.g., "Figure 1a and 1b" etc.
%\usepackage{url} It provides better support for handling and breaking URLs.

%The author can find the documentation of the above style file and any additional
%supporting files if required from "http://www.ctan.org"

% *** Do not adjust lengths that control margins, column widths, etc. ***

\begin{document}

\title{Insert the article title here}

\author{\au{S. Cheng$^{1,2}$}%%% First author
\au{J.C. Ji$^1$}%%% Second author
\au{J. Zhou$^2$}%%% Third author
}

\address{\add{1}{....}%%% Author address here
%%% First group represent author affiliation number and second group represent name.
\add{2}{....}
\email{....}}

\begin{abstract}
The abstract text goes here.
\end{abstract}

\maketitle

\section{Insert A head here}
This demo file is intended to serve as a ``starter file''
for IMAIAI journal papers produced under \LaTeX\ using
imaiai.cls v1.5e.

\subsection{Insert B head here}
Subsection text here.


\subsubsection{Insert C head here}
Subsubsection text here.

\section{Equations}

Sample equations.

%%% Numbered equation
\begin{align}\label{1.1}
\begin{split}
\frac{\partial u(t,x)}{\partial t} &= Au(t,x) \left(1-\frac{u(t,x)}{K}\right)-B\frac{u(t-\tau,x) w(t,x)}{1+Eu(t-\tau,x)},\\
\frac{\partial w(t,x)}{\partial t} &=\delta \frac{\partial^2w(t,x)}{\partial x^2}-Cw(t,x)+D\frac{u(t-\tau,x)w(t,x)}{1+Eu(t-\tau,x)},
\end{split}
\end{align}

\begin{align}\label{1.2}
\begin{split}
\frac{dU}{dt} &=\alpha U(t)(\gamma -U(t))-\frac{U(t-\tau)W(t)}{1+U(t-\tau)},\\
\frac{dW}{dt} &=-W(t)+\beta\frac{U(t-\tau)W(t)}{1+U(t-\tau)}.
\end{split}
\end{align}

%%%% Unnumbered equation
\[
\frac{\partial(F_1,F_2)}{\partial(c,\omega)}_{(c_0,\omega_0)} = \left|
\begin{array}{ll}
\frac{\partial F_1}{\partial c} &\frac{\partial F_1}{\partial \omega} \\\noalign{\vskip3pt}
\frac{\partial F_2}{\partial c}&\frac{\partial F_2}{\partial \omega}
\end{array}\right|_{(c_0,\omega_0)}
\]

\section{Enunciations}
%%%% Most of the enunciations like theorem, lemma, corollary, proposition, defintion,
%%%% condition, example, conjecture etc. are defined in the class file.

%%%% If the author wants to add or modify the enunciation style
%%%% they can define in the preamble as shown below.

%%%% \newtheoremstyle{theorem}{6pt}{6pt}{\rm}{}{\sffamily}{ }{ }{}
%%%% \theoremstyle{theorem}
%%%% \newtheorem{theorem}{\sc Theorem}[section]

%%%%\newtheoremstyle{corollary}{6pt}{6pt}{\rm}{}{\sffamily}{ }{ }{}
%%%%\theoremstyle{corollary}
%%%%\newtheorem{corollary}{\sc Corollary}[section]

%%%%\newtheoremstyle{definition}{6pt}{6pt}{\rm}{}{\sffamily}{ }{ }{}
%%%%\theoremstyle{definition}
%%%%\newtheorem{definition}[theorem]{\sc Definition}
%%%%
%%%%\newtheorem{exercise}[theorem]{Exercise}

\begin{theorem}\label{T0.1}
Assume that $\alpha>0, \gamma>1, \beta>\frac{\gamma+1}{\gamma-1}$.
Then there exists a small $\tau_1>0$, such that for $\tau\in
[0,\tau_1)$, if $c$ crosses $c(\tau)$ from the direction of
to  a small amplitude periodic traveling wave solution of
(2.1), and the period of $(\check{u}^p(s),\check{w}^p(s))$ is
\[
\check{T}(c)=c\cdot \left[\frac{2\pi}{\omega(\tau)}+O(c-c(\tau))\right].
\]
\end{theorem}


\section{Figures \& Tables}

The output for figure is:

\begin{figure}[!h]
%\centering\includegraphics[width=2.5in]{figurename.eps}
%%%call your figure name in the place "figurename.eps"
\caption{Insert figure caption here}
\label{fig_sim}
\end{figure}

 An example of a double column floating figure using two subfigures.
 (The subfig.sty package must be loaded for this to work.)
 The subfigure \verb+\label+ commands are set within each subfloat command, the
 \verb+\label+ for the overall figure must come after \verb+\caption+.
 \verb+\hfil+ must be used as a separator to get equal spacing.
 The subfigure.sty package works much the same way, except \verb+\subfigure+ is
 used instead of \verb+\subfloat+.

%\begin{figure*}[!h]
%\centerline{\subfloat[Case I]\includegraphics[width=2.5in]{figurename.eps}%
%\label{fig_first_case}}
%\hfil
%\subfloat[Case II]{\includegraphics[width=2.5in]{figurename.eps}%
%\label{fig_second_case}}}
%\caption{Simulation results}
%\label{fig_sim}
%\end{figure*}

\vskip2pc

\noindent The output for table is:

\begin{table}[!h]
\caption{An Example of a Table}%%%Table caption goes here
\label{table_example}
\centering
\begin{tabular}{|c||c|}%%%The number of columns has to be defined here
\hline
One & Two\\ %%%% Table body
\hline
Three & Four\\%%%% Table body
\hline
\end{tabular}
\end{table}%%%End of the table

\section{Conclusion}
The conclusion text goes here.

\section*{Acknowledgment}

Insert the Acknowledgment text here.

% can use a bibliography generated by BibTeX as a .bbl file
% BibTeX documentation can be easily obtained at:
% http://www.ctan.org/tex-archive/biblio/bibtex/contrib/doc/

%\bibliographystyle{imaiai}
%\bibliography{sample}
%
% once the .bbl file has been generated then place the text in your article.

% To get the numbered reference style the author should use [numbib]
%as an option in the document class.  For example: \documentclass[numbib]{imaiai}

\begin{thebibliography}{9}
\bibitem{1}
Vicsek T., Czir\'{o}k A., Ben-Jacob E., Cohen I.: `Novel type of phase transition in a system of self-driven particles', \textit{Phys. Rev. Lett.}, 1995, \textbf{75}, pp.~1226--1229

\bibitem{2}
Jadbabaie A., Lin J., Morse A.S.: `Coordination of groups of
mobile autonomous agents using nearest neighbor rules', \textit{IEEE
Trans. Autom. Control}, 2003, \textbf{48}, pp.~988--1001

\bibitem{3}
Olfati-Saber R., Murray R.M.: `Consensus problems in networks
of agents with switching topology and time-delays', \textit{IEEE Trans.
Autom. Control}, 2004, \textbf{49}, pp.~1520--1533
\end{thebibliography}

\end{document}
