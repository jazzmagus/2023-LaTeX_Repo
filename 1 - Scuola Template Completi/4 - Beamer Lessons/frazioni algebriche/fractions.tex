\documentclass[aspectratio=169]{beamer}

\usetheme[progressbar=frametitle]{metropolis}
%\usepackage{italian, babel}
\setbeamertemplate{frame numbering}[none]
\useoutertheme{metropolis}
\useinnertheme{metropolis}
\usepackage{graphics}
\renewcommand{\familydefault}{\rmdefault}
%\usecolortheme{spruce}
\setbeamercolor{background canvas}{bg=white}

\title{Lezione di Matematica: Derivate}
\subtitle{sottotitolo lezione}
\date{\today}
\author{Diego Fantinelli}
\institute{Matematica per il Liceo}

\begin{document}
%\metroset{block=fill}

\begin{frame}
	\titlepage
\end{frame}

\begin{frame}
    
    \begin{abstract}
        The Dunning Kruger Effect is a cognitive bias that makes people believe
        they are smarter and more capable than they actually are. The effect is
        related to people's general inaptitude to recognize their lack of
        ability. To learn how this comes about and what you can do to avoid it
        from happening to you.
    \end{abstract}
\end{frame}

\begin{frame}{sommario}
  \tableofcontents
\end{frame}


% \begin{frame}
% \begin{block}{cos'è una \textbf{frazione algebrica}}

% \begin{quote}
% Si tratta di una \textbf{divisione} tra polinomi, espressa sotto forma di
% frazione.
% \end{quote}

% \begin{quote}
% \end{quote}

% \begin{quote}
% \emph{esempio:} {\((x + 1):(x^{2} - 1)\)}
% \end{quote}

% {\[\frac{\text{numeratore}}{\text{denominatore}}\rightarrow\frac{N(x)}{D(x)}\rightarrow\frac{x + 1}{x^{2} - 1}\]}

% \begin{itemize}
% \item
%   Il \emph{dividendo} prende il nome di \textbf{numeratore}
% \item
%   Il \emph{divisore} prende il nome di \textbf{denominatore}
% \end{itemize}
% \end{block}
% \end{frame}

\begin{frame}{prerequisiti 1}
\begin{itemize}
\item
  \textbf{Fattorizzazione polinomiale}
\item
  Indispensabile per poter semplificare una \emph{frazione algebrica}
\item
  \textbf{mcm} tra polinomi
\item
  per potersi riportare alla \textbf{forma normale} di una
  \emph{frazione algebrica}: \[\dfrac{N(x)}{D(x)} = \dfrac{\text{numeratore polinomiale}}{\text{denominatore polinomiale}}\]
\end{itemize}
\end{frame}

% \begin{frame}
% \begin{block}{Le tre cose da fare}
% \begin{enumerate}
% \tightlist
% \item
%   Ridurla in \textbf{forma normale}, nel caso si trattasse di
%   un'\emph{espressione con frazioni algebriche}
% \end{enumerate}

% \begin{itemize}
% \item
%   fattorizzare tutti in denominatori
% \item
%   denominatore comune
% \end{itemize}

% \begin{enumerate}
% \item
%   Determinare le \textbf{Condizioni di Esistenza}, C.E.
% \item
%   fattorizzare il numeratore, se possibile
% \item
%   semplificare, se possibile
% \end{enumerate}
% \end{block}
% \end{frame}

% \begin{frame}{Riduzione di frazioni algebriche allo stesso denominatore}
% \begin{quote}
% Per ridurre più frazioni allo stesso denominatore, bisogna trasformarle
% in frazioni \textbf{equivalenti} aventi tutte lo stesso denominatore
% (\textbf{M.C.D.} minimo comune denominatore).
% \end{quote}
% \end{frame}

% \begin{frame}
% \begin{block}{Il procedimento}
% È analogo a quello usato per ridurre più frazioni numeriche allo stesso
% denominatore:

% \begin{enumerate}
% \item
%   si semplificano le frazioni date;
% \item
%   le frazioni così ottenute sono quelle a cui si applicano direttamente
%   i passaggi successivi;
% \item
%   il denominatore comune cercato (\emph{minimo comune denominatore}) è
%   il \textbf{mcm} dei denominatori;
% \end{enumerate}
% \end{block}
% \end{frame}

% \begin{frame}
% \begin{block}{LaTeX}
% \end{block}

% \begin{block}{esempio 1}
% \begin{itemize}
% \item
%   fattorizziamo e \textbf{semplifichiamo}:
% \item
%   {\(\frac{(x + 1)}{(x^{2} - 1)} = \frac{(x + 1)}{(x + 1)(x - 1)} = \frac{1}{(x - 1)}\)}
% \item
%   determiniamo le \textbf{condizioni di esistenza}, C.E.
% \item
%   poniamo il denominatore \textbf{uguale a zero} determinare per quali
%   valori di {\(x\)} il \textbf{denominatore si annulla}:
% \item
%   {\(D(x) = 0\rightarrow x - 1 = 0\Rightarrow x = 1\)}
% \end{itemize}
% \end{block}
% \end{frame}

% \begin{frame}
% \begin{block}{scriviamo correttamente la soluzione}
% \begin{itemize}
% \item
%   Condizioni di esistenza:
% \item
%   {\(C.E.:x \neq 1\)}
% \item
%   Insieme di Definizione:
% \item
%   {\(IdD = \overset{insieme,di,definizione}{\overbrace{{\forall x \in \mathbb{R}:x \neq 1}}}\)}
% \end{itemize}

% \begin{frame}
% \begin{block}{questi li facciamo alla lavagna...}
% \begin{itemize}
% \item
%   {\[\frac{3x + 15}{x^{2} - 25}\]}
% \item
%   {\[\frac{2x^{4} - 18}{(x - 1)(2x - 3) - (x - 2)(x - 3)}\]}
% \item
%   {\[\frac{x}{x + 2} - \frac{8}{x^{2} - 4} + \frac{2}{x - 2}\]}
% \item
%   {\[- \frac{10}{x - 2} + \frac{x + 2}{x} + \frac{2}{3x^{2} - x}\]}
% \end{itemize}
% \end{block}
% \end{frame}

% \begin{frame}{Due regole d'oro}
% \begin{block}{1. \textbf{fattorizzare} i \textbf{denominatori}}
% \begin{block}{{\(\Rightarrow\)} serve a calcolare il \emph{minimo comun
% denominatore}}
% \end{block}
% \end{block}


\end{document}