\documentclass[10pt, aspectratio=169]{beamer}

\usetheme[progressbar=frametitle]{metropolis}
\usefonttheme{serif}
\setbeamertemplate{frame numbering}[none]
\useoutertheme{metropolis}
\useinnertheme{metropolis}
\usecolortheme{beaver}
\setbeamercolor{background canvas}{bg=white}

\usepackage[italian]{babel}
\usepackage[utf8]{inputenc}
\usepackage{cancel}
\usepackage{multicol}
\usepackage{parskip}[5pt]

\title{Correzione esercizi di matematica}
\subtitle{Classe 1AST}
\date{\today}
\author{prof. Diego Fantinelli}
\institute{ITET Pasini - Schio}


\titlegraphic{\hfill\includegraphics[height=15mm]{Pasini_logo.jpeg}}
\begin{document}
\metroset{block=fill}


\begin{frame}
	\titlepage
\end{frame}

\section{Esercizi}

\begin{frame}{Esercizio n. 147 a pag. 247}
\textcolor{orange}{\Large $2 a(a-2 b)+a b(3 a+1)-3 a^2(b-1)-a(2 a-3 b)$} \\[10pt]
\textbf{svolgimento}:\\[10pt]
\hrule
    \begin{itemize}
        \item si eseguono le moltiplicazioni termine a termine ottenendo:
        \item $2a^2 - 4ab + 3a^2 b+ ab -3a^2 b + 3a^2 - 2a^2 + 3ab$
        \item dopo aver opportunamente semplificato i monomi simili:
        \item $\cancel{2a^2} - \cancel{4ab} + \cancel{3a^2 b}+ \cancel{ab} - \cancel{3a^2 b} + 3a^2 - \cancel{2a^2} + \cancel{3ab}$
        \item si ottiene: \Large{$+3a^2$}
    \end{itemize}
\end{frame}


\end{document}