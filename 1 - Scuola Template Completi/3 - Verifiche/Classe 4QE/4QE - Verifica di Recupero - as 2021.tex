% ----------------------------------------------------------------------
% Template VERIFICA
% ----------------------------------------------------------------------
% 2020 di d!egofantinelli at jazzmagus@gmail.com
% ----------------------------------------------------------------------

% ---------------------------------- Preambolo
\documentclass[11pt, a4paper]{exam}
\usepackage[T1]{fontenc}
\usepackage{mdframed}
%\usepackage{nicefrac}
%\usepackage[applemac]{inputenc}
%\usepackage[utf8]{inputenc}
\usepackage[italian]{babel}
\usepackage[margin=1.3in]{geometry}
\usepackage{amsmath,amssymb, systeme}
\usepackage{multicol}
\usepackage{anyfontsize}
\usepackage{graphicx}
\usepackage{tikz}
\usepackage{upquote}
\usepackage{caption}
\usepackage[normalem]{ulem}
%\usepackage{fancyhdr}
\usepackage{float}
\usepackage{array}


\renewcommand{\questionshook}{%
    \setlength{\leftmargin}{0pt}%
}
\renewcommand{\choiceshook}{%
    \setlength{\leftmargin}{20pt}%
}

\newenvironment{sistema}% 
{\left\lbrace\begin{array}{@{}l@{}}}% 
{\end{array}\right.} 

% ---------------------------------- Intestazione
\newcommand{\class}{\huge {Verifica di Matematica}}
\newcommand{\term}{2� Quadrimestre}
\newcommand{\examnum}{Recupero insufficienze a.s. 2020/'21}
\newcommand{\examdate}{data: \uline{\hspace{6em}}}
\newcommand{\timelimit}{60 minuti}

\CorrectChoiceEmphasis{\color{red}}
\SolutionEmphasis{\color{red} \footnotesize}
\renewcommand{\solutiontitle}{\noindent\textbf{Soluzione:}\par\noindent}
% ---------------------------------- Intestazione

\pagestyle{headandfoot}
\firstpageheader{IIS "G.A. Remondini" - Bassano del Grappa (VI)}{}{\examdate}
\runningheader{\footnotesize VERIFICA di MATEMATICA}{}{Classe 4\string^QE}
\runningheadrule

\firstpagefooter{}{}{pag. \thepage\ di \numpages}
\runningfooter{}{}{pag. \thepage\ di \numpages}
\runningfootrule

% ---------------------------------- Punteggi
\pointpoints{punto}{\em punti}
\pointformat{[{\footnotesize \thepoints}]}
\bonuspointpoints{punto bonus}{\em punti bonus}
\bonuspointformat{[{\footnotesize \thepoints}]}
\pointsinrightmargin
\setlength{\rightpointsmargin}{.2cm}
\chqword{Esercizio}
\chpword{Punti}
\chbpword{Punti Bonus}
\chsword{Punteggio}
\chtword{Totale}


\printanswers

\begin{document}

% ---------------------------------- Title Page
\begin{center}
	\rule[2ex]{\textwidth}{0.5pt}\\
	{\huge{\bf \class}}\\[12pt]
	%{\huge -\, \term \, - }\\[8pt]
	{\huge -\, \examnum \, - }\\[8pt]
	\rule[2ex]{\textwidth}{0.5pt}\\
	\vspace{1cm}
\color{red} {\fontsize{50}{10}\selectfont {\bf SOLUZIONI}}\\
\end{center}
\vspace{3cm}
\begin{tabular*}{\textwidth}{l @{\extracolsep{\fill}} r @{\extracolsep{6pt}} l}
\textbf{} & \textbf{Cognome e Nome:} & \makebox[2.5in]{\hrulefill}\\
\textbf{} &&\\
\textbf{} & \textbf{Classe:} & \makebox[2.5in]{\Large{\bf 4 \string^ QE}}\\
\textbf{} &&\\
\textbf{} & Tempo a disposizione: & \makebox[2.5in]{\timelimit}
\end{tabular*}\\[3cm]

\vspace{3cm}
% ---------------------------------- Avvertenze

\noindent
\rule[2ex]{\textwidth}{0.2pt}
\textbf{Avvertenze}:
\begin{itemize}
	\item La presente Verifica di Recupero - che viene somministrata in modalit� IN PRESENZA - contiene \numquestions \; quesiti, per un totale di \numpoints \;punti;
	\item Per gli eventuali studenti che dovessero svolgere la prova in DDI, la webcam dovr� rimanere accesa per tutto il tempo della verifica (\timelimit), salvo impossibilit� concrete di connessione; il microfono rester� spento e verr� acceso soltanto per chiarimenti e domande, che saranno consentite negli ultimi 20 min di prova.
	\item E' vietato l'utilizzo di calcolatrici - scientifiche e non -, smartphone, tablet e altri dispositivi digitali, nonch� la consultazione di testi, appunti e siti web.

\end{itemize}
%\rule[2ex]{\textwidth}{0.2pt}
\vfill
\newpage

% ===================================== Esercizi ==========================



\begin{questions}

% ------------------------------------- Esercizio 1
\addpoints
\question[8]
Determina le soluzioni della seguente disequazione frazionaria di secondo grado:\\

\(\dfrac{x^2 + 2}{25 - x^2} > 0\)

{\footnotesize
\begin{solution}
	\( \left[\; S = \left\{ \forall x \in \mathbb{R}: -5 < x < 5 \right\} \; \right] \) 
\end{solution}
}
\vspace{.5cm}

% ------------------------------------- Esercizio 2

\addpoints
\question [8]
Dopo averla ricondotta alla {\em forma normale}, determina le soluzioni della seguente disequazione frazionaria di secondo grado:\\

{\em suggerimento:} Una disequazione fratta � in {\em forma normale} quando al primo membro della disequazione vi � un'unica frazione e al secondo membro \(0\)
\\

\(\dfrac{x - 2}{x} + \dfrac{2x - 3}{x - 1} \ge \dfrac{1}{x - x^2}\)

%\fillwithlines{0.5in}

{\footnotesize
\begin{solution}
	\( \left[\; S = \left\{ \forall x \in \mathbb{R}: x < 0 \, \vee \, x > 1 \right\} \; \right] \) 
\end{solution}
\vspace{.5cm}
}
%\pagebreak
% ------------------------------------- Esercizio 3
\addpoints
\question [12] Risolvi il seguente sistema di disequazioni frazionarie di secondo grado: \\

\begin{equation*}
	\begin{cases}
		\begin{aligned}
			\dfrac{1}{x}	 > \dfrac{1}{x - 3}\\[2ex]
  			3x -1 -2x^2  & < 0 \\[2ex]
			\dfrac{x^2 -x -2}{x} 	& > 0
		\end{aligned}
	\end{cases}
\end{equation*}

%\fillwithlines{0.75in}

{\footnotesize
\begin{solution}	
	\( \left[\; S = \left\{ \forall x \in \mathbb{R}: 0 < x < \dfrac{1}{2} \vee 2 < x < 3 \right\} \; \right] \) 
\end{solution}
}

\end{questions}
\vfill
\rule[2ex]{\textwidth}{1pt}\\
%\pagebreak
\begin{center}
{\bf Tabella dei punteggi}
\vspace{10pt}

\combinedgradetable[h][questions]
\end{center}
\vspace{4pt}
\footnotesize La sufficienza � fissata a 12 punti, ma potr� subire delle modifiche in fase di correzione, al fine di garantire la validit� della prova ove si verificassero prestazioni della classe sensibilmente lontane dalla media-classe stimata.

\end{document}
