% ----------------------------------------------------------------------
% Template VERIFICA
% ----------------------------------------------------------------------
% 2020 di d!egofantinelli at jazzmagus@gmail.com
% ----------------------------------------------------------------------

% ---------------------------------- Preambolo
\documentclass[11pt, a4paper]{exam}
\usepackage[utf8]{inputenc}
\usepackage[italian]{babel}
\usepackage[margin=1.3in]{geometry}
\usepackage{amsmath,amssymb}
\usepackage{multicol}
\usepackage{graphicx}
\usepackage{tikz}
\usepackage{caption}
%\usepackage{fancyhdr}
\usepackage{float}

% ---------------------------------- Intestazione
\newcommand{\class}{\uppercase {\scriptsize verifica di matematica}}
\newcommand{\term}{Periodo, Semestre o Corso}
\newcommand{\examnum}{Verifica numero:}
\newcommand{\examdate}{\emph{\today}}
\newcommand{\timelimit}{1 h e 45 min}
% ---------------------------------- Intestazione

\pagestyle{headandfoot}
\firstpageheader{\small Istituto Tecnico Industriale Statale "E. Fermi - Bassano del Grappa"}{}{\examdate}
\runningheader{}{}{}
\runningheadrule

\firstpagefooter{}{}{pag. \thepage\ di \numpages}
\runningfooter{\class}{}{\examdate}
\runningfooter{\class}{}{pag. \thepage\ di \numpages}
\runningfootrule

% ---------------------------------- Tabella dei punteggi
\pointpoints{punto}{\em punti}
\pointformat{[{\scriptsize \thepoints}]}
\pointsinrightmargin
\setlength{\rightpointsmargin}{.5cm}
\hpword{Punti:}
\vpword{Punti}
\vtword{Totale:}
\htword{Totale}
\vsword{Risultato}
\hsword{Risultato:}
\vqword{Problema}
\hqword{Esercizio:}

\begin{document}

% ---------------------------------- Title Page
\begin{center}
\rule[2ex]{\textwidth}{0.5pt}\\
{\huge{\bf Verifica di Matematica}}\\[8pt]
\rule[2ex]{\textwidth}{0.5pt}\\
\end{center}
\vspace{3cm}
\begin{tabular*}{\textwidth}{l @{\extracolsep{\fill}} r @{\extracolsep{6pt}} l}
\textbf{} & \textbf{Nome e Cognome:} & \makebox[2.5in]{\hrulefill}\\
\textbf{} &&\\
\textbf{} & \textbf{Classe:} & \makebox[2.5in]{\Large{\bf 1C}}\\
\textbf{} &&\\
\textbf{} & Tempo a disposizione: & \makebox[2.5in]{\timelimit}
\end{tabular*}\\[3mm]

\vspace{5cm}
% ---------------------------------- Avvertenze

\noindent
%\rule[2ex]{\textwidth}{0.2pt}
\textbf{Avvertenze}:
\begin{itemize}
	\item Este examen contiene \numquestions \;planteamientos que corresponde a \numpoints \;puntos de la valoración final. Tenga presente que no esta autorizada la comunicación con sus compañeros, ni el uso de ayudas computacionales (calculadora, celular, etc) y que resolver el pliego a l\'apiz implica renunciar a cualquier reclamación después de entregados los resultados.
	\item Este examen contiene \numquestions \;planteamientos que corresponde a \numpoints \;puntos de la valoración final. Tenga presente que no esta autorizada la comunicación con sus compañeros, ni el uso de ayudas 
	\item ni el uso de ayudas computacionales (calculadora, celular, etc) y que resolver el pliego a l\'apiz implica renunciar a cualquier reclamación después de entregados los resultados.

\end{itemize}
%\rule[2ex]{\textwidth}{0.2pt}
\vfill
\newpage

% ---------------------------------- Esercizi
\begin{questions}
\addpoints
\question[25]  En la funci\'on dada se garantiza que hay tres puntos de inflexi\'on ubicados en las raices o ceros y el $y_i$. Determine: Dominio, Rango, Tipo de funci\'on (inyectiva, sobreyectiva o biyectiva), paridad e intervalos en los que es creciente o decreciente y construya una aproximaci\'on gr\'afica del lugar geom\'etrico de la funci\'on.
\[
	f(x) = x^4 - 4x^2 + 4
\]
\emph{suggerimento:} Para determinar las raices o ceros resuelva la ecuaci\'on $x^4 - 4x^2 + 4 = 0$

\emph{osservazione:} La determinaci\'on de la paridad (en particular) debe estar apoyada en el procedimiento algebraico que la sustenta. Dado que conoce con exactitud los puntos de inflexi\'on, los intervalos en que la curva es creciente o decreciente deben darse exactamente definidos.

\addpoints
\question [12]Which of these famous physicists invented time?

\begin{oneparchoices}
 \choice Stephen Hawking 
 \choice Albert Einstein
 \choice Emmy Noether
 \choice This makes no sense
\end{oneparchoices}

\addpoints
\question [20]Which of these famous physicists published a paper on Brownian Motion?

\begin{checkboxes}
 \choice Stephen Hawking 
 \choice Albert Einstein
 \choice Emmy Noether
 \choice I don't know
\end{checkboxes}

%\addpoints
%\question[25] 
%En la funci\'on dada se garantiza que hay tres puntos de inflexi\'on ubicados en las raices o ceros y el $y_i$. Determine: Dominio, Rango, Tipo de funci\'on (inyectiva, sobreyectiva o biyectiva), paridad e intervalos en los que es creciente o decreciente y construya una aproximaci\'on gr\'afica del lugar geom\'etrico de la funci\'on.
%\[
%	f(x) = x^4 - 4x^2 + 4
%\]
%\emph{Suggerimento:} Para determinar las raices o ceros resuelva la ecuaci\'on $x^4 - 4x^2 + 4 = 0$
%
%\emph{Osservazione:} La determinaci\'on de la paridad (en particular) debe estar apoyada en el procedimiento algebraico que la sustenta. Dado que conoce con exactitud los puntos de inflexi\'on, los intervalos en que la curva es creciente o decreciente deben darse exactamente definidos.
%
%\addpoints
%\question[25]  En la funci\'on dada se garantiza que hay tres puntos de inflexi\'on ubicados en las raices o ceros y el $y_i$. Determine: Dominio, Rango, Tipo de funci\'on (inyectiva, sobreyectiva o biyectiva), paridad e intervalos en los que es creciente o decreciente y construya una aproximaci\'on gr\'afica del lugar geom\'etrico de la funci\'on.
%\[
%	f(x) = x^4 - 4x^2 + 4
%\]
%\emph{Sugerencia:} Para determinar las raices o ceros resuelva la ecuaci\'on $x^4 - 4x^2 + 4 = 0$
%
%\emph{Observaci\'on:} La determinaci\'on de la paridad (en particular) debe estar apoyada en el procedimiento algebraico que la sustenta. Dado que conoce con exactitud los puntos de inflexi\'on, los intervalos en que la curva es creciente o decreciente deben darse exactamente definidos.

\addpoints
\question[25] Risolvi Analiticamente
	\[
		\lim _{x \to 1} \frac{x^3 - 1}{x^2 - 1}
	\]

\addpoints
\question[25] Determine, si los hay, los n\'umeros en los que la funci\'on dada es discontinua.
	\[
		f(x) = (x^2 - 9x + 18)^{-1}
	\]

% ------------------------------------- Domanda suddivisa in parti con bonus
%\begin{questions}
%
%\question Given the equation \(x^n + y^n = z^n\) for \(x,y,z\) and \(n\) positive
%integers.
%\begin{parts}
%\part[5] For what values of $n$ is the statement in the previous question true?
%\vspace{\stretch{1}}
%
%\part[2 \half] For $n=2$ there's a theorem with a special name. What's that name?
%\vspace{\stretch{1}}
%
%
%\bonuspart[2 \half] What famous mathematician had an elegant proof for this theorem but there was
%not enough space in the margin to write it down?
%\vspace{\stretch{1}}
%
%\end{parts}

%\droptotalpoints
%
%\question[20] Compute \[\int_{0}^{\infty} \frac{\sin(x)}{x}\]
%
%\vspace{\stretch{1}}

% ------------------------------------- Domanda Bonus

\bonusquestion[30] Prove that the real part of all non-trivial zeros of the function 
\(\zeta(z)\) is \(\frac{1}{2}\)
\vspace{\stretch{1}}

\end{questions}

%\addpoints
%\question[25] Encuentre $D_xy$
%\[
%	y = \frac{(x + 1)^2}{3x - 4}
%\]
%
%\end{questions}

%\begin{table}[h]
%\centering
%	\caption{Relazione tra $f$ e $f'$.}
%	\def\arraystretch{1.5}
%	\begin{tabular}{c|c|r} % con |c| si definiscono le colonne e poi si separano le righe tra loro con \hline
%
%	{$f(x)$} & {$f'(x)$}\\ \hline
%	$x>0$ & La funzione $f(x)$ è \emph{crescente}. & prova 1\\
%	\hline
%	$x<0$ & La funzione $f(x)$ è \emph{decrescente}. & prova 2\\
%	\hline	
%	$x<0$ & La funzione $f(x)$ è \emph{costante}. & prova 3\\	
%	\end{tabular}
%\end{table}

%\vfill
%% ================================= Bloque de instrucciones
%\noindent
%\rule[2ex]{\textwidth}{0.5pt}
%\textbf{Istruzioni}: Este examen contiene \numquestions \;planteamientos que corresponde a \numpoints \;puntos de la valoración final. Tenga presente que no esta autorizada la comunicación con sus compañeros, ni el uso de ayudas computacionales (calculadora, celular, etc) y que resolver el pliego a l\'apiz implica renunciar a cualquier reclamación después de entregados los resultados.\\

% Tabla de calificaciones
%\vspace{8pt}

\noindent
\rule[2ex]{\textwidth}{1pt}

\begin{center}
{\bf Tabella dei punteggi}
\vspace{10pt}

\combinedgradetable[h][questions]
\end{center}



\end{document}
