% -------------------------------------------------------------
% Template HOMEWORK
% -------------------------------------------------------------
% 2020 by d!egofantinelli at jazzmagus@gmail.com
% -------------------------------------------------------------

% ---------------------------------- Preambolo
\documentclass[11pt, a4paper, landscape]{exam}
\usepackage[T1]{fontenc}
\usepackage{mdframed}
%\usepackage{nicefrac}
%\usepackage[applemac]{inputenc}
%\usepackage[utf8]{inputenc}
\usepackage[italian]{babel}
\usepackage{wrapfig}
\usepackage[margin=1in]{geometry}
\usepackage{amsfonts, amsthm, amsmath, amssymb}
\usepackage{multicol}
\usepackage{mathrsfs}
\usepackage[dvipsnames]{xcolor}
\usepackage[none]{hyphenat}
\usepackage{bbm}
\usepackage{graphicx}
\usepackage{tcolorbox}
\usepackage{tikz}
%\usepackage[dvipsnames]
%\usepackage{upquote}
\usepackage{caption}
\usepackage{float}

%\printanswers


% ---------------------------------- General Math Setup
%\newcommand{\numberset}{\mathbb}
%\newcommand{\N}{\numberset{N}}
%\newcommand{\Z}{\numberset{Z}}
%\newcommand{\Q}{\numberset{Q}}
%\newcommand{\R}{\numberset{R}}
%\definecolor{mypink3}{cmyk}{0, 0.7808, 0.4429, 0.1412}
%\newcommand{\ChoiceLabel}[1]{\hspace{-1.6em}\makebox[1.6em][l]{\textbf{#1.}}\ignorespaces}
%
%\newcommand{\WrongChoice}[1]{\choice \ChoiceLabel{#1}}
%\newcommand{\RightChoice}[1]{\correctchoice \ChoiceLabel{#1}}
%\newcommand{\Item}[1]{\hspace{-17pt}\makebox[17pt][l]{\textbf{#1.}}\ignorespaces}
%
%\renewcommand{\solutiontitle}{\noindent\textbf{Soluzione:}\par\noindent}
%
%\renewcommand{\questionshook}{%
%    \setlength{\leftmargin}{0pt}%
%}
%\renewcommand{\choiceshook}{%
%    \setlength{\leftmargin}{20pt}%
%}

% ---------------------------------- Intestazione
\newcommand{\class}{\LARGE {Esercizi assegnati}}
%%\newcommand{\term}{I Quadrimestre}
%\newcommand{\examnum}{Verifica numero: 1}
%\newcommand{\examdate}{11 dicembre 2020}
%\newcommand{\timelimit}{40 minuti}
\CorrectChoiceEmphasis{\color{red}}
\SolutionEmphasis{\color{red} \footnotesize}
\renewcommand{\solutiontitle}{\noindent\textbf{Soluzione:}\par\noindent}


% ---------------------------------- Intestazione
\pagestyle{headandfoot}
\footrule
\headrule
\lhead{\emph{suggerimenti per le vacanze: ripasso ed esercizi}}
\rhead{Classe 2\string^C}
\chead{}
%\rfoot{{\emph {\color{red} scadenza consegna: venerd� 11 dicembre 2020, ore 10:00}}}
\lfoot{pag. \thepage\ of \numpages}
\rfoot{\emph{prof. Diego Fantinelli}}


\begin{document}

% ---------------------------------- Title Page

\begin{minipage}{0.16\textwidth}
\includegraphics[width=0.56\linewidth]{logo_it} 
\end{minipage}
\begin{minipage}{0.8\textwidth}
\begin{flushright}
\textbf{ISTITUTO DI ISTRUZIONE SUPERIORE "G.A. REMONDINI"}\\ 
TECNICO PER IL TURISMO, LE BIOTECNOLOGIE SANITARIE E LA LOGISTICA\\
PROFESSIONALE PER ISERVIZI COMMERCIALI E SOCIO-SANITARI \\
Via Travettore, 33 - 36061 Bassano del Grappa (VI)\\
Codice Ministeriale - VIIS01700L\\	
\end{flushright}


\end{minipage}

%\begin{flushleft}
%	\begin{wrapfigure}{l}{0.28\textwidth}
%\includegraphics[width=0.36\linewidth]{logo_it} 
%\end{wrapfigure}
%\end{flushleft}
%
%\begin{flushright}
%	\textbf{ISTITUTO DI ISTRUZIONE SUPERIORE "G.A. REMONDINI"}\\ 
%TECNICO PER IL TURISMO, LE BIOTECNOLOGIE SANITARIE E LA LOGISTICA\\
%PROFESSIONALE PER ISERVIZI COMMERCIALI E SOCIO-SANITARI \\
%Via Travettore, 33 - 36061 Bassano del Grappa (VI)\\
%Codice Ministeriale - VIIS01700L\\
%\end{flushright}
\vspace{10pt}
\hrule
\vspace{5pt}

\noindent 
Esercizi necessari al recupero di matematica sul programma svolto nella classe seconda e utili alla preparazione per la classe terza.\\
\textit{Indirizzo Professionale - Servizi Commerciali e Servizi per la Sanit� e l'Assistenza Sociale} 

\begin{multicols}{2}

\paragraph{\large{\bf Suggerimenti per il ripasso}} 

\begin{enumerate}
	\item {\bf Prodotti Notevoli}\\ 
	rif.: UNIT� 5, Par. 3, pag. 224 | Vol. 1
	\item {\bf Fattorizzazione polinomiale: principali metodi}\\ 
	rif.: UNIT� 10, pag. 422 | Vol. 1
	\begin{itemize}
		\item Prodotti Notevoli
		\item Trinomio particolare di Secondo Grado
		\item Regola di Ruffini 
	\end{itemize}
	 
	\item {\bf Frazioni Algebriche:}\\ 
	rif.: UNIT� 1, pag. 4 | Vol. 2
	
	\begin{itemize}
		\item Semplificazione
		\item Determinazione delle Condizioni di Esistenza: C.E.
	\end{itemize}
		\item {\bf Equazioni Frazionarie di Primo Grado}\\ 
	rif.: UNIT� 2, pag. 35 | Vol. 2
\end{enumerate}

\vspace{4pt}

\hrule
\vspace{8pt}

\noindent
\textbf{Libri di testo}: "Nuova Matematica a Colori - ed. GIALLA"\\
Vol. 1 e 2 - Leonardo Sasso - Ed. Petrini

%
%\begin{tcolorbox}[colback=green!5!white,colframe=green!75!black]
%\textbf{Libri di testo}: "Nuova Matematica a Colori - ed. GIALLA"\\
%Vol.1 e 2 - L. Sasso - Ed. Petrini
%\end{tcolorbox}

\vfill\null
\columnbreak

\break

\paragraph{\large{\bf Esercizi suggeriti}}

\begin{enumerate}
	\item UNIT� 1: FRAZIONI ALGEBRICHE | Vol. 2
		\begin{itemize}
			\item pag. 21 \quad n. 83, 84, 94, 106, 128
			\item pag. 25 \quad n. 195, 196
			\item pag. 27 \quad n. 231, 232
		\end{itemize}
	\item UNIT� 2: EQUAZIONI FRATTE DI I GRADO | Vol. 2
	\begin{itemize}
		\item pag. 48 \quad n. 45, 50, 52, 57, 65, 66, 67
	\end{itemize}
\end{enumerate}
\hrule
\vspace{8pt}
\begin{tcolorbox}[colback=orange!5!white,colframe=orange!75!black]
\emph{argomenti facoltativi:}
\begin{enumerate}
	\item UNIT�: SISTEMI LINEARI  
	\begin{itemize}
		\item pag. 182 \quad n. 45, 47, 49, 51, 54, 55; 
		\item pag. 184 \quad n. 97, 98, 99, 100
	\end{itemize}
	
	\item UNIT�: DISEQUAZIONI E SISTEMI DI DISEQUAZIONI
	\begin{itemize}
		\item pag. 48 \quad n. 45, 50, 52, 57, 65, 66, 67
		\item pag. 353 \quad n. 17, 18, 19, 21, 22, 23, 30, 31
	\end{itemize}
		
\end{enumerate}\end{tcolorbox}

\end{multicols}

\end{document}
