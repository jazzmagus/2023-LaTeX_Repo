% ----------------------------------------------------------------------
% Template VERIFICA
% ----------------------------------------------------------------------
% 2020 di d!egofantinelli at jazzmagus@gmail.com
% ----------------------------------------------------------------------

% ---------------------------------- Preambolo
\documentclass[11pt, a4paper]{exam}
\usepackage[T1]{fontenc}
\usepackage{mdframed}
%\usepackage{nicefrac}
%\usepackage[applemac]{inputenc}
%\usepackage[utf8]{inputenc}
\usepackage[italian]{babel}
\usepackage[margin=1.3in]{geometry}
\usepackage{amsmath,amssymb}
\usepackage{multicol}
\usepackage{graphicx}
\usepackage{tikz}
\usepackage{upquote}
\usepackage{caption}
%\usepackage{fancyhdr}
\usepackage{float}

%\printanswers

% ---------------------------------- Command
\renewcommand{\questionshook}{%
    \setlength{\leftmargin}{0pt}%
}
\renewcommand{\choiceshook}{%
    \setlength{\leftmargin}{20pt}%
}

\newcommand{\class}{\huge {Verifica di Matematica}}
\newcommand{\term}{Recupero Primo Quadrimestre}
%\newcommand{\examnum}{Verifica numero: 1}
\newcommand{\examdate}{22 febbraio 2021}
\newcommand{\timelimit}{50 minuti}

\CorrectChoiceEmphasis{\color{red}}
\SolutionEmphasis{\color{red} \footnotesize}
\renewcommand{\solutiontitle}{\noindent\textbf{Soluzione:}\par\noindent}
% ---------------------------------- Headers and Footers

\pagestyle{headandfoot}
\firstpageheader{IIS "G. A. Remondini" - Bassano del Grappa (VI)}{}{\examdate}
\runningheader{\footnotesize VERIFICA di Recupero di MATEMATICA}{}{Classe 2\string^C}
\runningheadrule

\firstpagefooter{}{}{pag. \thepage\ di \numpages}
\runningfooter{}{}{pag. \thepage\ di \numpages}
\runningfootrule

% ---------------------------------- Punteggi
\pointpoints{punto}{\em punti}
\pointformat{[{\footnotesize \thepoints}]}
\bonuspointpoints{punto bonus}{\em punti bonus}
\bonuspointformat{[{\footnotesize \thepoints}]}
\pointsinrightmargin
\setlength{\rightpointsmargin}{.2cm}
\chqword{Esercizio}
\chpword{Punti}
\chbpword{Punti Bonus}
\chsword{Punteggio}
\chtword{Totale}

\begin{document}

% ---------------------------------- Title Page
\begin{center}
\rule[2ex]{\textwidth}{0.5pt}\\
{\huge{\bf \class}}\\[12pt]
{\huge -\, \term \, - }\\[8pt]
\rule[2ex]{\textwidth}{0.5pt}\\
\end{center}
\vspace{3cm}
\begin{tabular*}{\textwidth}{l @{\extracolsep{\fill}} r @{\extracolsep{6pt}} l}
\textbf{} & \textbf{Nome e Cognome:} & \makebox[2.5in]{\hrulefill}\\
\textbf{} &&\\
\textbf{} & \textbf{Classe:} & \makebox[2.5in]{\Large{\bf 2 \string^ C}}\\
\textbf{} &&\\
\textbf{} & Tempo a disposizione: & \makebox[2.5in]{\timelimit}
\end{tabular*}\\[3cm]
\vspace{5cm}

% ---------------------------------- Avvertenze

\noindent
%\rule[2ex]{\textwidth}{0.2pt}
\textbf{Avvertenze}:
\begin{itemize}
	\item La presente Verifica di Recupero - che viene somministrata in modalit� IN PRESENZA - contiene \numquestions \;quesiti, per un totale di \numpoints \;punti.
%	\item La webcam dovr� rimanere accesa per tutto il tempo della verifica (\timelimit), salvo impossibilit� concrete di connessione; il microfono rester� spento e verr� acceso soltanto per chiarimenti e domande, che saranno consentite negli ultimi 20 min di prova.
	\item E' vietato l'utilizzo di calcolatrici scientifiche, smartphone, tablet e altri dispositivi digitali, nonch� la consultazione di testi, appunti e siti web.

\end{itemize}%\rule[2ex]{\textwidth}{0.2pt}
\vfill
\newpage

% ================================== Esercizi

% ---------------------------------- Esercizio 1
\begin{questions}

\addpoints
\question [5]Quale tra le risposte elencate � la {\em fattorizzazione} del seguente trinomio di secondo grado?\\ \[P(x) = 4x^2 - 12x + 9\]

\begin{choices}
\setlength{\leftmargin}{0pt}
 \choice \((x + 3)^2\) 
 \choice \((2x - 3) \cdot (2x + 3)\) 
 \CorrectChoice \((2x - 3)^2\) 
 \choice \((x - 3)^3\)
\end{choices}
\vspace{.5cm}

\addpoints
\question
Semplifica le seguenti espressioni algebriche, utilizzando il metodo di raccoglimento indicato:\\
\begin{parts}
\part[5] Raccoglimento totale: \(2a^4b^3 - 3a^3b^2 + 5a^2b^4\)\\
{\footnotesize
\begin{solution}
	\(a^2b^2(2a^2b - 3a + 5b^2)\)
\end{solution}
}
\vspace{.5cm}

\part[5] Raccoglimento parziale: \(t^5 - 5t^4 + t - 5\)\\
{\footnotesize
\begin{solution}
	\((t - 5)(t^4 + 1)\)
\end{solution}
}

\end{parts}
\vspace{.5cm}

% ---------------------------------- Esercizio 2
\question [6] Cosa significa {\em fattorizzare} un polinomio?
\fillwithlines{0.5in}

{\footnotesize
\begin{solution}
Significa riscriverlo come prodotto di polinomi di grado inferiore.
\end{solution}
}
\vspace{.5cm}

\addpoints
\question [4] Quali dei seguenti gruppi di monomi sono {\em simili}?\\

\begin{oneparchoices}
 \choice $x^3y^4, \quad x^3y^2$ 
 \choice $\dfrac{1}{6}rst, \quad \dfrac{1}{6}rs^2t$
 \CorrectChoice $\dfrac{1}{2}a^3bc^2, \quad 3a^3bc^2$
 \choice $abc,\quad 2abc, \quad 3a^2bc$
\end{oneparchoices}
\vspace{.5cm}

%
%\addpoints
%\question [20]Quali dei seguenti monomi sono {\em simili}?
%
%\begin{checkboxes}
% \choice Stephen Hawking \quad \dotfill
% \correctchoice Albert Einstein \quad \dotfill
% \choice Emmy Noether \quad \dotfill
% \choice I don't know \quad \dotfill
%\end{checkboxes}


% ---------------------------------- Esercizio 3

\addpoints
\question
Esegui i seguenti {\em Prodotti Notevoli}:\\
\begin{parts}
\part[3] \((6x - 5y) \cdot (6x + 5y)\)\\
{\footnotesize
\begin{solution}
	\(36x^2 - 25y^2\)
\end{solution}
}
\vspace{.5cm}

\part[3] \((2a - 3b)^2\)\\
{\footnotesize
\begin{solution}
	\((4a^2 - 12ab + 9b^2)\)
\end{solution}
}

\end{parts}
\vspace{.5cm}

%\addpoints
%\question [5]Quale delle seguenti affermazioni � VERA?\\
%\begin{choices}
%\setlength{\leftmargin}{0pt}
% \CorrectChoice \( 4x^2 - 16x + 16\) � un QUADRATO di BINOMIO. 
% \choice Il doppio del prodotto tra il quadrato di $x$ e il quadrato della somma di $x$ con $y$.
% \choice \( (x^2 - 1) \cdot (x^2 + 1)\) non � un {\em prodotto notevole}
% \CorrectChoice Il doppio del prodotto tra il quadrato di $x$ e la somma dei quadrati di $x$ e $y$.
%\end{choices}
%\vspace{.5cm}

% ---------------------------------- Esercizio 4

\addpoints
\question [7]Esegui la seguente divisione utilizzando la Regola di Ruffini, scrivendo alla fine la scomposizione ottenuta: \(A(x) = B(x) \cdot Q(x) + R\)
\[ (2x^4 - 5x^3 + 2) : (x - 1)\]
\fillwithlines{0.5in}

{\footnotesize
\begin{solution}
	\(Q(x) = 2x^3 - 3x^2 - 3x - 3, \quad R = -1\).
\end{solution}
}
% ------------------------------------- Domanda Bonus

%\bonusquestion[5] {\em Esercizio facoltativo:}
%
%Determina il M.C.D. e il m.c.m. fra i seguenti monomi: \[ 9a^2b^4c, \quad 3ac^4, \quad 6bc^2.\]
%{\footnotesize
%\begin{solution}
%	\(M.C.D. = 3c\), \quad \(m.c.m. = 18a^2b^4c^4\)
%\end{solution}
%}

\end{questions}
%\vfill
%\noindent
%\rule[2ex]{\textwidth}{1pt}
\pagebreak
\begin{center}
{\bf Tabella dei punteggi}
\vspace{10pt}

\combinedgradetable[h][questions]
\end{center}
\vspace{4pt}
\footnotesize La sufficienza � fissata a 20 punti, ma potr� subire delle modifiche in fase di correzione, al fine di garantire la validit� della prova anche nel caso in cui si riscontrino prestazioni della classe sensibilmente lontane dalla media-classe stimata.

\end{document}
