% ----------------------------------------------------------------------
% Template VERIFICA
% ----------------------------------------------------------------------
% 2020 di d!egofantinelli at jazzmagus@gmail.com
% ----------------------------------------------------------------------


% ---------------------------------- Preambolo
\documentclass[11pt, a4paper]{exam}
\usepackage[T1]{fontenc}
\usepackage{mdframed}
\usepackage{enumitem}
%\usepackage{nicefrac}
%\usepackage[applemac]{inputenc}
%\usepackage[utf8]{inputenc}
\usepackage[italian]{babel}
\usepackage[margin=1.3in]{geometry}
\usepackage{amsmath, amssymb}
\usepackage{multicol}
\usepackage{graphicx}
\usepackage{tikz}
\usepackage{upquote}
\usepackage{caption}
%\usepackage{fancyhdr}
\usepackage{float}

\renewcommand{\questionshook}{%
    \setlength{\leftmargin}{0pt}%
}
\renewcommand{\choiceshook}{%
    \setlength{\leftmargin}{20pt}%
}
% ---------------------------------- Intestazione
\newcommand{\class}{\huge {Verifica di Matematica}}
\newcommand{\term}{I Quadrimestre}
\newcommand{\examnum}{Verifica numero: 2}
\newcommand{\examdate}{18 dicembre 2020}
\newcommand{\timelimit}{50 minuti}
\CorrectChoiceEmphasis{\color{red}}
\SolutionEmphasis{\color{red} \footnotesize}
\renewcommand{\solutiontitle}{\noindent\textbf{Soluzione:}\par\noindent}


% ---------------------------------- Intestazione

\pagestyle{headandfoot}
\firstpageheader{ISS "G. A. Remondini" - Bassano del Grappa (VI)}{}{\examdate}
\runningheader{\footnotesize VERIFICA di MATEMATICA}{}{Classe 2\string^C}
\runningheadrule

\firstpagefooter{}{}{pag. \thepage\ di \numpages}
\runningfooter{}{}{pag. \thepage\ di \numpages}
\runningfootrule

% ---------------------------------- Punteggi
\pointpoints{punto}{\em punti}
\pointformat{[{\footnotesize \thepoints}]}
\bonuspointpoints{punto bonus}{\em punti bonus}
\bonuspointformat{[{\footnotesize \thepoints}]}
\pointsinrightmargin
\setlength{\rightpointsmargin}{.2cm}
\chqword{Esercizio}
\chpword{Punti}
\chbpword{Punti Bonus}
\chsword{Punteggio}
\chtword{Totale}


\printanswers

\begin{document}

% ---------------------------------- Title Page
\begin{center}
\rule[2ex]{\textwidth}{0.5pt}\\
{\huge{\bf \class}}\\[20pt]
{\huge{ \term \quad -\quad\examnum}}\\[8pt]
\rule[2ex]{\textwidth}{0.5pt}\\
\end{center}
\vspace{3cm}
\begin{tabular*}{\textwidth}{l @{\extracolsep{\fill}} r @{\extracolsep{6pt}} l}
\textbf{} & \textbf{Nome e Cognome:} & \makebox[2.5in]{\hrulefill}\\
\textbf{} &&\\
\textbf{} & \textbf{Classe:} & \makebox[2.5in]{\Large{\bf 2 \string^ C}}\\
\textbf{} &&\\
\textbf{} & Tempo a disposizione: & \makebox[2.5in]{\timelimit}\\
\textbf{} &&\\
\textbf{} &&\\
\textbf{} &&\\
\textbf{} & {\em prof.:} & \makebox[2.5in]{\em Diego Fantinelli}
\end{tabular*}\\

\vspace{5cm}
% ---------------------------------- Avvertenze

\noindent
%\rule[2ex]{\textwidth}{0.2pt}
\textbf{Avvertenze}:
\begin{itemize}
	\item La presente Verifica - che viene somministrata in modalit� DDI - contiene \numquestions \; quesiti, per un totale di \numpoints \;punti, di cui uno facoltativo di \numbonuspoints \;punti, che verr� conteggiato soltanto se verranno svolti anche tutti i precedenti.
	\item La webcam dovr� rimanere accesa per tutto il tempo della verifica (\timelimit), salvo impossibilit� concrete di connessione; il microfono rester� spento e verr� acceso soltanto per chiarimenti e domande, che saranno consentite negli ultimi 20 min di prova.
	\item E' vietato l'utilizzo di calcolatrici scientifiche, smartphone, tablet e altri dispositivi digitali, nonch� la consultazione di testi, appunti e siti web.
	\item La verifica dovr� essere consegnata in formato digitale (pdf, jpeg, png, etc.) e dovr� essere ben leggibile (si consiglia l'inquadratura verticale).

\end{itemize}
%\rule[2ex]{\textwidth}{0.2pt}
\vfill
\newpage

% ---------------------------------- Esercizi
\begin{questions}

\addpoints
\question[12]
Fattorizza i seguenti polinomi utilizzando i Prodotti Notevoli:
\begin{multicols}{2}
\begin{parts}
\part
\(81x^8 - 1\)

\begin{solution}
	\( (9x^4 + 1)(9x^4 - 1)\)
\end{solution} 

\vspace{.3cm}
\part
\(25y^2 - 20xy + 4x^2\)

\begin{solution}
	\( (5y^2 - 2x)^2 \)
\end{solution}

\vspace{.3cm}
\part
\(-1 + t^2\)

\begin{solution}
	\( (t + 1)(t -1) \) 
\end{solution}

\vspace{.3cm}
\part
\(125x^6 - 75x^4y + 15x^2y^2 - 27y^3\)

\begin{solution}
	\( (5x^2 - 3y)^3 \)
\end{solution}

\vspace{.3cm}
\part
\(36x^2 + y^2 + 81z^4 + 12xy - 54yz^2 + xz^2 \)

\begin{solution} 
	\( ( 6x - y + 3z^2)^2\)
\end{solution}

\vspace{.3cm}
\part
\(a^6 - b^{12} \)

\begin{solution} 
	\( (a^3 - b^6)(a^3 + b^6)\)
\end{solution}

\vspace{.3cm}
\end{parts}
\end{multicols}

\question
Rispondi in modo chiaro e sintetico alle seguenti domande:

\begin{parts}
\part[4]
Dimostra la seguente uguaglianza: \(A^3 + 3A^2B + 3AB^2 + B^3 = (A + B)^3\)
%\fillwithlines{0.5in}

\vspace{3pt}
\begin{solution}
\begin{align*} 
(A - B)^3 & = (A - B)^2 \cdot (A - B)\\ 
& = (A^2 - 2AB + B^2) \cdot (A - B)\\
& = A^3 - 2A^2B + AB^2 - A^2B + 2AB^2 - B^3 \\
&= A^3 + 3A^2B + 3AB^2 + B^3
\end{align*}
\end{solution}
% TODO Mancano le soluzioni

\part[4]

Scomponi in fattori i seguenti \emph {Trinomi Particolari}:\\
{\footnotesize \emph{Suggerimento:} con una rapida verifica potrai controllare se la scomposizione adottata � corretta.
}
\begin{enumerate}[label=\alph*.]
\begin{multicols}{2}
		\item \(x^2 - x - 6\)
			\begin{solution}
			\( (x - 3)(x + 2) \)	
			\end{solution}
		\item \(m^2 - 7m  + 10\)
		\begin{solution}
			\( (m - 5)(m - 2) \)	
			\end{solution}
		\item \(a^2 + 5a - 50\)
		\begin{solution}
			\( (a + 5)(a + 10) \)	
			\end{solution}
		\item \(x^2 - 11x + 30\)
		\begin{solution}
			\( (x - 5)(x - 6) \)	
			\end{solution}
\end{multicols}
\end{enumerate}

\end{parts}

%\begin{solution}
%{\bf Fattorizzare} un polinomio significa scriverlo come prodotto di fattori (per l'appunto) irriducibili, ovviamente di grado inferiore.\\ {\em Un polinomio � {\bf irriducibile} quando non pu� essere scritto come prodotto di due o pi� fattori di grado inferiore.}
%\end{solution}

%\fillwithlines{.5in}

\vspace{3pt}
% --------------------------- Esercizio 3
\addpoints
\question [2] \((x + 4)(x - 7)\) � la fattorizzazione di uno dei seguenti polinomi, quale?

\begin{choices}
\setlength{\leftmargin}{0pt}
\begin{multicols}{2}
 \choice \(x^2 - 2x + 1\) 
 \choice \(2x^2 - 2x + 3\)
 \CorrectChoice \(x^2 - 3x - 28\)
 \choice \(x^2 + 6x - 1\)
 \end{multicols}
\end{choices}
\vspace{3pt}

\addpoints
\question Semplifica le seguenti espressioni?\\
{\footnotesize \emph{Suggerimento:} Cerca di individuare la presenza di eventuali Prodotti Notevoli che facilitano i calcoli.}

\begin{parts}
\part [4]
	\(  (x + 2)^{2} + (x + 2)(x - 2) - 2 \cdot (x - 1)\)

\vspace{5pt}
\begin{solution}
			\( 2x^2 - 2x + 2 = 2 \cdot (x^2 -x + 1) \)	
\end{solution}

\part [4]	
	\(  6x - 6x \cdot (3x - 1) + (3x - 2)^{2} + (3x - 2)(3x + 2) \)
	
	\vspace{5pt}
\begin{solution}
			\( 0 \)	
\end{solution}
\end{parts}
\vspace{6pt}
%%\emph{Suggerimento:} leggere con molta attenzione il testo delle risposte perch� le differenze potrebbero essere minime.
%\begin{choices}
% \CorrectChoice Un \emph{numero decimale periodico semplice} di periodo 9 coincide esattamente con il numero intero successivo.
% \choice La parte intera di un numero decimale � quella che precede la virgola.
% \CorrectChoice Una frazione che ha 100 al denominatore genera un numero decimale finito.
% \choice La frazione generatrice di un \emph{numero periodico} ha al denominatore un numero le cui cifre sono tutte uguali a 9.
%\end{choices}
%\vspace{5pt}

% --------------------------- Esercizio 4
\bonusquestion[4] {\em Esercizio facoltativo:} Semplifica la seguente espressione: 

\((2x^2 + y)^2 - (2x^2 + y) \cdot (2x^2 - y) - 4(x^2)y\)

\begin{solution}

\( 2y^2\)

\end{solution}

\end{questions}

\vspace{5pt}

\noindent
\rule[2ex]{\textwidth}{1pt}

\begin{center}
\vfill
{\bf Tabella dei punteggi}\\
\vspace{10pt}
\combinedgradetable[h][questions]
\end{center}
\vspace{4pt}
\footnotesize La sufficienza � fissata a 20 punti, ma potrebbe subire delle modifiche in fase di correzione, al fine di garantire la validit� della prova anche in caso di andamenti troppo scostanti della media-classe.
\end{document}
