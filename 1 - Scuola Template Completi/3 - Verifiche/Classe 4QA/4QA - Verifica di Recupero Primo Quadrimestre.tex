% ----------------------------------------------------------------------
% Template VERIFICA
% ----------------------------------------------------------------------
% 2020 di d!egofantinelli at jazzmagus@gmail.com
% ----------------------------------------------------------------------

% ---------------------------------- Preambolo
\documentclass[11pt, a4paper]{exam}
\usepackage[T1]{fontenc}
\usepackage{mdframed}
%\usepackage{nicefrac}
%\usepackage[applemac]{inputenc}
%\usepackage[utf8]{inputenc}
\usepackage[italian]{babel}
\usepackage[margin=1.3in]{geometry}
\usepackage{amsmath,amssymb}
\usepackage{multicol}
\usepackage{graphicx}
\usepackage{tikz}
\usepackage{upquote}
\usepackage{caption}
%\usepackage{fancyhdr}
\usepackage{float}


\renewcommand{\questionshook}{%
    \setlength{\leftmargin}{0pt}%
}
\renewcommand{\choiceshook}{%
    \setlength{\leftmargin}{20pt}%
}

%\printanswers

% ---------------------------------- Intestazione
\newcommand{\class}{\huge {Verifica di Recupero di Matematica}}
\newcommand{\term}{1� Quadrimestre}
\newcommand{\examnum}{Verifica numero: 1}
\newcommand{\examdate}{22 febbraio 2021}
\newcommand{\timelimit}{40 minuti}

\CorrectChoiceEmphasis{\color{red}}
\SolutionEmphasis{\color{red} \footnotesize}
\renewcommand{\solutiontitle}{\noindent\textbf{Soluzione:}\par\noindent}
% ---------------------------------- Intestazione

\pagestyle{headandfoot}
\firstpageheader{IIS "G. A. Remondini" - Bassano del Grappa (VI)}{}{\examdate}
\runningheader{\footnotesize VERIFICA di MATEMATICA}{}{Classe 4\string^QA}
\runningheadrule

\firstpagefooter{}{}{pag. \thepage\ di \numpages}
\runningfooter{}{}{pag. \thepage\ di \numpages}
\runningfootrule

% ---------------------------------- Punteggi
\pointpoints{punto}{\em punti}
\pointformat{[{\footnotesize \thepoints}]}
\bonuspointpoints{punto bonus}{\em punti bonus}
\bonuspointformat{[{\footnotesize \thepoints}]}
\pointsinrightmargin
\setlength{\rightpointsmargin}{.2cm}
\chqword{Esercizio}
\chpword{Punti}
\chbpword{Punti Bonus}
\chsword{Punteggio}
\chtword{Totale}

\begin{document}

% ---------------------------------- Title Page
\begin{center}
\rule[2ex]{\textwidth}{0.5pt}\\
{\huge{\bf \class}}\\[12pt]
{\huge - Recupero Primo Quadrimestre -}\\[8pt]
\rule[2ex]{\textwidth}{0.5pt}\\
\end{center}
\vspace{3cm}
\begin{tabular*}{\textwidth}{l @{\extracolsep{\fill}} r @{\extracolsep{6pt}} l}
\textbf{} & \textbf{Nome e Cognome:} & \makebox[2.5in]{\hrulefill}\\
\textbf{} &&\\
\textbf{} & \textbf{Classe:} & \makebox[2.5in]{\Large{\bf 4 \string^ QA}}\\
\textbf{} &&\\
\textbf{} & Tempo a disposizione: & \makebox[2.5in]{\timelimit}
\end{tabular*}\\[3cm]

\vspace{5cm}
% ---------------------------------- Avvertenze

\noindent
%\rule[2ex]{\textwidth}{0.2pt}
\textbf{Avvertenze}:
\begin{itemize}
	\item La presente Verifica di Recupero - che viene somministrata in modalit� DDI - contiene \numquestions \; quesiti, per un totale di \numpoints \;punti;
	\item La webcam dovr� rimanere accesa per tutto il tempo della verifica (\timelimit), salvo impossibilit� concrete di connessione; il microfono rester� spento e verr� acceso soltanto per chiarimenti e domande, che saranno consentite negli ultimi 20 min di prova.
	\item E' vietato l'utilizzo di calcolatrici scientifiche, smartphone, tablet e altri dispositivi digitali, nonch� la consultazione di testi, appunti e siti web.

\end{itemize}
%\rule[2ex]{\textwidth}{0.2pt}
\vfill
\newpage

% ---------------------------------- Esercizi
\begin{questions}



% ------------------------------------- Esercizio 1
\addpoints
\question
Semplifica le seguenti espressioni di secondo grado {\em pure} o riconducibili a {\em pure}:\\
\begin{parts}
\part[4]
\((x + 1)^2 = 2\)\\
{\footnotesize
\begin{solution}
	\([-1 \pm \sqrt{2}]\)
\end{solution}
}
\vspace{.5cm}

\part[4]
\(\dfrac{1}{3}(x - 2)^2 = 3\)\\
{\footnotesize
\begin{solution}
	\([-1, 5]\)
\end{solution}
}

\end{parts}
\vspace{.5cm}

% ------------------------------------- Esercizio 2

\addpoints
\question
Semplifica le seguenti espressioni di secondo grado {\em spurie} o riconducibili a {\em spurie}:\\
\begin{parts}
\part[4]
\(2x^2 + 3x = 0\)\\
{\footnotesize
\begin{solution}
	\(\left[- \dfrac{1}{2}, \, 0 \right];\) 
\end{solution}
}
\vspace{.5cm}

\part[4]
\(\left(\dfrac{1}{2}x - 1 \right) \cdot \left(\dfrac{1}{3}x - 2 \right) = 2\)\\
{\footnotesize
\begin{solution}
	\([0, 8]\)
\end{solution}
}

\end{parts}
\vspace{.5cm}

% ------------------------------------- Esercizio 3
\addpoints
\question [4] Quale delle seguenti affermazioni riguardanti il {\em discriminante} � VERA?\\
\begin{choices}
\setlength{\leftmargin}{0pt}
 \choice Il discriminante si calcola nel seguente modo: \(\Delta = \sqrt{b^2 - 4ac} \). 
 \choice Se il discriminante � negativo \((\Delta < 0)\) la parabola � rivolta verso il basso.
 \choice Quando il discriminante � nullo l'equazione di secondo grado ha una sola soluzione reale.
 \CorrectChoice Il valore del discriminante permette di stabilire se l'equazione ammette soluzioni reali oppure no.
\end{choices}
\vspace{.5cm}

% ------------------------------------- Esercizio 4

\question 
Determina le soluzioni delle seguenti equazione di secondo grado e utilizzale per fattorizzare i trinomi:\\

\begin{parts}
\part[5]
\(x^2 + x - 30 = 0\)
\fillwithlines{0.75in}
{\footnotesize
\begin{solution}

	\([-5, 6]\)\\
	
	\(x^2 + x - 30 = (x + 5) \cdot (x - 6)\)
\end{solution}
}
\vspace{.5cm}

\part[5]
\(2x^2 + 3x - 9 = 0\)
\fillwithlines{0.75in}
{\footnotesize
\begin{solution}

	\(\left[-3, \dfrac{3}{2}\right]\)\\
	
	\(2x^2 + 3x - 9 = 2 \cdot (x + 3) \cdot \left(x - \dfrac{3}{2}\right)\)
\end{solution}
}
\end{parts}
\end{questions}
\vfill

\pagebreak
\begin{center}
{\bf Tabella dei punteggi}
\vspace{10pt}

\combinedgradetable[h][questions]
\end{center}
\vspace{4pt}
\footnotesize La sufficienza � fissata a 20 punti, ma potr� subire delle modifiche in fase di correzione, al fine di garantire la validit� della prova anche nel caso in cui si riscontrino prestazioni della classe sensibilmente lontane dalla media-classe stimata.

\end{document}
